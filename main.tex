% Monograph LaTeX Template for UFSC based on:
%
% 1. https://github.com/royertiago/tcc
% 2. http://portal.bu.ufsc.br/normalizacao/
% 3. https://github.com/AdrianoRuseler/abntex2-ufsc
%
% When the bibliography includes a cyclic reference to another bibliography,
% you need to run `pdfTeX` 5 times on the following order:
% 1. `pdfTeX`,
% 2. `biber`,
% 3. `pdfTeX`
% 4. `pdfTeX`
% 5. `pdfTeX`
% 6. `biber`
% 7. `pdfTeX`

% Allows you to write your thesis both in English and Portuguese
% https://tex.stackexchange.com/questions/5076/is-it-possible-to-keep-my-translation-together-with-original-text
\newif\ifenglish\englishfalse

% Uncomment the line `\englishtrue` to set the document default language to English
% \englishtrue

% https://tex.stackexchange.com/questions/131002/how-to-expand-ifthenelse-so-that-it-can-be-used-in-parshape
\newcommand{\lang}[2]{\ifenglish#1\else#2\fi}

% https://tex.stackexchange.com/questions/385895/how-to-make-passoptionstopackage-add-the-option-as-the-last
\ifenglish
    \PassOptionsToPackage{brazil,main=english,spanish,french}{babel}
\else
    \PassOptionsToPackage{main=brazil,english,spanish,french}{babel}
\fi

% Simple alias for English and Portuguese words
\newcommand{\brazilword}[1]{\foreignlanguage{brazil}{#1}}
\newcommand{\englishword}[1]{\foreignlanguage{english}{#1}}

% Allow you to write `Evandro's house` in latex as `Evandro\s house` instead of `Evandro\textquotesingle{}s house`
% https://tex.stackexchange.com/questions/31091/space-after-latex-commands
\newcommand{\s}[0]{\textquotesingle{}s\xspace}
\newcommand{\q}[0]{\textquotesingle{}\xspace}

% Uncomment the following line if you want to use other biblatex settings
% \PassOptionsToPackage{style=numeric,repeatfields=true,backend=biber,backref=true,citecounter=true}{biblatex}

% Disable the empty pages automatically put by memoir class, except the ones by \cleardoublepage
% \PassOptionsToClass{openany}{memoir}

% Fixes several `abntex2` class problems



% Applying options to already loaded package
% https://tex.stackexchange.com/questions/124049/applying-options-to-already-loaded-package
%
% For web links and paths with \path{..} and \url{https://www.python.org/downloads/}
% https://tex.stackexchange.com/questions/3033/forcing-linebreaks-in-url
% ftp://tug.ctan.org/pub/tex-archive/macros/latex/contrib/hyperref/doc/options.pdf
\PassOptionsToPackage{hyphens}{url}

% Use its macro adjustwidth* to extend tables out of outer text border.
% https://tex.stackexchange.com/questions/366155/how-to-write-a-table-a-little-larger-than-the-paragraphs-with-centered-columns
\PassOptionsToPackage{strict}{changepage}

% The class `abntex2` loads the `enumitem` package with options.
%
% With the package option shortlabels you can use an enumerate-like syntax, where A, a, I, i and 1
% stand for \Alph*, \alph*, \Roman*, \roman* and \arabic*. This is intended mainly as a sort of
% compatibility mode with the enumerate package, and therefore the following special rule applies:
% if the very first option (at any level) is not recognized as a valid key, then it will be
% considered a label with the enumerate-like syntax.
%
% No spacing between enumerated items with \usepackage{enumerate}
% https://tex.stackexchange.com/questions/119919/no-spacing-between-enumerated-items-with-usepackageenumerate
\PassOptionsToPackage{shortlabels}{enumitem}

% Fixes `pdfTeX warning (ext4): destination with the same identifier (name{figure.1.1}) has been
% already used, duplicate ignored`.
%
% The `abntex2` package loads the `hyperref` package, however there are several packages which are
% required to be loaded after and before `hyperref`.
%
% Which packages should be loaded after hyperref instead of before?
% https://tex.stackexchange.com/questions/1863/which-packages-should-be-loaded-after-hyperref-instead-of-before
%
% Hyperref is loaded by the class, and I need to load packages that are supposed to be loaded before
% https://tex.stackexchange.com/questions/50846/hyperref-is-loaded-by-the-class-and-i-need-to-load-packages-that-are-supposed-t
%
% Using \BeforePackage to load a package before hyperref does not work
% https://tex.stackexchange.com/questions/51094/using-beforepackage-to-load-a-package-before-hyperref-does-not-work
\RequirePackage{scrlfile}
\AfterClass{memoir}
{
    \RequirePackage{float}
}

% The class `abntex2` loads the package `enumitem`, but `paralist` must be loaded before it
% https://tex.stackexchange.com/questions/162799/compilation-error-when-including-enumitem-and-paralist-packages
\AfterClass{memoir}
{
    % How to make horizontal lists?
    % https://tex.stackexchange.com/questions/146306/how-to-make-horizontal-lists
    \RequirePackage{paralist}
}

% The UFSC font size is 10.5, but memoir embbeded on `abntex2` only accepts 10 and 11pt. However
% this will be fixed later by the abntex2-ufsc package.
%
% You can see whether the font sizes are correct with the \f@size at the `\begin{document}`.
\documentclass[
10pt,
a5paper,
twoside, % Impressão nos dois lados da folha
chapter=TITLE, % Título de capítulos em caixa alta
section=TITLE  % Título de seções em caixa alta
]{abntex2}

% Custom list throw LaTeX Error: Command \mycustomfiction already defined?
% https://tex.stackexchange.com/questions/388489/custom-list-throw-latex-error-command-mycustomfiction-already-defined
\usepackage{morewrites}

% If document class is equal to something then do not use package
% https://tex.stackexchange.com/questions/46223/if-document-class-is-equal-to-something-then-do-not-use-package
\makeatletter
\@ifclassloaded{abntex2}
{
    % Package hyperref Warning: Token not allowed in a PDF string?
    % https://tex.stackexchange.com/questions/384885/package-hyperref-warning-token-not-allowed-in-a-pdf-string
    \pdfstringdefDisableCommands{\let\uppercase\relax}
}{}
\makeatother

% backreferencing in classicthesis package does not work
% https://tex.stackexchange.com/questions/115828/backreferencing-in-classicthesis-package-does-not-work
\usepackage[brazilian,hyperpageref]{backref}

% Citação alfabética por autor-data [alf]
\usepackage[alf]{abntex2cite}

% Is it possible to translate the 'and' separator in references using BibTeX?
% https://tex.stackexchange.com/questions/305975/is-it-possible-to-translate-the-and-separator-in-references-using-bibtex
\usepackage{babelbib}

% Comment this, unless you are debugging pages' badness Underfull & Overflow
% https://tex.stackexchange.com/questions/115908/geometry-showframe-landscape
%
% What is the difference between \usepackage{showframe} and \usepackage[showframe]{geometry}?
% https://tex.stackexchange.com/questions/387077/what-is-the-difference-between-usepackageshowframe-and-usepackageshowframe
%
% How to do the memoir headings fix but not have my text going over the page bottom margins?
% https://tex.stackexchange.com/questions/387257/how-to-do-the-memoir-headings-fix-but-not-have-my-text-going-over-the-page-botto
%
% Print page margins of a document
% https://tex.stackexchange.com/questions/14508/print-page-margins-of-a-document
% \usepackage[showframe,pass]{geometry}




% The UFSC font size is 10.5, but memoir embedded by `abntex2` only accepts 10 and 11pt.
% However, problem will be fixed the `ufscthesisx` package.
\documentclass[
10pt,          % Padrão UFSC para versão final
a5paper,       % Padrão UFSC para versão final
% 12pt,        % Pode usar tamanho 12pt para defesa
% a4paper,     % Pode usar a4 para defesa
twoside,       % Impressão nos dois lados da folha
chapter=TITLE, % Título de capítulos em caixa alta
section=TITLE, % Título de seções em caixa alta
]{setup/ufscthesisx}

% Load extra commands for tables, lists, summaries, etc.


% Load all required basic packages
\input{setup/utilities/basic}
\input{setup/utilities/commands}

% Bad boxes settings and programming environments
\input{setup/utilities/badboxes}
\input{setup/utilities/programming}

% Input a empty list of commands when on debug mode
\input{setup/utilities/commands_list}




% % Utilize o arquivo aftertext/references.bib para incluir sua bibliografia.
\addbibresource{aftertext/references.bib}

% FIXME: Preencha com seus dados
\autor{\brazilword{Escreva aqui o Nome completo do Autor ou da Autora}}
\titulo{Título do trabalho}

% FIXME: Se houver subtítulo, descomente a linha abaixo
% \subtitulo{\lang{Subtitle}{Subtítulo}}

% FIXME: Siglas para grau de formação Dr./Dra., Me./Ma, Bel. Bela. (inglês: PhD., MSc., Bs.)
\orientador[Orientador(a)]{\brazilword{Nome completo do Orientador(a)}, Dr.}

% FIXME: Se houver coorientador, descomente a linha abaixo
% \coorientador[\lang{Co-supervisor}{Coorientador(a)}]{\brazilword{Nome do coorientador(a)}, \lang{Phd.}{Dr.}}

% FIXME: Preencher com o nome do Coordenador de TCCs/Teses do seu curso
\coordenador[Coordenador(a)]{\brazilword{Nome do Coordenador(a)}, Dr.}

% FIXME: Local da sua defesa
\local{\brazilword{Local da sua Defesa}}

% FIXME: Ano da sua defesa
\ano{2001}
\biblioteca{Biblioteca Universitária}

% FIXME: Sigla da sua instituição
\instituicaosigla{UFSC}
\instituicao{Universidade Federal de \brazilword{Santa Catarina}}

% FIXME: Preencha com Tese, Dissertação, Monografia ou Trabalho de Conclusão de Curso, Bachelor's Thesis, etc
\tipotrabalho{Monografia}

% FIXME: Se houver Área de Concentração, descomente a linha abaixo
% \area{\lang{Concentration Area}{Área de Concentração}}

% FIXME: Preencha com Doutor, Bacharel ou Mestrando
\formacao{Doutor em XXXXXX%
}
\programa{Programa de Pós-Graduação em XXXXXX%
}

% FIXME: Preencha com Departamento de XXXXXX, Centro de XXXXXX
\centro{Departamento de XXXXXX, Centro de XXXXXX%
}

% FIXME: Data da sua defesa
\data{30 de março de 2001}

% O preambulo deve conter tipo do trabalho, objetivo, nome da instituição e a área de concentração.
\preambulo{%
        \imprimirtipotrabalho~submetido ao \imprimirprograma~da
        \imprimirinstituicao~para a obtenção do Grau de \imprimirformacao.%
    %
}

\palavraschaveufsc{palavraschaveingles}   {Keyword 1}
\palavraschaveufsc{palavraschaveportugues}{Palavra Chave 1}

\palavraschaveufsc{palavraschaveingles}   {Keyword 2}
\palavraschaveufsc{palavraschaveportugues}{Palavra Chave 2}

\palavraschaveufsc{palavraschaveingles}   {Keyword 3}
\palavraschaveufsc{palavraschaveportugues}{Palavra Chave 3}

% Altere o arquivo 'settings.tex' para incluir customizações de aparência da sua tese
%----------------------------------------------------------------------------------------
%   Thesis Tweaks and Utilities
%----------------------------------------------------------------------------------------

% Uncomment this if you are debugging pages' badness Underfull & Overflow
% https://tex.stackexchange.com/questions/115908/geometry-showframe-landscape
% https://tex.stackexchange.com/questions/387077/what-is-the-difference-between-usepackageshowframe-and-usepackageshowframe
% https://tex.stackexchange.com/questions/387257/how-to-do-the-memoir-headings-fix-but-not-have-my-text-going-over-the-page-botto
% https://tex.stackexchange.com/questions/14508/print-page-margins-of-a-document
% \usepackage[showframe,pass]{geometry}

% To use the font Times New Roman, instead of the default LaTeX font
% more up-to-date than '\usepackage{mathptmx}'
% \usepackage{newtxtext}
% \usepackage{newtxmath}

% https://tex.stackexchange.com/questions/182569/how-to-manually-set-where-a-word-is-split
\hyphenation{Ge-la-im}

% Simple alias for English and Portuguese words
\newcommand{\brazilword}[1]{\foreignlanguage{brazil}{#1}}
\newcommand{\englishword}[1]{\foreignlanguage{english}{#1}}

\lang % Environments to switch between english and brazil for big text blocks
{\includecomment{englishtext}\excludecomment{braziltext}}
{\includecomment{braziltext}\excludecomment{englishtext}}

% Backward compatibility with `abntex2cite` package `\citeonline` command with the new `biblatex` package
% https://tex.stackexchange.com/questions/655/what-is-the-difference-between-def-and-newcommand
\let\citeonline\textcite

% Add missing translations for Portuguese
\addto\captionsbrazil { %
    \renewcommand{\mytextpreliminarylistname}{Breve Sumário}
    % Remove the colon appended to variables by 'abntex2' because they show up on places they shouldn't
    \renewcommand{\orientadorname}{Orientador}
    \renewcommand{\coorientadorname}{Coorientador}
}

% Add English translations and fix names from abnTeX2
% https://tex.stackexchange.com/questions/8564/what-is-the-right-way-to-redefine-macros-defined-by-babel
\addto\captionsenglish { %
    \renewcommand{\coorientadorname}{Co\hyp{}supervisor}
    % Remove the colon appended to variables by 'abntex2' because they show up on places they shouldn't
    \renewcommand{\orientadorname}{Supervisor}
    \renewcommand{\coorientadorname}{Co\hyp{}supervisor}
}

% Selects a sans serif font family
% \renewcommand{\sfdefault}{cmss}

% Selects a monospaced (“typewriter”) font family
% \renewcommand{\ttdefault}{cmtt}

% Source Code Settings in Document
\makeatletter
\@ifpackageloaded{listings}
{
\ifenglish
    % These default values are already in English

\else
    % Listing -> Codigo fonte
    \renewcommand{\lstlistingname}{Código--fonte}

    % List of Listings -> Lista de códigos-fonte
    \renewcommand{\lstlistlistingname}{Lista de códigos--fonte}

    % Recalculate the size of the header
    \calculatelisteningsheader
\fi
}{}
\makeatother

% Spacing between lines and paragraphs
% https://tex.stackexchange.com/questions/70212/ifpackageloaded-question
\makeatletter
\@ifclassloaded{memoir}
{
    % New custom chapter style VZ14, see other chapters styles in:
    % http://repositorios.cpai.unb.br/ctan/info/latex-samples/MemoirChapStyles/MemoirChapStyles.pdf
    \newcommand\thickhrulefill{\leavevmode \leaders \hrule height 1ex \hfill \kern \z@}
    \makechapterstyle{VZ14} { %
        % \thispagestyle{empty}
        \setlength\beforechapskip{50pt}
        \setlength\midchapskip{20pt}
        \setlength\afterchapskip{20pt}
        \renewcommand\chapternamenum{}
        \renewcommand\printchaptername{}
        \renewcommand\chapnamefont{\Huge\scshape}
        \renewcommand\printchapternum {%
            \chapnamefont\null\thickhrulefill\quad
            \@chapapp\space\thechapter\quad\thickhrulefill
        }
        \renewcommand\printchapternonum {%
            \par\thickhrulefill\par\vskip\midchapskip
            \hrule\vskip\midchapskip
        }
        \renewcommand\chaptitlefont{\huge\scshape\centering}
        \renewcommand\afterchapternum {%
            \par\nobreak\vskip\midchapskip\hrule\vskip\midchapskip
        }
        \renewcommand\afterchaptertitle {%
            \par\vskip\midchapskip\hrule\nobreak\vskip\afterchapskip
        }
    }

    % Apply the style `VZ14` just created
    % \chapterstyle{VZ14}

    % Controlling the spacing between one paragraph and another, try also \onelineskip
    % Default value for UFSC 0.0cm
    \setlength{\parskip}{0.2cm}

    % Paragraph size is given by
    % Default value for UFSC 1.0cm
    \setlength{\parindent}{1.3cm}

    % http://mirrors.ibiblio.org/CTAN/macros/latex/contrib/memoir/memman.pdf
    \setlength\beforechapskip{0pt}
    \setlength\midchapskip{15pt}
    \setlength\afterchapskip{15pt}

    % Memoir: Warnings “The material used in the headers is too large” w/ accented titles
    % https://tex.stackexchange.com/questions/387293/how-to-change-the-page-layout-with-memoir
    \setheadfoot{30.0pt}{\footskip}
    \checkandfixthelayout
}{}
\makeatother

% Color settings across the document
\makeatletter
\@ifpackageloaded{xcolor}
{
    % RGB colors in absolute values from 0 to 255 by using `RGB` tag
    \definecolor{darkblue}{RGB}{26,13,178}

    % Colors names definitions as RGB colors in percentage notation by using `rgb` tag
    \definecolor{mygreen}{rgb}{0,0.6,0}
    \definecolor{mygray}{rgb}{0.5,0.5,0.5}
    \definecolor{mymauve}{rgb}{0.58,0,0.82}
    \definecolor{figcolor}{rgb}{1,0.4,0}
    \definecolor{tabcolor}{rgb}{1,0.4,0}
    \definecolor{eqncolor}{rgb}{1,0.4,0}
    \definecolor{linkcolor}{rgb}{1,0.4,0}
    \definecolor{citecolor}{rgb}{1,0.4,0}
    \definecolor{seccolor}{rgb}{0,0,1}
    \definecolor{abscolor}{rgb}{0,0,1}
    \definecolor{titlecolor}{rgb}{0,0,1}
    \definecolor{biocolor}{rgb}{0,0,1}
    \definecolor{blue}{RGB}{41,5,195}

    % PDF Hyperlinks settings
    \@ifpackageloaded{hyperref}
    {
        \hypersetup
        {
            pdftitle={\@title},
            colorlinks=true,     % false: boxed links; true: colored links
            linkcolor=darkblue,  % color of internal links
            citecolor=darkgreen, % color of links to bibliography
            filecolor=black,     % color of file links
            urlcolor=linkcolor,
            bookmarksdepth=4
        }
        \ifenglish
            \hypersetup
            {
                pdfauthor={Author},
                pdfsubject={Thesis' Abstract},
                pdfcreator={LaTeX with abnTeX2 for UFSC},
                pdfkeywords={abnt}{latex}{UFSC}{abntex2}{thesis},
            }
        \else
            \hypersetup
            {
                pdfauthor={Autores},
                pdfsubject={Resumo da tese},
                pdfcreator={LaTeX com abnTeX2 para UFSC},
                pdfkeywords={abnt}{latex}{UFSC}{abntex2}{tese},
            }
        \fi
    }
}{}
\makeatother

% https://tex.stackexchange.com/questions/14314/changing-the-font-of-the-numbers-in-the-toc-in-the-memoir-class
\renewcommand{\cftpartfont}{\ABNTEXpartfont\color{ultramarine}}
\renewcommand{\cftpartpagefont}{\ABNTEXpartfont\color{black}}

\renewcommand{\cftchapterfont}{\ABNTEXchapterfont\color{ultramarine}}
\renewcommand{\cftchapterpagefont}{\ABNTEXchapterfont\color{black}}

\renewcommand{\cftsectionfont}{\ABNTEXsectionfont\color{ultramarine}}
\renewcommand{\cftsectionpagefont}{\ABNTEXsectionfont\color{black}}

\renewcommand{\cftsubsectionfont}{\ABNTEXsubsectionfont\color{ultramarine}}
\renewcommand{\cftsubsectionpagefont}{\ABNTEXsubsectionfont\color{black}}

\renewcommand{\cftsubsubsectionfont}{\ABNTEXsubsubsectionfont\color{ultramarine}}
\renewcommand{\cftsubsubsectionpagefont}{\ABNTEXsubsubsectionfont\color{black}}

\renewcommand{\cftparagraphfont}{\ABNTEXsubsubsubsectionfont\color{ultramarine}}
\renewcommand{\cftparagraphpagefont}{\ABNTEXsubsubsubsectionfont\color{black}}

% Memoir has another mechanism for the job: \cftsetindents{‹kind›}{indent}{numwidth}. Here kind is
% chapter, section, or whatever; the indent specifies the ‘margin’ before the entry starts; and the
% width is of the box into which the number is typeset (so needs to be wide enough for the largest
% number, with the necessary spacing to separate it from what comes after it in the line.
% http://www.tex.ac.uk/FAQ-tocloftwrong.html
% https://tex.stackexchange.com/questions/264668/memoir-indentation-of-unnumbered-sections-in-table-of-contents
% https://tex.stackexchange.com/questions/394227/memoir-toc-indent-the-second-line-by-numberspace
%
% `\cftlastnumwidth` and these `\cftsetindents` are defined by the abntex2 class,
% obeying the `ABNTEXsumario-abnt-6027-2012`. \newlength{\cftlastnumwidth}
\setlength{\cftlastnumwidth}{\cftsubsubsectionnumwidth}
\addtolength{\cftlastnumwidth}{-1em}

% http://www.tex.ac.uk/FAQ-tocloftwrong.html
% Use \setlength\cftsectionnumwidth{4em} to override all these values at once
\cftsetindents{part}         {0em}{\cftlastnumwidth}
\cftsetindents{chapter}      {0em}{\cftlastnumwidth}
\cftsetindents{section}      {0em}{\cftlastnumwidth}
\cftsetindents{subsection}   {0em}{\cftlastnumwidth}
\cftsetindents{subsubsection}{0em}{\cftlastnumwidth}
\cftsetindents{paragraph}    {0em}{\cftlastnumwidth}
\cftsetindents{subparagraph} {0em}{\cftlastnumwidth}

% Backref package settings, pages with citations in bibliography
\newcommand{\biblatexcitedntimes}{\autocap{c}ited \arabic{citecounter} times}
\newcommand{\biblatexcitedonetime}{\autocap{c}ited one time}
\newcommand{\biblatexcitednotimes}{\autocap{n}o citation in the text}

\addto\captionsbrazil { %
    \renewcommand{\biblatexcitedntimes}{\autocap{c}itado \arabic{citecounter} vezes}
    \renewcommand{\biblatexcitedonetime}{\autocap{c}itado uma vez}
    \renewcommand{\biblatexcitednotimes}{\autocap{n}enhuma citação no texto}
}
\makeatletter
\@ifpackageloaded{biblatex}
{%
    % https://tex.stackexchange.com/questions/483707/how-to-detect-whether-the-option-citecounter-was-enabled-on-biblatex
    \ifx\blx@citecounter\relax
        \message{Is citecounter defined? NO!^^J}
    \else
        \message{Is citecounter defined? YES!^^J}
        \ifbacktracker
            \message{Is backtracker defined? YES!^^J}
            \renewbibmacro*{pageref}
            {
                \iflistundef{pageref}
                {\printtext{\biblatexcitednotimes}}
                {%
                    \printtext
                    {%
                        \ifnumgreater{\value{citecounter}}{1}
                            {\biblatexcitedntimes}
                            {\biblatexcitedonetime}
                    }%
                    \setunit{\addspace}%
                    \ifnumgreater{\value{pageref}}{1}
                        {\bibstring{backrefpages}\ppspace}
                        {\bibstring{backrefpage}\ppspace}%
                    \printlist[pageref][-\value{listtotal}]{pageref}%
                }%
            }

            \DefineBibliographyStrings{brazil}
            {
                backrefpage  = {na página},
                backrefpages = {nas páginas},
            }

            \DefineBibliographyStrings{english}
            {
                backrefpage  = {on page},
                backrefpages = {on pages},
            }
        \else
            \message{Is backtracker defined? NO!^^J}
        \fi
    \fi
}{}
\makeatother

% https://tex.stackexchange.com/questions/391695/is-possible-to-remove-the-link-color-of-the-comma-on-the-citation-link
% \DeclareFieldFormat{citehyperref}{#1}

% https://tex.stackexchange.com/questions/19105/how-can-i-put-more-space-between-bibliography-entries-biblatex
% \setlength\bibitemsep{2.1\itemsep}

% % https://tex.stackexchange.com/questions/203764/reduce-font-size-of-bibliography-overfull-bibliography
% \newcommand{\bibliographyfontsize}{\fontsize{10.0pt}{10.5pt}\selectfont}
% \renewcommand*{\bibfont}{\bibliographyfontsize}

% % Uncomment this to insert the abstract into your bibliography entries when the abstract is available
% % https://tex.stackexchange.com/questions/398666/how-to-correctly-insert-and-justify-abstract
% \DeclareFieldFormat{abstract}%
% {%
%     \vspace*{-0.5mm}\par\justifying
%     \begin{adjustwidth}{1cm}{}
%         \textbf{\bibsentence\bibstring{abstract}:} #1
%     \end{adjustwidth}
% }
% \renewbibmacro*{finentry}%
% {%
%     \iffieldundef{abstract}
%     {\finentry}
%     {\finentrypunct
%         \printfield{abstract}%
%         \renewcommand*{\finentrypunct}{}%
%         \finentry
%     }
% }



% Allows you to use ~= instead of `\hyp{}`
% https://tex.stackexchange.com/questions/488008/how-to-create-an-alternative-to-shortcut-or-hyp
% \useshorthands{~}\defineshorthand{~=}{\hyp{}}

% When writing a large document, it is sometimes useful to work on selected sections of the document
% to speed up compilation time: https://en.wikibooks.org/wiki/TeX/includeonly
\newif\ifforcedinclude\forcedincludefalse

% \addtoincludeonly{beforetext/agradecimentos}
% \addtoincludeonly{beforetext/epigrafe}
% \addtoincludeonly{beforetext/fichacatalografica}
% \addtoincludeonly{beforetext/folhadeaprovacao}
% \addtoincludeonly{beforetext/resumos}
% \addtoincludeonly{beforetext/siglas}
% \addtoincludeonly{beforetext/simbolos}

% \addtoincludeonly{chapters/intro}
% \addtoincludeonly{chapters/chapter_1}
% \addtoincludeonly{chapters/chapter_2}
% \addtoincludeonly{chapters/chapter_3}
% \addtoincludeonly{chapters/chapter_4}
% \addtoincludeonly{chapters/conclusion}

% \addtoincludeonly{aftertext/anexo_a}
% \addtoincludeonly{aftertext/anexo_b}
% \addtoincludeonly{aftertext/apendice_a}

% Control whether the full document will be generated
% Note: It will also generate severals errors like the following, which can be ignored
%       Latexmk: Missing input file: 'chapters/test.aux'
%
% You can make latex stop generate these errors, if you generate a full version
% of the document, before uncommenting these lines.
%
% Uncomment these two lines, to only partially generate the document
% \doincludeonly
% \forcedincludetrue

\ifenglish
    \hypersetup
    {
        pdfauthor={Author},
        pdfsubject={Thesis' Abstract},
        pdfcreator={LaTeX with abnTeX2 for UFSC},
        pdfkeywords={abnt}{latex}{UFSC}{abntex2}{thesis},
    }
\else
    \hypersetup
    {
        pdfauthor={Autores},
        pdfsubject={Resumo da tese},
        pdfcreator={LaTeX com abnTeX2 para UFSC},
        pdfkeywords={abnt}{latex}{UFSC}{abntex2}{tese},
    }
\fi

\begin{document}

    % FIXME: Comment this after finishing the thesis, so you can start fixing the \flushbottom vs \raggedbottom
    % https://tex.stackexchange.com/questions/65355/flushbottom-vs-raggedbottom
    \raggedbottom

    % https://tex.stackexchange.com/questions/4705/double-space-between-sentences
    \frenchspacing

    % ELEMENTOS PRÉ-TEXTUAIS
    

% How to fix destination with the same identifier (name{page.A}) has been already used, duplicate ignored?
% https://tex.stackexchange.com/questions/386446/how-to-fix-destination-with-the-same-identifier-namepage-a-has-been-already
\hypersetup{pageanchor=false}


% ELEMENTOS PRÉ-TEXTUAIS
\ifforcedinclude\else
    % pretextual settings
    \pretextual\PRIVATEbookmarkthis{\covername}

    % Capa
    % \includepdf{pictures/FrenteCapaUFSC.pdf}
    \imprimircapa

    % Folha de rosto (o * indica que haverá a ficha bibliográfica)
    \imprimirfolhaderosto*
\fi

% Inserir a ficha bibliografica
%
% Isto é um exemplo de Ficha Catalográfica, ou ``Dados internacionais de
% catalogação-na-publicação''. Você pode utilizar este modelo como referência.
% Porém, provavelmente a biblioteca da sua universidade lhe fornecerá um PDF
% com a ficha catalográfica definitiva após a defesa do trabalho. Quando estiver
% com o documento, salve-o como PDF no diretório do seu projeto e substitua todo
% o conteúdo de implementação deste arquivo pelo comando abaixo:


% http://portalbu.ufsc.br/ficha
% http://portal.bu.ufsc.br/servicos/ficha-de-identificacao-da-obra/
\begin{fichacatalografica}
    \vspace*{\fill}

    \begin{center}

        \lang
        {Cataloging at source by the University Library of the Federal University of Santa Catarina.}
        {Catalogação na fonte pela Biblioteca Universitária da Universidade Federal de Santa Catarina.}

        \lang
        {File compiled at \currenttime h of the day \today.}
        {Arquivo compilado às \currenttime h do dia \today.}

        \framebox[\textwidth]
        {
            \begin{minipage}{0.98\textwidth}

                \ttfamily
                \imprimirautor

                \hspace{0.5cm} \imprimirtitulo~:~\imprimirsubtitulo~/~\imprimirautor;
                \imprimirorientadorRotulo,~\imprimirorientador;~\imprimircoorientadorRotulo,~\imprimircoorientador
                ~--~\imprimirlocal,~\currenttime,~\imprimirdata.

                % Prints how much pages there are on the document and links to the last page
                \hspace{0.5cm} \pageref{LastPage} p.
                \bigskip

                \hspace{0.5cm} \imprimirtipotrabalho~--~\imprimirinstituicao,
                \imprimircentro,~\imprimirprograma.
                \bigskip

                \hspace{0.5cm} \lang{Includes references}{Inclui referências}
                \bigskip

                \hspace{0.45cm}
                \begin{inparaenum}
                    \lang{\palavraschaveingles}{\palavraschaveportugues}
                \end{inparaenum}
                I. \imprimirorientador~
                II. \imprimircoorientador~
                III. \imprimirprograma~
                IV. \imprimirtitulo~
                \bigskip

                \hspace{7.75cm} CDU 02:141:005.7

            \end{minipage}
        }

    \end{center}

\end{fichacatalografica}


\ifforcedinclude\else\cleardoublepage\fi

% Custom list throw LaTeX Error: Command \mycustomfiction already defined?
% https://tex.stackexchange.com/questions/388489/custom-list-throw-latex-error-command-mycustomfiction-already-defined/
% \ifforcedinclude\else
    % Manually add the `\textpreliminarycontents` to the Table of Contents here
    % to keep the hyper link pointing to the beginning of the page, instead of
    % the beginning of `\textpreliminarycontents`
    %
    % When do I need to invoke \phantomsection?
    % https://tex.stackexchange.com/questions/44088/when-do-i-need-to-invoke-phantomsection
    % \phantomsection
    % \addcontentsline{toc}{chapter}{\mytextpreliminarylistname}

    % \begin{KeepFromToc}
        % \textpreliminarycontents
    % \end{KeepFromToc}
% \fi

% Inserir errata

% Inserir folha de aprovação. Isto é um exemplo de Folha de aprovação, elemento obrigatório da
% NBR 14724/2011 (seção 4.2.1.3). Você pode utilizar este modelo até a aprovação do trabalho.
% Após isso, substitua todo o conteúdo deste arquivo por uma imagem da página assinada pela banca com o comando  abaixo:

\ifforcedinclude\else\cleardoublepage\fi


\addtotextpreliminarycontent{Folha de Aprovação}

\begin{folhadeaprovacao}

    \begin{center}
        {\imprimirautor}

        \begin{center}
            \ABNTEXchapterfont\bfseries\MakeUppercase{\imprimirtitulo}\ifnotempty{\imprimirsubtitulo}{: \imprimirsubtitulo}
        \end{center}

        \begin{minipage}{\textwidth}
            Este(a) \imprimirtipotrabalho~ foi julgado adequado(a) para obtenção do Título de \imprimirformacao,
            \ifnotempty{\imprimirarea}{na área de concentração \imprimirarea,}
            e foi aprovado em sua forma final pelo \imprimirprograma~
            do \imprimircentro~ da \imprimirinstituicao.
         \end{minipage}%

    \end{center}
    \begin{center}
        \imprimirlocal, \imprimirdata.
    \end{center}

    \assinatura{\textbf{\imprimircoordenador} \\
     \imprimircoordenadorRotulo~do \imprimirprograma }


    % \newpage
    \begin{flushleft}
        \textbf{\lang{Examination Board:}{Banca Examinadora:}}
    \end{flushleft}


    \assinatura{\textbf{\imprimirorientador} \\ \imprimirorientadorRotulo\\
    \imprimirinstituicao~--~\imprimirinstituicaosigla}

    \ifnotempty{\imprimircoorientador}{
    \assinatura{\textbf{\imprimircoorientador} \\ \imprimircoorientadorRotulo \\
    \imprimirinstituicao~--~\imprimirinstituicaosigla}
    }

    \assinatura{\textbf{Prof. Convidado 1} \\
    Instituição 1 -- Sigla 1}

     \assinatura{\textbf{Prof. Convidado 2} \\
    Instituição 2 -- Sigla 2}

     \assinatura{\textbf{Prof. Convidado 3} \\
    Instituição 3 -- Sigla 3}

     \assinatura{\textbf{Prof. Convidado 4} \\
    Instituição 4 -- Sigla 4}

\end{folhadeaprovacao}


% \includepdf{folhadeaprovacao_final.pdf}


% Dedicatória
\ifforcedinclude\else\cleardoublepage\fi
\ifforcedinclude\else

\addtotextpreliminarycontent{Dedicatória}

\begin{dedicatoria}

    \vspace*{\fill}
    \centering
    \noindent
    \textit{
        Este trabalho é dedicado às crianças adultas que,\\
        quando pequenas, sonharam em se tornar cientistas.
    }
    \vspace*{\fill}

\end{dedicatoria}


\fi

% Agradecimentos
\ifforcedinclude\else\cleardoublepage\fi


\addtotextpreliminarycontent{Agradecimentos}

\begin{agradecimentos}


    Os agradecimentos principais são direcionados à Gerald Weber, Miguel Frasson,
    Leslie H. Watter, Bruno Parente Lima, Flávio de Vasconcellos Corrêa, Otavio Real
    Salvador, Renato Machnievscz\footnote{Os nomes dos integrantes do primeiro
    projeto abn\TeX\ foram extraídos de
    \url{http://codigolivre.org.br/projects/abntex/}} e todos aqueles que
    contribuíram para que a produção de trabalhos acadêmicos conforme
    as normas ABNT com \LaTeX{} fosse possível.

    Agradecimentos especiais são direcionados ao Centro de Pesquisa em Arquitetura
    da Informação\footnote{\url{http://www.cpai.unb.br/}} da Universidade de
    Brasília (CPAI), ao grupo de usuários
    \emph{latex-br}\footnote{\url{http://groups.google.com/group/latex-br}} e aos
    novos voluntários do grupo
    \emph{\abnTeX{}}\footnote{\url{http://groups.google.com/group/abntex2} e
    \url{http://abntex2.googlecode.com/}}~que contribuíram e que ainda
    contribuirão para a evolução do \abnTeX{}.


\end{agradecimentos}


%Mesmo padrão da seção primária, porém sem indicativo numérico. Assim como: Dedicatória, Resumo, Abstract, Sumário, Listas, Referências, Apêndices e Anexos.
%
%
%Corpo do texto, fonte 10,5, justificado, recuo especial da primeira linha de 1 cm, espaçamento simples.
%


% Epígrafe
\ifforcedinclude\else\cleardoublepage\fi


\addtotextpreliminarycontent{Epigrafe}

\begin{epigrafe}

\vspace*{\fill}
    \begin{flushright}
        \textit{``Assim como aquele pecado da juventude, este documento te perseguirá pelo resto da vida.''} \\ Enio Valmor Kassick
    \end{flushright}
    \begin{flushright}
        \textit{``Estupidez trará mais autoconfiança do que o conhecimento e a bravura juntas. \englishword{\showfont}''} \\ Adriano Ruseler
    \end{flushright}


\end{epigrafe}





% Ajusta o espaçamento dos parágrafos do resumo
\setlength{\absparsep}{18pt}

% RESUMOS
\ifforcedinclude\else\cleardoublepage\fi


\newcommand{\imprimirbrazilabstract}{%
    \cleardoublepage\phantomsection
    \addtotextpreliminarycontent{Resumo em Português}
    \begin{otherlanguage*}{brazil}
    \begin{resumo}[Resumo]

        Segundo a \textcite[3.1-3.2]{NBR6028:2003}, o resumo deve ressaltar o
        objetivo, o método, os resultados e as conclusões do documento. A ordem e a extensão
        destes itens dependem do tipo de resumo (informativo ou indicativo) e do
        tratamento que cada item recebe no documento original. O resumo deve ser
        precedido da referência do documento, com exceção do resumo inserido no
        próprio documento. (\ldots) As palavras-chave devem figurar logo abaixo do
        resumo, antecedidas da expressão Palavras-chave:, separadas entre si por
        ponto e finalizadas também por ponto.

        Além disso, na UFSC o texto do resumo deve ser digitado, em um único bloco, sem espaço de parágrafo. O resumo deve
        ser significativo, composto de uma sequência de frases concisas, afirmativas e não de uma
        enumeração de tópicos. Não deve conter citações. Deve usar o verbo na voz passiva. Abaixo do
        resumo, deve-se informar as palavras-chave (palavras ou expressões significativas retiradas do
        texto) ou, termos retirados de thesaurus da área.

        \imprimirpalavraschave{Palavras-chaves}{\begin{inparaitem}[]\palavraschaveportugues\end{inparaitem}}

    \end{resumo}
    \end{otherlanguage*}
}


\newcommand{\imprimirenglishabstract}{%
    % https://tex.stackexchange.com/questions/20987/changing-babel-package-inside-a-single-chapter
    % https://tex.stackexchange.com/questions/36526/multiple-language-document-babel-selectlanguage-vs-begin-endotherlanguage
    \cleardoublepage\phantomsection
    \addtotextpreliminarycontent{English's Abstract}
    \begin{otherlanguage*}{english}
    \begin{resumo}[Abstract]

        This is the English abstract.

        \imprimirpalavraschave{Keywords}{\begin{inparaitem}[]\palavraschaveingles\end{inparaitem}}

    \end{resumo}
    \end{otherlanguage*}
}


% \newcommand{\imprimirfrenchabstract}{%
%     \addtotextpreliminarycontent{Français Résumé}
%     \begin{resumo}[Résumé]
%       \begin{otherlanguage*}{french}
%           Il s'agit d'un résumé en français.

%           \imprimirpalavraschave{Mots-clés}{latex. abntex. publication de textes.}
%       \end{otherlanguage*}
%     \end{resumo}
% }


% \newcommand{\imprimirspanishabstract}{%
%     \addtotextpreliminarycontent{Español Resumen}
%     \begin{resumo}[Resumen]
%       \begin{otherlanguage*}{spanish}
%           Este es el resumen en español.

%           \imprimirpalavraschave{Palabras clave}{latex. abntex. publicación de textos.}
%       \end{otherlanguage*}
%     \end{resumo}
% }


\makeatletter


\ifenglish
    \@ifundefined{imprimirbrazilabstract}{}{\imprimirbrazilabstract}

    % https://tex.stackexchange.com/questions/11707/how-to-force-output-to-a-left-or-right-page
    % https://tex.stackexchange.com/questions/132966/do-not-display-chapter-title-in-memoir-class
    \cleardoublepage\phantomsection
    \pretextualchapter{Resumo Expandido}
    \addtotextpreliminarycontent{Resumo Expandido}

    \begin{otherlanguage*}{brazil}
        \setlength{\parskip}{0.2cm}
        \setlength{\parindent}{0.0cm}

        \section*{Introdução}
        O resumo expandido é previsto na Resolução Normativa nº 95/CUn/2017, Art. 55, § 2, de 4 de
        abril de 2017, e exigido para teses e dissertações escritas em idiomas estrangeiros (com
        exceção dos cursos pertinentes ao estudo de idiomas estrangeiros – Programa de Pós-Graduação
        em Estudos da Tradução e Programa de Pós-Graduação em Inglês: Estudos Linguísticos e
        Literários).

        O resumo expandido é considerado um elemento pré-textual e deverá ser incluído no trabalho
        após o resumo e antes do abstract. Deverá iniciar em página impar (no anverso de uma folha)
        continuando no verso da folha. O texto deverá seguir o formato A5, com margens espelhadas:
        superior 2,0 cm, inferior 1,5 cm, interna 2,5 cm e externa 1,5. Deve ser empregada a fonte
        Time New Roman.  Todo o texto deve ser digitado em tamanho 10,5. O espaçamento entre as
        linhas deverá ser simples. A expressão “resumo expandido” deve seguir a mesma tipografia das
        demais sessões primárias do trabalho.

        O texto do resumo expandido deve ser redigido em português e conter as seguintes seções (ver
        modelo): Introdução, Objetivos, Metodologia, Resultados e Discussão e Considerações Finais.
        Deve apresentar no mínimo duas (02) e, no máximo, cinco (05) páginas contendo a mesma
        formatação em A5 do resumo e do abstract, bem como palavras-chave.

        \section*{Objetivos}
        Lorem ipsum dolor sit amet, consectetur adipiscing elit. Phasellus vitae dolor lacus. Ut
        accumsan vitae felis nec porttitor. Integer interdum fringilla feugiat. Nullam pulvinar sit
        amet tellus eget maximus. Donec sit amet magna eget justo semper fermentum vel eget velit.
        In iaculis imperdiet mauris, ac ornare libero placerat non. Nulla libero lectus, ullamcorper
        ac ornare eget, pulvinar ac nulla. Curabitur vestibulum non nisl eget sagittis. Proin
        gravida lacus id eros bibendum interdum. Mauris ullamcorper elementum tortor sed consequat.
        Integer tempus, est a lobortis vehicula, nisi mi fringilla augue, non semper leo metus in
        quam. Etiam in leo maximus, pulvinar mi eget, vehicula risus. Donec sed dui semper, dictum
        eros at, suscipit felis.

        Nam sagittis vel orci at tempus. Nulla non pellentesque eros.
        Quisque cursus leo massa, eu ultricies nisl lacinia a. Nulla sit amet elementum ligula.
        Proin sodales venenatis dictum. Ut et est cursus, vulputate velit et, viverra odio. Interdum
        et malesuada fames ac ante ipsum primis in faucibus. Maecenas purus diam, tempor a semper
        et, finibus a ex. Cras sagittis felis urna, et consequat arcu lacinia ut. Praesent blandit
        venenatis ante nec porta. Duis rutrum, tellus vitae ullamcorper auctor, lectus ex laoreet
        est, ac tristique ipsum arcu vitae nibh. Nam efficitur felis ut mi consectetur, nec auctor
        odio ornare. In tempor vulputate urna, vitae cursus enim egestas eu. Proin diam augue,
        dignissim vitae ligula eget, lobortis ornare odio. Duis quis elit augue. Fusce quis rhoncus
        tortor. Donec hendrerit at massa a mattis. Sed ipsum neque, aliquam ut sem sed, ultrices
        varius ligula. Suspendisse blandit, dolor ac rhoncus lacinia, dolor purus cursus purus, et
        accumsan orci neque a leo.

        \section*{Metodologia}
        Quisque efficitur dolor in lectus dapibus elementum. Nam ultrices blandit consectetur.
        Nullam ultricies sit amet odio quis placerat. Aenean eget est elit. Maecenas et nulla dolor.
        Orci varius natoque penatibus et magnis dis parturient montes, nascetur ridiculus mus. In
        pulvinar velit sed mi sagittis ornare. Aenean rutrum suscipit egestas. Phasellus pharetra
        eget ex in volutpat. Quisque eu arcu nunc. Vivamus arcu ligula, pharetra at rhoncus sit
        amet, pulvinar sed eros. Sed porta ipsum ipsum, et fermentum magna volutpat sed. Vivamus
        pharetra facilisis orci, sit amet luctus nisl pretium id. Sed consequat, arcu et congue
        pulvinar, risus enim aliquet purus, eget venenatis libero leo sit amet metus. Maecenas vitae
        elit sapien. Fusce mollis libero et gravida placerat. Proin ut quam quis justo aliquam
        dictum. Donec volutpat convallis suscipit. Vivamus metus nisl, placerat ac enim vitae,
        tempus ultricies odio.

        Aliquam ac vehicula arcu, non bibendum nulla. Morbi libero sem,
        imperdiet vel quam et, posuere tempus nunc. Maecenas dictum magna sit amet ligula facilisis
        commodo. Aliquam tellus diam, ornare vel elementum in, dignissim id purus. Ut at tortor non
        sem molestie euismod non at turpis. Phasellus vitae bibendum tellus. Suspendisse odio enim,
        faucibus eget congue quis, semper sit amet tortor. Sed ac lectus est. Pellentesque nec
        mattis mi, et varius dolor. Aliquam quis massa ac tellus malesuada sollicitudin. Maecenas
        ultrices risus massa, nec auctor risus sagittis id. Praesent a sapien nulla. Donec
        tincidunt, metus quis hendrerit facilisis, enim augue convallis elit, sed consequat lacus
        odio vitae magna.

        \section*{Resultados e Discussão}
        Nullam sed cursus leo. Donec commodo volutpat hendrerit. Fusce et tempus lectus, feugiat
        consequat est. Class aptent taciti sociosqu ad litora torquent per conubia nostra, per
        inceptos himenaeos. Nam quis cursus mauris, non tempus orci. Phasellus lobortis et mauris at
        vulputate. Sed nec nisl elementum lorem commodo gravida non a enim. Phasellus neque erat,
        aliquet ac ligula ac, maximus vestibulum sem. Vestibulum vel tincidunt turpis. Donec lacinia
        rutrum dolor dapibus bibendum. Mauris pharetra nibh nec tincidunt iaculis. Vivamus pharetra
        bibendum nisl eget blandit. In lobortis diam non justo eleifend, id lobortis ante fringilla.
        Donec libero tortor, suscipit vestibulum vestibulum id, rutrum accumsan turpis. Phasellus
        sollicitudin luctus tincidunt. Suspendisse potenti. Nam semper metus et mi pharetra, in
        pretium ligula fermentum. Integer consectetur, orci non placerat feugiat, dui ex gravida
        augue, vel placerat ligula augue vel velit. Aliquam sollicitudin pellentesque congue. Donec
        vitae turpis in ante posuere posuere. Pellentesque eu justo leo. Donec quis elit vitae leo
        varius luctus quis eget justo.

        Vestibulum elementum ex neque, quis commodo tortor porttitor
        mattis. Mauris vel sagittis turpis. Aenean ligula turpis, eleifend at felis sed, cursus
        condimentum orci. Fusce accumsan est odio, eu venenatis massa sodales in. Curabitur a tempor
        nisl. Quisque consequat sed arcu a congue. In viverra, ex ut hendrerit condimentum, urna sem
        euismod eros, nec suscipit turpis dolor eget augue. Aenean posuere tellus et consectetur
        condimentum. Mauris et massa et nulla fringilla interdum. Duis quis posuere elit. Donec at
        ex non arcu faucibus rutrum et vel lectus. Vivamus pellentesque vestibulum rutrum. Sed
        pretium, purus sed efficitur feugiat, nisi justo eleifend nibh, id suscipit nunc massa nec
        lectus. In euismod enim eu sapien dictum sodales. Fusce sit amet vulputate orci. Nulla
        rutrum mauris at purus aliquet, ac sollicitudin leo laoreet. Etiam elementum posuere
        feugiat. Maecenas sed libero non augue fermentum ultricies eget at mi. Aenean auctor
        bibendum lacus, dignissim aliquet est tempus eget. Maecenas tempus, nulla id rhoncus
        suscipit, augue leo auctor mi, eget tincidunt magna mi quis dui. Maecenas ut elit in turpis
        tincidunt ultrices. Nulla id nulla aliquet, porttitor eros quis, egestas justo. Nunc nisi
        quam, egestas a accumsan fermentum, ultricies ac elit.

        Nulla porta auctor vestibulum. Sed
        consectetur lacus molestie iaculis ullamcorper. Proin porta posuere massa a lacinia. Nunc a
        lacinia orci, non vehicula ante. Vestibulum ipsum velit, congue et neque aliquam, imperdiet
        ornare augue. Donec et congue sapien. Pellentesque consequat consectetur neque ut varius. In
        aliquam ex quis ante venenatis dapibus. Vivamus et imperdiet urna. Vestibulum quis nibh
        magna. In a congue lectus, eu sodales nunc. Suspendisse id.

        \section*{Considerações Finais}
        Lorem ipsum dolor sit amet, consectetur adipiscing elit. Phasellus vitae dolor lacus. Ut
        accumsan vitae felis nec porttitor. Integer interdum fringilla feugiat. Nullam pulvinar sit
        amet tellus eget maximus. Donec sit amet magna eget justo semper fermentum vel eget velit.
        In iaculis imperdiet mauris, ac ornare libero placerat non. Nulla libero lectus, ullamcorper
        ac ornare eget, pulvinar ac nulla. Curabitur vestibulum non nisl eget sagittis. Proin
        gravida lacus id eros bibendum interdum. Mauris ullamcorper elementum tortor sed consequat.
        Integer tempus, est a lobortis vehicula, nisi mi fringilla augue, non semper leo metus in
        quam. Etiam in leo maximus, pulvinar mi eget, vehicula risus. Donec sed dui semper, dictum
        eros at, suscipit felis.

        Nam sagittis vel orci at tempus. Nulla non pellentesque eros.
        Quisque cursus leo massa, eu ultricies nisl lacinia a. Nulla sit amet elementum ligula.
        Proin sodales venenatis dictum. Ut et est cursus, vulputate velit et, viverra odio. Interdum
        et malesuada fames ac ante ipsum primis in faucibus. Maecenas purus diam, tempor a semper
        et, finibus a ex. Cras sagittis felis urna, et consequat arcu lacinia ut. Praesent blandit
        venenatis ante nec porta. Duis rutrum, tellus vitae ullamcorper auctor, lectus ex laoreet
        est, ac tristique ipsum arcu vitae nibh. Nam efficitur felis ut mi consectetur, nec auctor
        odio ornare. In tempor vulputate urna, vitae cursus enim egestas eu. Proin diam augue,
        dignissim vitae ligula eget, lobortis ornare odio. Duis quis elit augue. Fusce quis rhoncus
        tortor. Donec hendrerit at massa a mattis. Sed ipsum neque, aliquam ut sem sed, ultrices
        varius ligula. Suspendisse blandit, dolor ac rhoncus lacinia, dolor purus cursus purus, et
        accumsan orci neque a leo.


        \imprimirpalavraschave{Palavras-chaves}{\begin{inparaitem}[]\palavraschaveportugues\end{inparaitem}}

    \end{otherlanguage*}

    \@ifundefined{imprimirenglishabstract}{}{\imprimirenglishabstract}

\else
    \@ifundefined{imprimirbrazilabstract}{}{\imprimirbrazilabstract}
    \@ifundefined{imprimirenglishabstract}{}{\imprimirenglishabstract}
\fi

\@ifundefined{imprimirfrenchabstract}{}{\imprimirfrenchabstract}
\@ifundefined{imprimirspanishabstract}{}{\imprimirspanishabstract}


\makeatother



% Some tables of contents
\ifforcedinclude\else
    % inserir lista de figuras
    \ifforcedinclude\else\cleardoublepage\fi
    \addtotextpreliminarycontent{\listfigurename}
    \pdfbookmark[0]{\listfigurename}{lof}
    \listoffigures*

    % inserir lista de quadros
    \ifforcedinclude\else\cleardoublepage\fi
    \addtotextpreliminarycontent{\listofquadrosname}
    \pdfbookmark[0]{\listofquadrosname}{loq}
    \listofquadros*

    % inserir lista de tabelas
    \ifforcedinclude\else\cleardoublepage\fi
    \addtotextpreliminarycontent{\listtablename}
    \pdfbookmark[0]{\listtablename}{lot}
    \listoftables*

    % inserir códigos fonte (List of Listings `lol`)
    \ifforcedinclude\else\cleardoublepage\fi
    \addtotextpreliminarycontent{\lstlistingname}
    \pdfbookmark[0]{\lstlistingname}{lol}
    \lstlistoflistings*
\fi


% inserir lista de abreviaturas e siglas
\ifforcedinclude\else\cleardoublepage\fi


\addtotextpreliminarycontent{\lang{List of Acronyms}{Lista de Siglas}}

\begin{siglas}
    \item[ABNT] Associação Brasileira de Normas Técnicas \lang{}{, Brazilian Association of Technical Standards}
    \item[abnTeX] ABsurdas Normas para TeX \lang{}{, Absurd Standards for TeX}
\end{siglas}



% Inserir lista de símbolos
\ifforcedinclude\else\cleardoublepage\fi


\addtotextpreliminarycontent{Lista de Símbolos}

% Devam aparecer na mesma ordem de ocorrência no texto.
\begin{simbolos}
    \item[$ \Gamma $] Letra grega Gama
    \item[$ \Lambda $] Lambda
    \item[$ \zeta $] Letra grega minúscula zeta
    \item[$ \in $] Pertence
\end{simbolos}


% Add the table of contents to the brief table of contents
\addtotextpreliminarycontent{\contentsname}

% How to remove the self-reference of the ToC from the ToC?
% https://tex.stackexchange.com/questions/10943/how-to-remove-the-self-reference-of-the-toc-from-the-toc
\begin{KeepFromToc}
    \pdfbookmark[0]{\contentsname}{toc}

    % What does “overfull hbox” mean?
    % https://tex.stackexchange.com/questions/35/what-does-overfull-hbox-mean
    %
    % How to avoid using \sloppy document-wide to fix overfull \hbox problems?
    % https://tex.stackexchange.com/questions/59122/how-to-avoid-using-sloppy-document-wide-to-fix-overfull-hbox-problems
    %
    % Adding color to table of contents and section headings
    % https://tex.stackexchange.com/questions/257007/adding-color-to-table-of-contents-and-section-headings
    {
        % underfull vbox (badness 10000) has occurred while \output is active with memoir
        % https://tex.stackexchange.com/questions/65711/underfull-vbox-badness-10000-with-memoir
        \raggedbottom

        % Overfull \hbox warning for TOC entries when using memoir documentclass
        % https://tex.stackexchange.com/questions/49887/overfull-hbox-warning-for-toc-entries-when-using-memoir-documentclass
        % \makeatletter
            % \renewcommand{\@pnumwidth}{2em}
            % \renewcommand{\@tocrmarg}{3em}
        % \makeatother

        % Memoir mysterious overfull hbox in TOC when mathptmx is used
        % https://tex.stackexchange.com/questions/57544/memoir-mysterious-overfull-hbox-in-toc-when-mathptmx-is-used
        % \setlength{\cftchapternumwidth}{2.25em}

        % Disable `colorlinks` locally (or just for the ToC)
        % https://tex.stackexchange.com/questions/179506/disable-colorlinks-locally-or-just-for-the-toc
        \hypersetup{hidelinks}

        \tableofcontents
    }

\end{KeepFromToc}



% How to fix destination with the same identifier (name{page.A}) has been already used, duplicate ignored?
% https://tex.stackexchange.com/questions/386446/how-to-fix-destination-with-the-same-identifier-namepage-a-has-been-already
\hypersetup{pageanchor=true}




    % ELEMENTOS TEXTUAIS
    \textual

    % Uncomment this to put a ←← | ← (Go To Top/Go Back) on each section header
    % \addGoToSummary
    \setlength\beforechapskip{50pt}
    \setlength\midchapskip{20pt}
    \setlength\afterchapskip{20pt}

    % PARTE
    % \ifforcedinclude\else\part{\lang{Research}{Pesquisa}}\fi
    % \label{sec:primeira_parte}

    % Introdução (exemplo de capítulo sem numeração, mas presente no Sumário)
    %% intro.tex
%%
%% Copyright 2017 Evandro Coan
%% Copyright 2012-2016 by abnTeX2 group at http://www.abntex.net.br/
%%
%% This work may be distributed and/or modified under the
%% conditions of the LaTeX Project Public License, either version 1.3
%% of this license or (at your option) any later version.
%% The latest version of this license is in
%%   http://www.latex-project.org/lppl.txt
%% and version 1.3 or later is part of all distributions of LaTeX
%% version 2005/12/01 or later.
%%
%% This work has the LPPL maintenance status `maintained'.
%% The Current Maintainer of this work is the Evandro Coan.
%%
%% The last Maintainer of this work was the abnTeX2 team, led
%% by Lauro César Araujo. Further information are available on
%% https://www.abntex.net.br/
%%
%% This work consists of a bunch of files. But originally there ware 3 files
%% which are renamed as follows:
%% Renamed the `abntex2-modelo-include-comandos` to `chapters/chapter_1.tex`
%% Renamed the `abntex2-modelo-trabalho-academico.tex` to `chapters/intro.tex`
%% Renamed the `abntex2-modelo-references.bib` to `aftertext/modelo-ufsc-references.bib`
%%
%% This file was originally the main template file, however this main file was
%% split into several new files, which are respectively drastically changed,
%% except this files which contains most of the main documentation message.
%%

% ------------------------------------------------------------------------
% ------------------------------------------------------------------------
% abnTeX2: Modelo de Trabalho Academico (tese de doutorado, dissertacao de
% mestrado e trabalhos monograficos em geral) em conformidade com
% ABNT NBR 14724:2011: Informacao e documentacao - Trabalhos academicos -
% Apresentacao
% ------------------------------------------------------------------------
% ------------------------------------------------------------------------


% The \phantomsection command is needed to create a link to a place in the document that is not a
% figure, equation, table, section, subsection, chapter, etc.
%
% When do I need to invoke \phantomsection?
% https://tex.stackexchange.com/questions/44088/when-do-i-need-to-invoke-phantomsection
\phantomsection


% Is it possible to keep my translation together with original text?
% https://tex.stackexchange.com/questions/5076/is-it-possible-to-keep-my-translation-together-with-original-text
\chapter*{\lang{Introduction}{Introdução}}
\phantomsection
\addcontentsline{toc}{chapter}{INTRODUÇÃO}%
\markboth{}{Introdução}\markboth{}{Introdução}




A Tabela~\ref{tab:a_table_formatacao_de_texto} mostra  informações do modelo de teses da Biblioteca Universitária da UFSC (BU-UFSC).

% What does [t] and [ht] mean?
% https://tex.stackexchange.com/questions/8652/what-does-t-and-ht-mean
%
% How can I get rid of the LaTeX warning: Float too large for page?
% https://tex.stackexchange.com/questions/36252/how-can-i-get-rid-of-the-latex-warning-float-too-large-for-page
%
% "warning: Text page X contains only floats" How to suppress this warning?
% https://tex.stackexchange.com/questions/223149/warning-text-page-x-contains-only-floats-how-to-suppress-this-warning
%
% Make a table span multiple pages
% https://tex.stackexchange.com/questions/26462/make-a-table-span-multiple-pages
%
% How to make the longtable to work with centering & caption on memoir class?
% https://tex.stackexchange.com/questions/386541/how-to-make-the-longtable-to-work-with-centering-caption-on-memoir-class
%
% How to fix this Package array Error: Only one column-spec allowed?
% https://tex.stackexchange.com/questions/367069/how-to-fix-this-package-array-error-only-one-column-spec-allowed
%
% How to auto adjust my last table column width, and why is there Underfull \vbox badness on this table?
% https://tex.stackexchange.com/questions/387238/how-to-auto-adjust-my-last-table-column-width-and-why-is-there-underfull-vbox/387251
\setlength\extrarowheight{2pt}
\begin{tabularx}{\linewidth}{>{\RaggedRight}p{3cm}|>{\arraybackslash}X}

\caption{Formatação do texto \protect }
\label{tab:a_table_formatacao_de_texto} \\
\hline
\endfirsthead

% How to set font size of footnotes correctly in memoir?
% https://tex.stackexchange.com/questions/213927/how-to-set-font-size-of-footnotes-correctly-in-memoir
\multicolumn{2}{p{\dimexpr\textwidth-2\tabcolsep\relax}}{\ufsccaptionsize\tablename~\thetable:
Formatação do texto (continuação) \protect } \\
\hline
\endhead

% Set multicolumn width to default table width
% https://tex.stackexchange.com/questions/99326/set-multicolumn-width-to-default-table-width
\hline
\multicolumn{2}{p{\dimexpr\textwidth-2\tabcolsep\relax}}{\footnotesize continua na próxima página\protect }
\endfoot

\hline
\multicolumn{2}{p{\dimexpr\textwidth-2\tabcolsep\relax}}{\fonte{O autor} }
\endlastfoot

    Cor                          & Branco -                                                 \\ \hline
    Formato do papel             & A5                                                               \\ \hline
    Gramatura                    & 75                                                               \\ \hline
    Impressão                    & Frente e verso                                                   \\ \hline
    Margens                      & Espelhadas: superior 2, Inferior: 1,5, Externa 1,5 e Externa: 2. \\ \hline
    Cabeçalho                    & 0,7                                                              \\ \hline
    Rodapé                       & 0,7                                                              \\ \hline
    Paginação                    & Externa                                                          \\ \hline
    Alinhamento vertical         & Superior                                                         \\ \hline
    Alinhamento do texto         & Justificado                                                      \\ \hline
    Fonte sugerida               & Times New Roman                                                  \\ \hline
    Tamanho da fonte             & 10,5 para o texto incluindo os títulos das seções e subseções.
                                   As citações com mais de três linhas as legendas das ilustrações
                                   e tabelas, fonte 9,5.                                            \\ \hline
    Espaçamento entre linhas     & Um (1) simples                                                   \\ \hline
    Espaçamento entre parágrafos & Anterior 0,0; Posterior 0,0                                      \\ \hline
    Numeração da seção           & As seções  primárias devem  começar  sempre em páginas ímpares.
                                   Deixar um espaço (simples) entre o título da seção e o texto e
                                   entre o texto e o título da subseção.                            \\ \hline

\end{tabularx}




\begin{figure}
    \centering
    \caption{Exemplo de figura}
    \includegraphics[width=\linewidth]{pictures/ex01}
    \label{fig:ex01}
\fonte{o autor}
\end{figure}

Por exemplo, na \figref{fig:ex01}, tem-se...


\begin{figure}
    \centering
    \caption{Exemplo de aquisição}
    \includegraphics[width=0.9\linewidth]{pictures/tek0009}
    \label{fig:tek0009}
    \fonte{o autor}
\end{figure}

Este documento e seu código-fonte são exemplos de referência de uso da classe
\textsf{abntex2} e do pacote \textsf{abntex2cite}. O documento
exemplifica a elaboração de trabalho acadêmico (tese, dissertação e outros do
gênero) produzido conforme a ABNT NBR 14724:2011 \emph{Informação e documentação
- Trabalhos acadêmicos - Apresentação}.

A expressão ``Modelo Canônico'' é utilizada para indicar que \abnTeX{} não é
modelo específico de nenhuma universidade ou instituição, mas que implementa tão
somente os requisitos das normas da ABNT. Uma lista completa das normas
observadas pelo \abnTeX{} é apresentada em \textcite{abntex2classe}.

Sinta-se convidado a participar do projeto \abnTeX{}! Acesse o site do projeto em
\url{http://abntex2.googlecode.com/}. Também fique livre para conhecer,
estudar, alterar e redistribuir o trabalho do \abnTeX{}, desde que os arquivos
modificados tenham seus nomes alterados e que os créditos sejam dados aos
autores originais, nos termos da ``The \LaTeX{} Project Public
License''\footnote{\url{http://www.latex-project.org/lppl.txt}}.

Encorajamos que sejam realizadas customizações específicas deste exemplo para
universidades e outras instituições --- como capas, folha de aprovação, etc.
Porém, recomendamos que ao invés de se alterar diretamente os arquivos do
\abnTeX{}, distribua-se arquivos com as respectivas customizações.
Isso permite que futuras versões do \abnTeX{}~não se tornem automaticamente
incompatíveis com as customizações promovidas. Consulte
\textcite{abntex2-wiki-como-customizar} par mais informações.

Este documento deve ser utilizado como complemento dos manuais do \abnTeX{}
\cite{abntex2classe,abntex2cite,abntex2cite-alf} e da classe \textsf{memoir}
\cite{memoir}.

Esperamos, sinceramente, que o \abnTeX{} aprimore a qualidade do trabalho que
você produzirá, de modo que o principal esforço seja concentrado no principal:
na contribuição científica.

Equipe \abnTeX{}

Lauro César Araujo




    % Capitulo com exemplos de comandos inseridos de arquivo externo
    %% chapter_1.tex
%%
%% Copyright 2017 Evandro Coan
%% Copyright 2012-2016 by abnTeX2 group at http://www.abntex.net.br/
%%
%% This work may be distributed and/or modified under the
%% conditions of the LaTeX Project Public License, either version 1.3
%% of this license or (at your option) any later version.
%% The latest version of this license is in
%%   http://www.latex-project.org/lppl.txt
%% and version 1.3 or later is part of all distributions of LaTeX
%% version 2005/12/01 or later.
%%
%% This work has the LPPL maintenance status `maintained'.
%% The Current Maintainer of this work is the Evandro Coan.
%%
%% The last Maintainer of this work was the abnTeX2 team, led
%% by Lauro César Araujo. Further information are available on
%% https://www.abntex.net.br/
%%
%% This work consists of a bunch of files. But originally there ware 3 files
%% which are renamed as follows:
%% Renamed the `abntex2-modelo-include-comandos` to `chapters/chapter_2.tex`
%% Renamed the `abntex2-modelo-trabalho-academico.tex` to `chapters/chapter_1.tex`
%% Renamed the `abntex2-modelo-references.bib` to `postexto/modelo-ufsc-references.bib`
%%
%% This file was originally the main template file, however this main file was
%% split into several new files, which are respectively drastically changed,
%% except this files which contains most of the main documentation message.
%%

% ------------------------------------------------------------------------
% ------------------------------------------------------------------------
% abnTeX2: Modelo de Trabalho Academico (tese de doutorado, dissertacao de
% mestrado e trabalhos monograficos em geral) em conformidade com
% ABNT NBR 14724:2011: Informacao e documentacao - Trabalhos academicos -
% Apresentacao
% ------------------------------------------------------------------------
% ------------------------------------------------------------------------


% The \phantomsection command is needed to create a link to a place in the document that is not a
% figure, equation, table, section, subsection, chapter, etc.
%
% When do I need to invoke \phantomsection?
% https://tex.stackexchange.com/questions/44088/when-do-i-need-to-invoke-phantomsection
\cleardoublepage
\phantomsection


% Is it possible to keep my translation together with original text?
% https://tex.stackexchange.com/questions/5076/is-it-possible-to-keep-my-translation-together-with-original-text
\chapter{\lang{Introduction}{Introdução} \showfont}


% Multiple-language document - babel - selectlanguage vs begin/end{otherlanguage}
% https://tex.stackexchange.com/questions/36526/multiple-language-document-babel-selectlanguage-vs-begin-endotherlanguage
\begin{otherlanguage*}{english}

    How to display the font size in use in the final output
    \url{https://tex.stackexchange.com/questions/20451/how-to-display-the-font-size-in-use-in-the-final-output}

    \begin{verbatim}
    \makeatletter
    original: \f@size
    \verb+\small+ \small \f@size
    \verb+\normalsize+ \normalsize \f@size
    \verb+\large+ \large \f@size
    \verb+\Huge+ \Huge \f@size
    \makeatother
    \end{verbatim}



    % Write epigraphs
    \begin{flushright}
        \textit{``The curious paradox is that when I accept myself just as I am, then I can change.''} \\
        Carl Rogers
    \end{flushright}

    \newpage

    \showfont

    \section[Some encoding tests]{\showfont}

    % Why latex is letting my text goes out of the screen?
    % https://tex.stackexchange.com/questions/386762/why-latex-is-letting-my-text-goes-out-of-the-screen
    \sloppy
    \textbf{textbf: \showfont}
    \fussy

\end{otherlanguage*}




A Tabela~\ref{tab:a_table_formatacao_de_texto} mostra mais informações do template BU.

% What does [t] and [ht] mean?
% https://tex.stackexchange.com/questions/8652/what-does-t-and-ht-mean
%
% How can I get rid of the LaTeX warning: Float too large for page?
% https://tex.stackexchange.com/questions/36252/how-can-i-get-rid-of-the-latex-warning-float-too-large-for-page
%
% "warning: Text page X contains only floats" How to suppress this warning?
% https://tex.stackexchange.com/questions/223149/warning-text-page-x-contains-only-floats-how-to-suppress-this-warning
%
% Make a table span multiple pages
% https://tex.stackexchange.com/questions/26462/make-a-table-span-multiple-pages
%
% How to make the longtable to work with centering & caption on memoir class?
% https://tex.stackexchange.com/questions/386541/how-to-make-the-longtable-to-work-with-centering-caption-on-memoir-class
%
% How to fix this Package array Error: Only one column-spec allowed?
% https://tex.stackexchange.com/questions/367069/how-to-fix-this-package-array-error-only-one-column-spec-allowed
%
% How to auto adjust my last table column width, and why is there Underfull \vbox badness on this table?
% https://tex.stackexchange.com/questions/387238/how-to-auto-adjust-my-last-table-column-width-and-why-is-there-underfull-vbox/387251
\setlength\extrarowheight{2pt}
\begin{tabularx}{\linewidth}{|>{\RaggedRight}p{3cm}|>{\arraybackslash}X|}

\caption{Formatação do texto \protect\showfont}
\label{tab:a_table_formatacao_de_texto} \\
\hline
\endfirsthead

% How to set font size of footnotes correctly in memoir?
% https://tex.stackexchange.com/questions/213927/how-to-set-font-size-of-footnotes-correctly-in-memoir
\multicolumn{2}{p{\dimexpr\textwidth-2\tabcolsep\relax}}{\ufsccaptionsize\tablename~\thetable:
Formatação do texto (continued) \protect\showfont} \\
\hline
\endhead

% Set multicolumn width to default table width
% https://tex.stackexchange.com/questions/99326/set-multicolumn-width-to-default-table-width
\hline
\multicolumn{2}{p{\dimexpr\textwidth-2\tabcolsep\relax}}{\footnotesize to be continued \protect\showfont}
\endfoot

\hline
\endlastfoot

    Cor                          & Branco - \showfont                                               \\ \hline
    Formato do papel             & A5                                                               \\ \hline
    Gramatura                    & 75                                                               \\ \hline
    Impressão                    & Frente e verso                                                   \\ \hline
    Margens                      & Espelhadas: superior 2, Inferior: 1,5, Externa 1,5 e Externa: 2. \\ \hline
    Cabeçalho                    & 0,7                                                              \\ \hline
    Rodapé                       & 0,7                                                              \\ \hline
    Paginação                    & Externa                                                          \\ \hline
    Alinhamento vertical         & Superior                                                         \\ \hline
    Alinhamento do texto         & Justificado                                                      \\ \hline
    Fonte sugerida               & Times New Roman                                                  \\ \hline
    Tamanho da fonte             & 10,5 para o texto incluindo os títulos das seções e subseções.
                                   As citações com mais de três linhas as legendas das ilustrações
                                   e tabelas, fonte 9,5.                                            \\ \hline
    Espaçamento entre linhas     & Um (1) simples                                                   \\ \hline
    Espaçamento entre parágrafos & Anterior 0,0; Posterior 0,0                                      \\ \hline
    Numeração da seção           & As seções  primárias devem  começar  sempre em páginas ímpares.
                                   Deixar um espaço (simples) entre o título da seção e o texto e
                                   entre o texto e o título da subseção.                            \\ \hline

\end{tabularx}




\begin{figure}
    \centering
    \caption{Exemplo de figura}
    \includegraphics[width=\linewidth]{pictures/ex01}
    \label{fig:ex01}
\end{figure}

Por exemplo, na \figref{fig:ex01}, tem-se...


\begin{figure}
    \centering
    \includegraphics[width=0.9\linewidth]{pictures/tek0009}
    \caption{Exemplo de aquisição}
    \label{fig:tek0009}
\end{figure}

Este documento e seu código-fonte são exemplos de referência de uso da classe
\textsf{abntex2} e do pacote \textsf{abntex2cite}. O documento
exemplifica a elaboração de trabalho acadêmico (tese, dissertação e outros do
gênero) produzido conforme a ABNT NBR 14724:2011 \emph{Informação e documentação
- Trabalhos acadêmicos - Apresentação}.

A expressão ``Modelo Canônico'' é utilizada para indicar que \abnTeX{} não é
modelo específico de nenhuma universidade ou instituição, mas que implementa tão
somente os requisitos das normas da ABNT. Uma lista completa das normas
observadas pelo \abnTeX{} é apresentada em \citeonline{abntex2classe}.

Sinta-se convidado a participar do projeto \abnTeX{}! Acesse o site do projeto em
\url{http://abntex2.googlecode.com/}. Também fique livre para conhecer,
estudar, alterar e redistribuir o trabalho do \abnTeX{}, desde que os arquivos
modificados tenham seus nomes alterados e que os créditos sejam dados aos
autores originais, nos termos da ``The \LaTeX{} Project Public
License''\footnote{\url{http://www.latex-project.org/lppl.txt}}.

Encorajamos que sejam realizadas customizações específicas deste exemplo para
universidades e outras instituições --- como capas, folha de aprovação, etc.
Porém, recomendamos que ao invés de se alterar diretamente os arquivos do
\abnTeX{}, distribua-se arquivos com as respectivas customizações.
Isso permite que futuras versões do \abnTeX{}~não se tornem automaticamente
incompatíveis com as customizações promovidas. Consulte
\citeonline{abntex2-wiki-como-customizar} par mais informações.

Este documento deve ser utilizado como complemento dos manuais do \abnTeX{}
\cite{abntex2classe,abntex2cite,abntex2cite-alf} e da classe \textsf{memoir}
\cite{memoir}.

Esperamos, sinceramente, que o \abnTeX{} aprimore a qualidade do trabalho que
você produzirá, de modo que o principal esforço seja concentrado no principal:
na contribuição científica.

Equipe \abnTeX{}

Lauro César Araujo




    % Capitulo de revisão de literatura
    %% chapters/chapter_2.tex
%%
%% Copyright 2017 Evandro Coan
%% Copyright 2012-2014 by abnTeX2 group at http://abntex2.googlecode.com/
%%
%% This work may be distributed and/or modified under the
%% conditions of the LaTeX Project Public License, either version 1.3
%% of this license or (at your option) any later version.
%% The latest version of this license is in
%%   http://www.latex-project.org/lppl.txt
%% and version 1.3 or later is part of all distributions of LaTeX
%% version 2005/12/01 or later.
%%
%% This work has the LPPL maintenance status `maintained'.
%%
%% The Current Maintainer of this work is the Evandro Coan.
%%
%% This work consists of a bunch of files. But originally there were 2 files
%% which are renamed as follows:
%% Deleted the `abntex2-modelo-img-marca.pdf`
%% Renamed the `abntex2-modelo-include-comandos.tex, v-1.9.2 laurocesar` to `chapters/chapter_2.tex`
%%
% ---
% Este capítulo, utilizado por diferentes exemplos do abnTeX2, ilustra o uso de
% comandos do abnTeX2 e de LaTeX.
% ---

\cleardoublepage
\phantomsection %The \phantomsection command is needed to create a link to a place in the document that is not a figure, equation, table, section, subsection, chapter, etc.
\addcontentsline{toc}{chapter}{\texorpdfstring{\MakeTextUppercase{Capítulo 2}}{Capítulo 2}}

\chapter{Axiomas, Teoremas, Postulados, corolários, lemas}
\label{cap_exemplos}

\begin{flushright}
    \showfont
\end{flushright}

\newpage


\section{Axiomas ou postulados}

Na lógica tradicional, um axioma ou postulado é uma sentença ou proposição que não é provada ou demonstrada e é considerada como óbvia ou como um consenso inicial necessário para a construção ou aceitação de uma teoria. Por essa razão, é aceito como verdade e serve como ponto inicial para dedução e inferências de outras verdades (dependentes de teoria).


Na matemática, um axioma é uma hipótese inicial de qual outros enunciados são logicamente derivados. Pode ser uma sentença, uma proposição, um enunciado ou uma regra que permite a construção de um sistema formal. Diferentemente de teoremas, axiomas não podem ser derivados por princípios de dedução e nem são demonstráveis por derivações formais, simplesmente porque eles são hipóteses iniciais. Isto é, não há mais nada a partir do que eles seguem logicamente (em caso contrário eles seriam chamados teoremas). Em muitos contextos, "axioma", "postulado" e "hipótese" são usados como sinônimos.


\begin{axioma}[Axioma de Igualdade]
Supondo $\mathfrak{L}$, uma linguagem de primeira ordem. para cada variável $x$, a fórmula $x = x$ é universalmente válida.
\end{axioma}


\begin{postulado}[Postulado de Igualdade]
    Supondo $\mathfrak{L}$, uma linguagem de primeira ordem. para cada variável $x$, a fórmula $x = x$ é universalmente válida.
\end{postulado}


\section{Teorema}


Na matemática, um teorema é uma afirmação que pode ser provada como verdadeira através de outras afirmações já demonstradas, como outros teoremas, juntamente com afirmações anteriormente aceitas, como axiomas. Prova é o processo de mostrar que um teorema está correto. O termo teorema foi introduzido por Euclides, em Elementos, para significar "afirmação que pode ser provada". Em grego, originalmente significava "espetáculo" ou "festa". Atualmente, é mais comum deixar o termo "teorema" apenas para certas afirmações que podem ser provadas e de grande "importância matemática", o que torna a definição um tanto subjetiva.

\begin{teorema}[Teorema de Pitágoras]
    Em qualquer triângulo retângulo, o quadrado do comprimento da hipotenusa é igual à soma dos quadrados dos comprimentos dos catetos.
\end{teorema}


\subsection{Terminologia}



Usualmente deixa-se o termo ``teorema'' apenas para as afirmações que podem ser provadas de grande importância. Assim, são dados outros nomes para os outros tipos dessas afirmações:

\begin{description}
    \item[Proposição:] Uma Proposição é uma sentença não associada a algum outro teorema, de simples prova e de importância matemática menor.
    \item[Lema:] Um Lema é um "pré-teorema", um teorema que serve para ajudar na prova de outro teorema maior. A distinção entre teoremas e lemas é um tanto quanto arbitrária, uma vez que grandes resultados são usados para provar outros. Por exemplo, o Lema de Gauss e o Lema de Zorn são muito interessantes de per se, e muitos autores os denominam de Lemas, mesmo que não os usem para provar alguma outra coisa.
    \item[Corolário:] Um Corolário é uma consequência direta de outro teorema ou de uma definição, muitas vezes tendo suas demonstrações omitidas, por serem simples.
\end{description}


\begin{corolario}
    Em qualquer triângulo retângulo, a hipotenusa é maior que qualquer um dos catetos, mas menor que a soma deles.
\end{corolario}

Alguns outros termos também são usados, por mais que raros e com definição menos rigorosa, basicamente sendo usadas quando não se quer usar a a palavra "teorema":

Regra.
Lei, que também pode se referir a axiomas, regras de dedução e a distribuições de Probabilidade.
Princípio.
Algoritmo (como em Algoritmo da Divisão), muito raro e diferente do conceito com o mesmo nome que é um dos estudos centrais da Ciência da Computação.
Paradoxo, usado quando a afirmação vai aparentemente de encontro com alguma outra verdade ou com alguma noção intuitiva. Entretanto, tal termo também pode ser usado para afirmações falsas que aparentem ser verdadeiras em um primeiro momento.

Alguns teoremas continuam a ser chamados de Conjecturas logo após serem provados (por exemplo, a Conjectura de Poincaré). O termo conjectura é usado para afirmações que não se sabe se são verdadeiras, e que acredita-se que são verdadeiras, mas nunca ninguém conseguiu prová-las nem negá-las (às vezes conjecturas são chamadas de hipóteses (como em Hipótese de Riemann), obviamente, num sentido diferente do aqui já descrito).


\subsection{Conjectura ou hipótese}

Uma conjectura é uma ideia, fórmula ou frase, a qual não foi provada ser verdadeira, baseada em suposições ou ideias com fundamento não verificado. As conjecturas utilizadas como prova de resultados matemáticos recebem o nome de hipóteses.



\begin{conjectura}[Conjectura dos primos gêmeos]
Existem infinitos números primos gêmeos.
\end{conjectura}

Um par de primos é chamado de primos gêmeos se eles são dois números primos $p$, $q$ tais que $q = p + 2$.



\subsection{Lema}

    Na Matemática, um lema é um teorema que é usado como um passo intermediário para atingir um resultado maior, provado em outro teorema. Normalmente o lema tem pouca serventia além de servir ao propósito do teorema que o utiliza, mas isto não é uma regra, e a classificação entre lemas e teoremas é arbitrária\footnote{Wikipédia}.


\begin{lema}
    Given two line segments whose lengths are $a$ and $b$ respectively there is a
    real number $r$ such that $b=ra$.
\end{lema}



Unnumbered theorem-like environments are also possible.

\begin{observacao}
    This statement is true, I guess.
\end{observacao}

And the next is a somewhat informal definition


\begin{definicao}[Fibration]
    A fibration is a mapping between two topological spaces that has the homotopy lifting property for every space $X$.
\end{definicao}

\begin{exemplo}[Fibration]
    A fibration is a mapping between two topological spaces that has the homotopy lifting property for every space $X$.
\end{exemplo}


\begin{exercicio}
    Este é um exercício

\end{exercicio}

\begin{exercicio}
    Mais um exercício para vocês...

\end{exercicio}


\begin{condicao}[Fibration]
    A fibration is a mapping between two topological spaces that has the homotopy lifting property for every space $X$.
\end{condicao}
Theorem styles

\begin{description}
    \item[definition] boldface title, romand body. Commonly used in definitions, conditions, problems and examples.
\item[plain] boldface title, italicized body. Commonly used in theorems, lemmas, corollaries, propositions and conjectures.
\item[remark] italicized title, romman body. Commonly used in remarks, notes, annotations, claims, cases, acknowledgments and conclusions.
\end{description}


\section{Rotação de equações}

trecho de código para rotacionar e reduzir a fonte de equações.

%\begin{sideways}%
%   \parbox{1\textheight}{%
%       \begin{tiny}
%           \begin{equation}
%
%           \end{equation}
%       \end{tiny}}
%\end{sideways}


Segue um exemplo de rotação de páginas:

\begin{sideways}%
    \parbox{1\textheight}{%
        \begin{tiny}
\begin{equation}
\left[ {{L_{sr}}} \right] = \left[ {\begin{array}{*{20}{c}}
    {\cos \left( \theta  \right)}&{\cos \left( {\theta  - 8\alpha } \right)}&{\cos \left( {\theta  - 7\alpha } \right)}&{\cos \left( {\theta  - 6\alpha } \right)}&{\cos \left( {\theta  - 5\alpha } \right)}&{\cos \left( {\theta  - 4\alpha } \right)}&{\cos \left( {\theta  - 3\alpha } \right)}&{\cos \left( {\theta  - 2\alpha } \right)}&{\cos \left( {\theta  - \alpha } \right)}\\
    {\cos \left( {\theta  - \alpha } \right)}&{\cos \left( \theta  \right)}&{\cos \left( {\theta  - 8\alpha } \right)}&{\cos \left( {\theta  - 7\alpha } \right)}&{\cos \left( {\theta  - 6\alpha } \right)}&{\cos \left( {\theta  - 5\alpha } \right)}&{\cos \left( {\theta  - 4\alpha } \right)}&{\cos \left( {\theta  - 3\alpha } \right)}&{\cos \left( {\theta  - 2\alpha } \right)}\\
    {\cos \left( {\theta  - 2\alpha } \right)}&{\cos \left( {\theta  - \alpha } \right)}&{\cos \left( \theta  \right)}&{\cos \left( {\theta  - 8\alpha } \right)}&{\cos \left( {\theta  - 7\alpha } \right)}&{\cos \left( {\theta  - 6\alpha } \right)}&{\cos \left( {\theta  - 5\alpha } \right)}&{\cos \left( {\theta  - 4\alpha } \right)}&{\cos \left( {\theta  - 3\alpha } \right)}\\
    {\cos \left( {\theta  - 3\alpha } \right)}&{\cos \left( {\theta  - 2\alpha } \right)}&{\cos \left( {\theta  - \alpha } \right)}&{\cos \left( \theta  \right)}&{\cos \left( {\theta  - 8\alpha } \right)}&{\cos \left( {\theta  - 7\alpha } \right)}&{\cos \left( {\theta  - 6\alpha } \right)}&{\cos \left( {\theta  - 5\alpha } \right)}&{\cos \left( {\theta  - 4\alpha } \right)}\\
    {\cos \left( {\theta  - 4\alpha } \right)}&{\cos \left( {\theta  - 3\alpha } \right)}&{\cos \left( {\theta  - 2\alpha } \right)}&{\cos \left( {\theta  - \alpha } \right)}&{\cos \left( \theta  \right)}&{\cos \left( {\theta  - 8\alpha } \right)}&{\cos \left( {\theta  - 7\alpha } \right)}&{\cos \left( {\theta  - 6\alpha } \right)}&{\cos \left( {\theta  - 5\alpha } \right)}\\
    {\cos \left( {\theta  - 5\alpha } \right)}&{\cos \left( {\theta  - 4\alpha } \right)}&{\cos \left( {\theta  - 3\alpha } \right)}&{\cos \left( {\theta  - 2\alpha } \right)}&{\cos \left( {\theta  - \alpha } \right)}&{\cos \left( \theta  \right)}&{\cos \left( {\theta  - 8\alpha } \right)}&{\cos \left( {\theta  - 7\alpha } \right)}&{\cos \left( {\theta  - 6\alpha } \right)}\\
    {\cos \left( {\theta  - 6\alpha } \right)}&{\cos \left( {\theta  - 5\alpha } \right)}&{\cos \left( {\theta  - 4\alpha } \right)}&{\cos \left( {\theta  - 3\alpha } \right)}&{\cos \left( {\theta  - 2\alpha } \right)}&{\cos \left( {\theta  - \alpha } \right)}&{\cos \left( \theta  \right)}&{\cos \left( {\theta  - 8\alpha } \right)}&{\cos \left( {\theta  - 7\alpha } \right)}\\
    {\cos \left( {\theta  - 7\alpha } \right)}&{\cos \left( {\theta  - 6\alpha } \right)}&{\cos \left( {\theta  - 5\alpha } \right)}&{\cos \left( {\theta  - 4\alpha } \right)}&{\cos \left( {\theta  - 3\alpha } \right)}&{\cos \left( {\theta  - 2\alpha } \right)}&{\cos \left( {\theta  - \alpha } \right)}&{\cos \left( \theta  \right)}&{\cos \left( {\theta  - 8\alpha } \right)}\\
    {\cos \left( {\theta  - 8\alpha } \right)}&{\cos \left( {\theta  - 7\alpha } \right)}&{\cos \left( {\theta  - 6\alpha } \right)}&{\cos \left( {\theta  - 5\alpha } \right)}&{\cos \left( {\theta  - 4\alpha } \right)}&{\cos \left( {\theta  - 3\alpha } \right)}&{\cos \left( {\theta  - 2\alpha } \right)}&{\cos \left( {\theta  - \alpha } \right)}&{\cos \left( \theta  \right)}
    \end{array}} \right]
\end{equation}
\end{tiny}
}
\end{sideways}


\begin{landscape}

Outra forma é utilizar o pacote pdflscape

% Reducing font size in equation
% https://tex.stackexchange.com/questions/60453/reducing-font-size-in-equation
\tiny
\begin{equation}
\left[ {{L_{sr}}} \right] = \left[ {\begin{array}{*{20}{c}}
    {\cos \left( \theta  \right)}&{\cos \left( {\theta  - 8\alpha } \right)}&{\cos \left( {\theta  - 7\alpha } \right)}&{\cos \left( {\theta  - 6\alpha } \right)}&{\cos \left( {\theta  - 5\alpha } \right)}&{\cos \left( {\theta  - 4\alpha } \right)}&{\cos \left( {\theta  - 3\alpha } \right)}&{\cos \left( {\theta  - 2\alpha } \right)}&{\cos \left( {\theta  - \alpha } \right)}\\
    {\cos \left( {\theta  - \alpha } \right)}&{\cos \left( \theta  \right)}&{\cos \left( {\theta  - 8\alpha } \right)}&{\cos \left( {\theta  - 7\alpha } \right)}&{\cos \left( {\theta  - 6\alpha } \right)}&{\cos \left( {\theta  - 5\alpha } \right)}&{\cos \left( {\theta  - 4\alpha } \right)}&{\cos \left( {\theta  - 3\alpha } \right)}&{\cos \left( {\theta  - 2\alpha } \right)}\\
    {\cos \left( {\theta  - 2\alpha } \right)}&{\cos \left( {\theta  - \alpha } \right)}&{\cos \left( \theta  \right)}&{\cos \left( {\theta  - 8\alpha } \right)}&{\cos \left( {\theta  - 7\alpha } \right)}&{\cos \left( {\theta  - 6\alpha } \right)}&{\cos \left( {\theta  - 5\alpha } \right)}&{\cos \left( {\theta  - 4\alpha } \right)}&{\cos \left( {\theta  - 3\alpha } \right)}\\
    {\cos \left( {\theta  - 3\alpha } \right)}&{\cos \left( {\theta  - 2\alpha } \right)}&{\cos \left( {\theta  - \alpha } \right)}&{\cos \left( \theta  \right)}&{\cos \left( {\theta  - 8\alpha } \right)}&{\cos \left( {\theta  - 7\alpha } \right)}&{\cos \left( {\theta  - 6\alpha } \right)}&{\cos \left( {\theta  - 5\alpha } \right)}&{\cos \left( {\theta  - 4\alpha } \right)}\\
    {\cos \left( {\theta  - 4\alpha } \right)}&{\cos \left( {\theta  - 3\alpha } \right)}&{\cos \left( {\theta  - 2\alpha } \right)}&{\cos \left( {\theta  - \alpha } \right)}&{\cos \left( \theta  \right)}&{\cos \left( {\theta  - 8\alpha } \right)}&{\cos \left( {\theta  - 7\alpha } \right)}&{\cos \left( {\theta  - 6\alpha } \right)}&{\cos \left( {\theta  - 5\alpha } \right)}\\
    {\cos \left( {\theta  - 5\alpha } \right)}&{\cos \left( {\theta  - 4\alpha } \right)}&{\cos \left( {\theta  - 3\alpha } \right)}&{\cos \left( {\theta  - 2\alpha } \right)}&{\cos \left( {\theta  - \alpha } \right)}&{\cos \left( \theta  \right)}&{\cos \left( {\theta  - 8\alpha } \right)}&{\cos \left( {\theta  - 7\alpha } \right)}&{\cos \left( {\theta  - 6\alpha } \right)}\\
    {\cos \left( {\theta  - 6\alpha } \right)}&{\cos \left( {\theta  - 5\alpha } \right)}&{\cos \left( {\theta  - 4\alpha } \right)}&{\cos \left( {\theta  - 3\alpha } \right)}&{\cos \left( {\theta  - 2\alpha } \right)}&{\cos \left( {\theta  - \alpha } \right)}&{\cos \left( \theta  \right)}&{\cos \left( {\theta  - 8\alpha } \right)}&{\cos \left( {\theta  - 7\alpha } \right)}\\
    {\cos \left( {\theta  - 7\alpha } \right)}&{\cos \left( {\theta  - 6\alpha } \right)}&{\cos \left( {\theta  - 5\alpha } \right)}&{\cos \left( {\theta  - 4\alpha } \right)}&{\cos \left( {\theta  - 3\alpha } \right)}&{\cos \left( {\theta  - 2\alpha } \right)}&{\cos \left( {\theta  - \alpha } \right)}&{\cos \left( \theta  \right)}&{\cos \left( {\theta  - 8\alpha } \right)}\\
    {\cos \left( {\theta  - 8\alpha } \right)}&{\cos \left( {\theta  - 7\alpha } \right)}&{\cos \left( {\theta  - 6\alpha } \right)}&{\cos \left( {\theta  - 5\alpha } \right)}&{\cos \left( {\theta  - 4\alpha } \right)}&{\cos \left( {\theta  - 3\alpha } \right)}&{\cos \left( {\theta  - 2\alpha } \right)}&{\cos \left( {\theta  - \alpha } \right)}&{\cos \left( \theta  \right)}
    \end{array}} \right]
\end{equation}
\normalsize

\end{landscape}




% ---
\section{Codificação dos arquivos: UTF8}
% ---

\showfont


A codificação de todos os arquivos do \abnTeX{} é \texttt{UTF8}. É necessário que
você utilize a mesma codificação nos documentos que escrever, inclusive nos
arquivos de base bibliográficas |.bib|.

% ---
\section{Citações diretas}
\label{sec-citacao}
% ---

\index{citações!diretas}Utilize o ambiente \texttt{citacao} para incluir
citações diretas com mais de três linhas:

\begin{citacao}
As citações diretas, no texto, com mais de três linhas, devem ser
destacadas com recuo de 4 cm da margem esquerda, com letra menor que a do texto
utilizado e sem as aspas. No caso de documentos datilografados, deve-se
observar apenas o recuo \cite[5.3]{NBR10520:2002}.

\showfont
\end{citacao}

\showfont
Use o ambiente assim:

\begin{lstlisting}[language=tex]
\begin{citacao}
As citações diretas, no texto, com mais de três linhas [...]
deve-se observar apenas o recuo \cite[5.3]{NBR10520:2002}.
\end{citacao}
\end{lstlisting}

\showfont

O ambiente \texttt{citacao} pode receber como parâmetro opcional um nome de
idioma previamente carregado nas opções da classe (\autoref{sec-hifenizacao}). Nesse
caso, o texto da citação é automaticamente escrito em itálico e a hifenização é
ajustada para o idioma selecionado na opção do ambiente. Por exemplo:

\begin{lstlisting}[language=tex]
\begin{citacao}[english]
Text in English language in italic with correct hyphenation.
\end{citacao}
\end{lstlisting}

Tem como resultado:

\begin{citacao}[english]
Text in English language in italic with correct hyphenation.
\end{citacao}

\index{citações!simples}Citações simples, com até três linhas, devem ser
incluídas com aspas. Observe que em \LaTeX{} as aspas iniciais são diferentes das
finais: ``Amor é fogo que arde sem se ver''.

% ---
\section{Notas de rodapé}
% ---

As notas de rodapé são detalhadas pela NBR 14724:2011 na seção 5.2.1\footnote{As
notas devem ser digitadas ou datilografadas dentro das margens, ficando
separadas do texto por um espaço simples de entre as linhas e por filete de 5
cm, a partir da margem esquerda. Devem ser alinhadas, a partir da segunda linha
da mesma nota, abaixo da primeira letra da primeira palavra, de forma a destacar
o expoente, sem espaço entre elas e com fonte menor
\citeonline[5.2.1]{NBR14724:2011}.}\footnote{Caso uma série de notas sejam
criadas sequencialmente, o \abnTeX{} instrui o \LaTeX{} para que uma vírgula seja
colocada após cada número do expoente que indica a nota de rodapé no corpo do
texto.}\footnote{Verifique se os números do expoente possuem uma vírgula para
dividi-los no corpo do texto.}.


% ---
\section{Tabelas}
% ---

\index{tabelas}A \autoref{tab-nivinv} é um exemplo de tabela construída em
\LaTeX{}.

\begin{table}[htb]
\ABNTEXfontereduzida
\caption[Níveis de investigação]{Níveis de investigação.}
\label{tab-nivinv}
\resizebox{\textwidth}{!}{%
\begin{tabular}{p{2.6cm}|p{6.0cm}|p{2.25cm}|p{3.40cm}}
  %\hline
   \textbf{Nível de Investigação} & \textbf{Insumos}  & \textbf{Sistemas de Investigação}  & \textbf{Produtos}  \\
    \hline
    Meta-nível & Filosofia\index{filosofia} da Ciência  & Epistemologia &
    Paradigma  \\
    \hline
    Nível do objeto & Paradigmas do metanível e evidências do nível inferior &
    Ciência  & Teorias e modelos \\
    \hline
    Nível inferior & Modelos e métodos do nível do objeto e problemas do nível inferior & Prática & Solução de problemas  \\
   % \hline
\end{tabular}
}
\legend{Fonte: \citeonline{van86}}
\end{table}

Já a \autoref{tabela-ibge} apresenta uma tabela criada conforme o padrão do
\citeonline{ibge1993} requerido pelas normas da ABNT para documentos técnicos e
acadêmicos.

\begin{table}[htb]
\IBGEtab{%
  \caption{Um Exemplo de tabela alinhada que pode ser longa
  ou curta, conforme padrão IBGE.}%
  \label{tabela-ibge}
}{%
  \begin{tabular}{ccc}
  \toprule
   Nome & Nascimento & Documento \\
  \midrule \midrule
   Maria da Silva & 11/11/1111 & 111.111.111-11 \\
  \midrule
   João Souza & 11/11/2111 & 211.111.111-11 \\
  \midrule
   Laura Vicuña & 05/04/1891 & 3111.111.111-11 \\
  \bottomrule
\end{tabular}%
}{%
  \fonte{Produzido pelos autores.}%
  \nota{Esta é uma nota, que diz que os dados são baseados na
  regressão linear.}%
  \nota[Anotações]{Uma anotação adicional, que pode ser seguida de várias
  outras.}%
  }
\end{table}



% What does [t] and [ht] mean?
% https://tex.stackexchange.com/questions/8652/what-does-t-and-ht-mean
\begin{table}[!ht]
    \caption{Exemplo de tabela utilizando o pacote \emph{siunitx} e \emph{resizebox}.}
    \label{tab:SimulationResults}
    \centering
    \resizebox{\linewidth}{!}{%
        \begin{tabular}{cc|cc|cc} \toprule
            \multicolumn{2}{l}{\textbf{Fase A} } & \multicolumn{2}{l}{\textbf{Fase B}}
            & \multicolumn{2}{l}{\textbf{Fase C}} \\ \hline
            Parâmetro       & Valor       & Parâmetro        & Valor        & Parâmetro         & Valor            \\\hline
            $I_{La1}$&$\SI{2.9082866e+000}{\A}$&  $I_{Lb1}$&$\SI{3.3878432e+000}{\A}$& $I_{Lc1}$&$\SI{3.0354175e+000}{\A}$ \\
            $I_{La2}$&$\SI{2.9083278e+000}{\A}$&  $I_{Lb2}$&$\SI{3.3935604e+000}{\A}$& $I_{Lc2}$&$\SI{3.0238770e+000}{\A}$ \\
            $I_{La3}$&$\SI{2.9057255e+000}{\A}$&  $I_{Lb3}$&$\SI{3.3936165e+000}{\A}$& $I_{Lc3}$&$\SI{3.0252536e+000}{\A}$ \\
            $P_{a1}$& $\SI{625.50259e+00}{\W}$&  $P_{b1}$& $\SI{724.85424e+00}{\W}$& $P_{c1}$& $\SI{662.06883e+00}{\W}$ \\
            $P_{a2}$& $\SI{625.31121e+00}{\W}$&  $P_{b2}$& $\SI{725.62100e+00}{\W}$& $P_{c2}$& $\SI{660.36375e+00}{\W}$ \\
            $P_{a3}$& $\SI{625.96179e+00 }{\W}$&  $P_{b3}$& $\SI{725.28968e+00}{\W}$& $P_{c3}$& $\SI{660.14426e+00}{\W}$ \\
            $Q_{a1}$& $\SI{36.605745e+00 }{\VA}$&  $Q_{b1}$& $\SI{45.613691e+00}{\VA}$& $Q_{c1}$& $\SI{54.531747e+00}{\VA}$ \\
            $Q_{a2}$& $\SI{19.160357e+00}{\VA}$&  $Q_{b2}$& $\SI{36.608133e+00}{\VA}$& $Q_{c2}$& $\SI{19.939460e+00}{\VA}$ \\
            $Q_{a3}$& $\SI{18.867027e+00}{\VA}$&  $Q_{b3}$& $\SI{47.791169e+00}{\VA}$& $Q_{c3}$& $\SI{13.797842e+00}{\VA}$ \\
            $V_{Ca1}$&$\SI{400.04695e+00}{\V}$&  $V_{Cb1}$&$\SI{400.00862e+00}{\V}$& $V_{Cc1}$&$\SI{400.11656e+00}{\V}$ \\
            $V_{Ca2}$&$\SI{399.93041e+00}{\V}$&  $V_{Cb2}$&$\SI{400.05835e+00}{\V}$& $V_{Cc2}$&$\SI{399.97514e+00}{\V}$ \\
            $V_{Ca3}$&$\SI{400.02312e+00}{\V}$&  $V_{Cb3}$&$\SI{399.93403e+00}{\V}$& $V_{Cc3}$&$\SI{399.90881e+00}{\V}$ \\
            $I_{Ca1}$&$\SI{1.2605228e+000}{\A}$&  $I_{Cb1}$&$\SI{1.4684945e+000}{\A}$& $I_{Cc1}$&$\SI{1.3054048e+000}{\A}$ \\
            $I_{Ca2}$&$\SI{1.2661075e+000}{\A}$&  $I_{Cb2}$&$\SI{1.4720236e+000}{\A}$& $I_{Cc2}$&$\SI{1.3089556e+000}{\A}$ \\
            $I_{Ca3}$&$\SI{1.2598194e+000}{\A}$&  $I_{Cb3}$&$\SI{1.4708279e+000}{\A}$& $I_{Cc3}$&$\SI{1.3017673e+000}{\A}$ \\
            \bottomrule
        \end{tabular}}
\end{table}





\clearpage
% ---
\section{Figuras}
% ---

\index{figuras}Figuras podem ser criadas diretamente em \LaTeX{},
como o exemplo da \autoref{fig_circulo}.

\begin{figure}[htb]
    \caption{\label{fig_circulo}A delimitação do espaço}
    \begin{center}
        \setlength{\unitlength}{5cm}
        \begin{picture}(1,1)
        \put(0,0){\line(0,1){1}}
        \put(0,0){\line(1,0){1}}
        \put(0,0){\line(1,1){1}}
        \put(0,0){\line(1,2){.5}}
        \put(0,0){\line(1,3){.3333}}
        \put(0,0){\line(1,4){.25}}
        \put(0,0){\line(1,5){.2}}
        \put(0,0){\line(1,6){.1667}}
        \put(0,0){\line(2,1){1}}
        \put(0,0){\line(2,3){.6667}}
        \put(0,0){\line(2,5){.4}}
        \put(0,0){\line(3,1){1}}
        \put(0,0){\line(3,2){1}}
        \put(0,0){\line(3,4){.75}}
        \put(0,0){\line(3,5){.6}}
        \put(0,0){\line(4,1){1}}
        \put(0,0){\line(4,3){1}}
        \put(0,0){\line(4,5){.8}}
        \put(0,0){\line(5,1){1}}
        \put(0,0){\line(5,2){1}}
        \put(0,0){\line(5,3){1}}
        \put(0,0){\line(5,4){1}}
        \put(0,0){\line(5,6){.8333}}
        \put(0,0){\line(6,1){1}}
        \put(0,0){\line(6,5){1}}
        \end{picture}
    \end{center}
    \legend{Fonte: os autores}
\end{figure}






Ou então figuras podem ser incorporadas de arquivos externos, como é o caso da
\autoref{fig_grafico}. Se a figura que ser incluída se tratar de um diagrama, um
gráfico ou uma ilustração que você mesmo produza, priorize o uso de imagens
vetoriais no formato PDF. Com isso, o tamanho do arquivo final do trabalho será
menor, e as imagens terão uma apresentação melhor, principalmente quando
impressas, uma vez que imagens vetorias são perfeitamente escaláveis para
qualquer dimensão. Nesse caso, se for utilizar o Microsoft Excel para produzir
gráficos, ou o Microsoft Word para produzir ilustrações, exporte-os como PDF e
os incorpore ao documento conforme o exemplo abaixo. No entanto, para manter a
coerência no uso de software livre (já que você está usando \LaTeX{} e \abnTeX{}),
teste a ferramenta \textsf{InkScape}\index{InkScape}
(\url{http://inkscape.org/}). Ela é uma excelente opção de código-livre para
produzir ilustrações vetoriais, similar ao CorelDraw\index{CorelDraw} ou ao Adobe
Illustrator\index{Adobe Illustrator}. De todo modo, caso não seja possível
utilizar arquivos de imagens como PDF, utilize qualquer outro formato, como
JPEG, GIF, BMP, etc. Nesse caso, você pode tentar aprimorar as imagens
incorporadas com o software livre \textsf{Gimp}\index{Gimp}
(\url{http://www.gimp.org/}). Ele é uma alternativa livre ao Adobe
Photoshop\index{Adobe Photoshop}.

\begin{figure}[htb]
    \caption{\label{fig_grafico}Gráfico produzido em Excel e salvo como PDF}
    \begin{center}
        \includegraphics[scale=0.35]{pictures/abntex2-modelo-img-grafico.pdf}
    \end{center}
    \legend{Fonte: \citeonline[p. 24]{araujo2012}}
\end{figure}

% ---
\subsection{Figuras em minipages}
% ---

Minipages são usadas para inserir textos ou outros elementos em quadros
com tamanhos e posições controladas. Veja o exemplo da
\autoref{fig_minipage_imagem1} e da \autoref{fig_minipage_grafico2}.

\begin{figure}[htb]
\label{teste}
\centering
 \begin{minipage}{0.49\textwidth}
   \centering
   \caption{Imagem 1 da minipage} \label{fig_minipage_imagem1}
   \includegraphics[width=\textwidth]{pictures/abntex2-modelo-img-marca.pdf}
   \legend{Fonte: Produzido pelos autores}
 \end{minipage}
 \hfill
 \begin{minipage}{0.49\textwidth}
   \centering
   \caption{Grafico 2 da minipage} \label{fig_minipage_grafico2}
   \includegraphics[width=\textwidth]{pictures/abntex2-modelo-img-grafico.pdf}
   \legend{Fonte: \citeonline[p. 24]{araujo2012}}
 \end{minipage}
\end{figure}



Observe que, segundo a \citeonline[seções 4.2.1.10 e 5.8]{NBR14724:2011}, as
ilustrações devem sempre ter numeração contínua e única em todo o documento:

\begin{citacao}
Qualquer que seja o tipo de ilustração, sua identificação aparece na parte
superior, precedida da palavra designativa (desenho, esquema, fluxograma,
fotografia, gráfico, mapa, organograma, planta, quadro, retrato, figura,
imagem, entre outros), seguida de seu número de ordem de ocorrência no texto,
em algarismos arábicos, travessão e do respectivo título. Após a ilustração, na
parte inferior, indicar a fonte consultada (elemento obrigatório, mesmo que
seja produção do próprio autor), legenda, notas e outras informações
necessárias à sua compreensão (se houver). A ilustração deve ser citada no
texto e inserida o mais próximo possível do trecho a que se
refere. \cite[seções 5.8]{NBR14724:2011}
\end{citacao}

% ---
\section{Expressões matemáticas}
% ---

\index{expressões matemáticas} Use o ambiente \texttt{equation} para escrever
expressões matemáticas numeradas:

\begin{equation}
  \forall x \in X, \quad \exists \: y \leq \epsilon
\end{equation}

Escreva expressões matemáticas entre \$ e \$, como em

 $\lim_{x \to \infty}
\exp(-x) = 0 $, para que fiquem na mesma linha.

Também é possível usar colchetes para indicar o início de uma expressão
matemática que não é numerada.

\[
\left|\sum_{i=1}^n a_ib_i\right|
\le
\left(\sum_{i=1}^n a_i^2\right)^{1/2}
\left(\sum_{i=1}^n b_i^2\right)^{1/2}
\]

Consulte mais informações sobre expressões matemáticas em
\url{https://code.google.com/p/abntex2/wiki/Referencias}.


\subsection{Expressões matemáticas em títulos e subtítulos: \texorpdfstring{$\boldsymbol{\gamma_x}$}{Gammax}}






% ---
\section{Enumerações: alíneas e subalíneas}
% ---

\index{alíneas}\index{subalíneas}\index{incisos}Quando for necessário enumerar
os diversos assuntos de uma seção que não possua título, esta deve ser
subdividida em alíneas \cite[4.2]{NBR6024:2012}:

\begin{alineas}

  \item os diversos assuntos que não possuam título próprio, dentro de uma mesma
  seção, devem ser subdivididos em alíneas;

  \item o texto que antecede as alíneas termina em dois pontos;
  \item as alíneas devem ser indicadas alfabeticamente, em letra minúscula,
  seguida de parêntese. Utilizam-se letras dobradas, quando esgotadas as
  letras do alfabeto;

  \item as letras indicativas das alíneas devem apresentar recuo em relação à
  margem esquerda;

  \item o texto da alínea deve começar por letra minúscula e terminar em
  ponto-e-vírgula, exceto a última alínea que termina em ponto final;

  \item o texto da alínea deve terminar em dois pontos, se houver subalínea;

  \item a segunda e as seguintes linhas do texto da alínea começa sob a
  primeira letra do texto da própria alínea;

  \item subalíneas \cite[4.3]{NBR6024:2012} devem ser conforme as alíneas a
  seguir:

  \begin{alineas}
     \item as subalíneas devem começar por travessão seguido de espaço;

     \item as subalíneas devem apresentar recuo em relação à alínea;

     \item o texto da subalínea deve começar por letra minúscula e terminar em
     ponto-e-vírgula. A última subalínea deve terminar em ponto final, se não
     houver alínea subsequente;

     \item a segunda e as seguintes linhas do texto da subalínea começam sob a
     primeira letra do texto da própria subalínea.
  \end{alineas}

  \item no \abnTeX{} estão disponíveis os ambientes \texttt{incisos} e
  \texttt{subalineas}, que em suma são o mesmo que se criar outro nível de
  \texttt{alineas}, como nos exemplos à seguir:

  \begin{incisos}
    \item \textit{Um novo inciso em itálico};
  \end{incisos}

  \item Alínea em \textbf{negrito}:

  \begin{subalineas}
    \item \textit{Uma subalínea em itálico};
    \item \underline{\textit{Uma subalínea em itálico e sublinhado}};
  \end{subalineas}

  \item Última alínea com \emph{ênfase}.

\end{alineas}

% ---
\section{Espaçamento entre parágrafos e linhas}
% ---

\index{espaçamento!dos parágrafos}O tamanho do parágrafo, espaço entre a margem
e o início da frase do parágrafo, é definido por:

\begin{lstlisting}[language=tex]
   \setlength{\parindent}{1.3cm}
\end{lstlisting}

\index{espaçamento!do primeiro parágrafo}Por padrão, não há espaçamento no
primeiro parágrafo de cada início de divisão do documento
(\autoref{sec-divisoes}). Porém, você pode definir que o primeiro parágrafo
também seja indentado, como é o caso deste documento. Para isso, apenas inclua o
pacote \textsf{indentfirst} no preâmbulo do documento:

\begin{lstlisting}[language=tex]
   \usepackage{indentfirst}      % Indenta o primeiro parágrafo de cada seção.
\end{lstlisting}

\index{espaçamento!entre os parágrafos}O espaçamento entre um parágrafo e outro
pode ser controlado por meio do comando:

\begin{verbatim}
  \setlength{\parskip}{0.2cm}  % tente também \onelineskip
\end{verbatim}

\index{espaçamento!entre as linhas}O controle do espaçamento entre linhas é
definido por:

\begin{lstlisting}[language=tex]
  \OnehalfSpacing       % espaçamento um e meio (padrão);
  \DoubleSpacing        % espaçamento duplo
  \SingleSpacing        % espaçamento simples
\end{lstlisting}

Para isso, também estão disponíveis os ambientes:

\begin{lstlisting}[language=tex]
  \begin{SingleSpace} ...\end{SingleSpace}
  \begin{Spacing}{hfactori} ... \end{Spacing}
  \begin{OnehalfSpace} ... \end{OnehalfSpace}
  \begin{OnehalfSpace*} ... \end{OnehalfSpace*}
  \begin{DoubleSpace} ... \end{DoubleSpace}
  \begin{DoubleSpace*} ... \end{DoubleSpace*}
\end{lstlisting}

Para mais informações, consulte \citeonline[p. 47-52 e 135]{memoir}.

% ---
\section{Inclusão de código fonte}\label{sec-codeinsert}
% ---

\begin{lstlisting}[caption={Leitura dos dados simulados e conversão para estados topológicos.},label={lst:leituradadossim}]
% Pré definições iniciais
nsub=3;  % Numero de Submódulos
nbits=2*nsub; % Numero de bits necessários para representar os estados
nlevels=2*nsub+1; % Numero total de níveis

% Leitura dos pontos gerados por simulação
time=data(1,:)'; % extrai vetor de tempo
PWM=logical(data(2:end,:))'; % Conversão dos pulsos PWM para estados lógicos

% Cria vetor de string binário com os estados correspondentes
binstates=num2str([PWM(:,1) PWM(:,3) PWM(:,5) PWM(:,7) PWM(:,9) PWM(:,11)]);
state=fi(bin2dec(binstates),0,nbits,0); % Objeto numérico de ponto-fixo

\end{lstlisting}



\begin{lstlisting}[language=Python, caption=Python example]
import numpy as np

def incmatrix(genl1,genl2):
    m = len(genl1)
    n = len(genl2)
    M = None #to become the incidence matrix
    VT = np.zeros((n*m,1), int)  #dummy variable

    #compute the bitwise xor matrix
    M1 = bitxormatrix(genl1)
    M2 = np.triu(bitxormatrix(genl2),1)

    for i in range(m-1):
        for j in range(i+1, m):
            [r,c] = np.where(M2 == M1[i,j])
            for k in range(len(r)):
                VT[(i)*n + r[k]] = 1;
                VT[(i)*n + c[k]] = 1;
                VT[(j)*n + r[k]] = 1;
                VT[(j)*n + c[k]] = 1;

                if M is None:
                    M = np.copy(VT)
                else:
                M = np.concatenate((M, VT), 1)

                VT = np.zeros((n*m,1), int)

    return M
\end{lstlisting}




% ---
\section{Inclusão de outros arquivos}\label{sec-include}
% ---

É uma boa prática dividir o seu documento em diversos arquivos, e não
apenas escrever tudo em um único. Esse recurso foi utilizado neste
documento. Para incluir diferentes arquivos em um arquivo principal,
de modo que cada arquivo incluído fique em uma página diferente, utilize o
comando:

\begin{lstlisting}[language=tex]
   \include{documento-a-ser-incluido}      % sem a extensão .tex
\end{lstlisting}

Para incluir documentos sem quebra de páginas, utilize:

\begin{lstlisting}[language=tex]
   \input{documento-a-ser-incluido}      % sem a extensão .tex
\end{lstlisting}

% ---
\section{Compilar o documento \LaTeX{}}
% ---

\showfont

Geralmente os editores \LaTeX{}, como o
TeXlipse\footnote{\url{http://texlipse.sourceforge.net/}}, o
Texmaker\footnote{\url{http://www.xm1math.net/texmaker/}}, entre outros,
compilam os documentos automaticamente, de modo que você não precisa se
preocupar com isso.

No entanto, você pode compilar os documentos \LaTeX{} usando os seguintes
comandos, que devem ser digitados no \emph{Prompt de Comandos} do Windows ou no
\emph{Terminal} do Mac ou do Linux:

\begin{lstlisting}[language=bash,caption={Você pode compilar os documentos \LaTeX{} usando os seguintes
comandos.},label={lst:compilarLatex}]
   pdflatex ARQUIVO_PRINCIPAL.tex
   bibtex ARQUIVO_PRINCIPAL.aux
   makeindex ARQUIVO_PRINCIPAL.idx
   makeindex ARQUIVO_PRINCIPAL.nlo -s nomencl.ist -o
     ARQUIVO_PRINCIPAL.nls
   pdflatex ARQUIVO_PRINCIPAL.tex
   pdflatex ARQUIVO_PRINCIPAL.tex
\end{lstlisting}

\begin{lstlisting}[language=bash]
a very long and totruous path which you can check to see if it breaks and where at the end of the line
\end{lstlisting}


% ---
\section{Remissões internas}
% ---

Ao nomear a \autoref{tab-nivinv} e a \autoref{fig_circulo}, apresentamos um
exemplo de remissão interna, que também pode ser feita quando indicamos o
\autoref{cap_exemplos}, que tem o nome \emph{\nameref{cap_exemplos}}. O número
do capítulo indicado é \ref{cap_exemplos}, que se inicia à
\autopageref{cap_exemplos}\footnote{O número da página de uma remissão pode ser
obtida também assim:
\pageref{cap_exemplos}.}.
Veja a \autoref{sec-divisoes} para outros exemplos de remissões internas entre
seções, subseções e subsubseções.

O código usado para produzir o texto desta seção é:

\begin{lstlisting}[language=TeX, caption=TeX example]
Ao nomear a \autoref{tab-nivinv} e a \autoref{fig_circulo}, apresentamos um
exemplo de remissão interna, que também pode ser feita quando indicamos o
\autoref{cap_exemplos}, que tem o nome \emph{\nameref{cap_exemplos}}. O número
do capítulo indicado é \ref{cap_exemplos}, que se inicia à
\autopageref{cap_exemplos}\footnote{O número da página de uma remissão pode ser
obtida também assim:
\pageref{cap_exemplos}.}.
Veja a \autoref{sec-divisoes} para outros exemplos de remissões internas entre
seções, subseções e subsubseções.
\end{lstlisting}

% ---
\section{Divisões do documento: seção}\label{sec-divisoes}
% ---

Esta seção testa o uso de divisões de documentos. Esta é a
\autoref{sec-divisoes}. Veja a \autoref{sec-divisoes-subsection}.

\subsection{Divisões do documento: subseção}\label{sec-divisoes-subsection}

Isto é uma subseção. Veja a \autoref{sec-divisoes-subsubsection}, que é uma
\texttt{subsubsection \showfont} do \LaTeX{}, mas é impressa chamada de ``subseção'' porque
no Português não temos a palavra ``subsubseção''.

\showfont


\subsubsection{Divisões do documento: subsubseção}
\label{sec-divisoes-subsubsection}

Isto é uma subsubseção.

\subsubsection{Divisões do documento: subsubseção}

Isto é outra subsubseção.

\subsection{Divisões do documento: subseção}\label{sec-exemplo-subsec}

Isto é uma subseção.




% ---
\section[Exemplo muito longo]{Este é um exemplo de nome de seção longo. Ele deve estar
alinhado à esquerda e a segunda e demais linhas devem iniciar logo abaixo da
primeira palavra da primeira linha}
% ---

Isso atende à norma \citeonline[seções de 5.2.2 a 5.2.4]{NBR14724:2011}
 e \citeonline[seções de 3.1 a 3.8]{NBR6024:2012}.

% ---
\section{Diferentes idiomas e hifenizações}
\label{sec-hifenizacao}
% ---

Para usar hifenizações de diferentes idiomas, inclua nas opções do documento o
nome dos idiomas que o seu texto contém. Por exemplo (para melhor
visualização, as opções foram quebras em diferentes linhas):

\begin{verbatim}
\documentclass[
10.5pt, % Tamanho da fonte
a5paper, % Tamanho do papel
twoside, % Impressão nos dois lados da folha
english,
brazil,
%sumario=tradicional,
%sumario=abnt-6027-2012, % memoir v3.6k ou superior
sumario=UFSC,
chapter=TITLE, % Título de capítulos em caixa alta
section=TITLE  % Título de seções em caixa alta
]{ufsc-inep-thesis}
\end{verbatim}

O idioma português-brasileiro (\texttt{brazil}) é incluído automaticamente pela
classe \textsf{abntex2}. Porém, mesmo assim a opção \texttt{brazil} deve ser
informada como a última opção da classe para que todos os pacotes reconheçam o
idioma. Vale ressaltar que a última opção de idioma é a utilizada por padrão no
documento. Desse modo, caso deseje escrever um texto em inglês que tenha
citações em português e em francês, você deveria usar o preâmbulo como abaixo:

\begin{verbatim}
\documentclass[
    12pt,
    openright,
    twoside,
    a5paper,
    french,
    brazil,
    english
    ]{ufsc-inep-thesis}
\end{verbatim}

A lista completa de idiomas suportados, bem como outras opções de hifenização,
estão disponíveis em \citeonline[p.~5-6]{babel}.

Exemplo de hifenização em inglês\footnote{Extraído de:
\url{http://en.wikibooks.org/wiki/LaTeX/Internationalization}}:

\begin{otherlanguage*}{english}
\textit{Text in English language. This environment switches all language-related
definitions, like the language specific names for figures, tables etc. to the other
language. The starred version of this environment typesets the main text
according to the rules of the other language, but keeps the language specific
string for ancillary things like figures, in the main language of the document.
The environment hyphenrules switches only the hyphenation patterns used; it can
also be used to disallow hyphenation by using the language name
`nohyphenation'.}
\end{otherlanguage*}

Exemplo de hifenização em francês\footnote{Extraído de:
\url{http://bigbrowser.blog.lemonde.fr/2013/02/17/tu-ne-tweeteras-point-le-vatican-interdit-aux-cardinaux-de-tweeter-pendant-le-conclave/}}:

\begin{otherlanguage*}{french}
\textit{Texte en français. Pas question que Twitter ne vienne faire une
concurrence déloyale à la traditionnelle fumée blanche qui marque l'élection
d'un nouveau pape. Pour éviter toute fuite précoce, le Vatican a donc pris un
peu d'avance, et a déjà interdit aux cardinaux qui prendront part au vote
d'utiliser le réseau social, selon Catholic News Service. Une mesure valable
surtout pour les neuf cardinaux – sur les 117 du conclave – pratiquants très
actifs de Twitter, qui auront interdiction pendant toute la période de se
connecter à leur compte.}
\end{otherlanguage*}

Pequeno texto em espanhol\footnote{Extraído de:
\url{http://internacional.elpais.com/internacional/2013/02/17/actualidad/1361102009_913423.html}}:

\foreignlanguage{spanish}{\textit{Decenas de miles de personas ovacionan al pontífice en su
penúltimo ángelus dominical, el primero desde que anunciase su renuncia. El Papa se
centra en la crítica al materialismo}}.

O idioma geral do texto por ser alterado como no exemplo seguinte:

\begin{verbatim}
  \selectlanguage{english}
\end{verbatim}

Isso altera automaticamente a hifenização e todos os nomes constantes de
referências do documento para o idioma inglês. Consulte o manual da classe
\cite{abntex2classe} para obter orientações adicionais sobre internacionalização de
documentos produzidos com \abnTeX{}.

A \autoref{sec-citacao} descreve o ambiente \texttt{citacao} que pode receber
como parâmetro um idioma a ser usado na citação.

% ---
\section{Consulte o manual da classe \textsf{abntex2}}
% ---

Consulte o manual da classe \textsf{abntex2} \cite{abntex2classe} para uma
referência completa das macros e ambientes disponíveis.

Além disso, o manual possui informações adicionais sobre as normas ABNT
observadas pelo \abnTeX{} e considerações sobre eventuais requisitos específicos
não atendidos, como o caso da \citeonline[seção 5.2.2]{NBR14724:2011}, que
especifica o espaçamento entre os capítulos e o início do texto, regra
propositalmente não atendida pelo presente modelo.

% ---
\section{Referências bibliográficas}
% ---

A formatação das referências bibliográficas conforme as regras da ABNT são um
dos principais objetivos do \abnTeX{}. Consulte os manuais
\citeonline{abntex2cite} e \citeonline{abntex2cite-alf} para obter informações
sobre como utilizar as referências bibliográficas.

%-
\subsection{Acentuação de referências bibliográficas}
%-

Normalmente não há problemas em usar caracteres acentuados em arquivos
bibliográficos (\texttt{*.bib}). Porém, como as regras da ABNT fazem uso quase
abusivo da conversão para letras maiúsculas, é preciso observar o modo como se
escreve os nomes dos autores. Na ~\autoref{tabela-acentos} você encontra alguns
exemplos das conversões mais importantes. Preste atenção especial para `ç' e `í'
que devem estar envoltos em chaves. A regra geral é sempre usar a acentuação
neste modo quando houver conversão para letras maiúsculas.

\begin{table}[htbp]
\caption{Tabela de conversão de acentuação.}
\label{tabela-acentos}

\begin{center}
\begin{tabular}{ll}\hline\hline
acento & \textsf{bibtex}\\
à á ã & \verb+\`a+ \verb+\'a+ \verb+\~a+\\
í & \verb+{\'\i}+\\
ç & \verb+{\c c}+\\
\hline\hline
\end{tabular}
\end{center}
\end{table}


% ---
\section{Precisa de ajuda?}
% ---

Consulte a FAQ com perguntas frequentes e comuns no portal do \abnTeX{}:
\url{https://code.google.com/p/abntex2/wiki/FAQ}.

Inscreva-se no grupo de usuários \LaTeX{}:
\url{http://groups.google.com/group/latex-br}, tire suas dúvidas e ajude
outros usuários.

Participe também do grupo de desenvolvedores do \abnTeX{}:
\url{http://groups.google.com/group/abntex2} e faça sua contribuição à
ferramenta.

% ---
\section{Você pode ajudar?}
% ---

Sua contribuição é muito importante! Você pode ajudar na divulgação, no
desenvolvimento e de várias outras formas. Veja como contribuir com o \abnTeX{}
em \url{https://code.google.com/p/abntex2/wiki/ComoContribuir}.

% ---
\section{Quer customizar os modelos do \abnTeX{} para sua instituição ou
universidade?}
% ---

Veja como customizar o \abnTeX{} em:
\url{https://code.google.com/p/abntex2/wiki/ComoCustomizar}.



    % Primeiro capitulo de Resultados
    
% The \phantomsection command is needed to create a link to a place in the document that is not a
% figure, equation, table, section, subsection, chapter, etc.
% https://tex.stackexchange.com/questions/44088/when-do-i-need-to-invoke-phantomsection
\phantomsection

% Multiple-language document - babel - selectlanguage vs begin/end{otherlanguage}
% https://tex.stackexchange.com/questions/36526/multiple-language-document-babel-selectlanguage-vs-begin-endotherlanguage
\begin{otherlanguage*}{english}

    \chapter[Graduated]{Graduated carton sauce}

    \begin{flushright}

    \end{flushright}

    % \newpage
    \section{Before the very first basketball}

    lipsum me [21-22]

    \newpage

\end{otherlanguage*}



    % PARTE
    % \ifforcedinclude\else\part{\lang{Implementation}{Implementação}}\fi
    % \label{sec:segunda_parte}

    % Segundo capitulo de Resultados
    
% The \phantomsection command is needed to create a link to a place in the document that is not a
% figure, equation, table, section, subsection, chapter, etc.
% https://tex.stackexchange.com/questions/44088/when-do-i-need-to-invoke-phantomsection
\phantomsection

% Multiple-language document - babel - selectlanguage vs begin/end{otherlanguage}
% https://tex.stackexchange.com/questions/36526/multiple-language-document-babel-selectlanguage-vs-begin-endotherlanguage
\begin{otherlanguage*}{english}

    \chapter
    [Some very big title]
    {Some very big title you are cutting it out so if this nice on the table of contents}

    \begin{flushright}

    \end{flushright}

    % \newpage
    \section[Some encoding tests]{ }

    Nutrition foot carrots and salad deductible hydrogen

    Nutrition foot carrots

    Lipsum me [24-26]

    \newpage

\end{otherlanguage*}



    % Finaliza a parte no bookmark do PDF para que se inicie o bookmark na raiz
    % e adiciona espaço de parte no Sumário
    \phantompart

    % Conclusão (outro exemplo de capítulo sem numeração e presente no sumário)
    
% The \phantomsection command is needed to create a link to a place in the document that is not a
% figure, equation, table, section, subsection, chapter, etc.
% https://tex.stackexchange.com/questions/44088/when-do-i-need-to-invoke-phantomsection
\phantomsection

% ---
\chapter{Considerações Finais}
\phantomsection

    Lipsum me [31-33]



    % ELEMENTOS PÓS-TEXTUAIS
    \postextual
    \setlength\beforechapskip{0pt}
    \setlength\midchapskip{15pt}
    \setlength\afterchapskip{15pt}

    % Referências bibliográficas
    \begingroup
        % https://tex.stackexchange.com/questions/163559/how-to-modify-line-spacing-per-entry-of-bibliography
        % \linespread{1.18}\selectfont

        % https://tex.stackexchange.com/questions/17128/using-bibtex-to-make-a-list-of-references-without
        % \nocite{*}
        \printbibliography[title=REFERÊNCIAS]
    \endgroup

    % Glossário, consulte o manual da classe abntex2 para orientações sobre o glossário.
    % \ifforcedinclude\else\glossary\fi

    % Inicia os apêndices
    \begin{apendicesenv}
        % Imprime uma página indicando o início dos apêndices
        \ifforcedinclude\else\partapendices\fi
        \setlength\beforechapskip{50pt}
        \setlength\midchapskip{20pt}
        \setlength\afterchapskip{20pt}

        


%
% How to fix the Underfull \vbox badness has occurred while \output is active on my memoir chapter style?
% https://tex.stackexchange.com/questions/387881/how-to-fix-the-underfull-vbox-badness-has-occurred-while-output-is-active-on-m
%

% ---

\chapter[Página não gerada]{Como esta página não está sendo completamente preenchida, ele está gerando a caixa inferior inferior da página}
% ---


% Multiple-language document - babel - selectlanguage vs begin/end{otherlanguage}
% https://tex.stackexchange.com/questions/36526/multiple-language-document-babel-selectlanguage-vs-begin-endotherlanguage
\begin{otherlanguage*}{english}

\englishword{\showfont}

1. How to display the font size in use in the final output,
2. How to display the font size in use in the final output,
3. How to display the font size in use in the final output,
4. How to display the font size in use in the final output,
5. How to display the font size in use in the final output,
6. How to display the font size in use in the final output,
7. How to display the font size in use in the final output,
8. How to display the font size in use in the final output,
9. How to display the font size in use in the final output,


% As this page is not being completely filled, it is generating the page bottom bad box.
% Fix Underfull \vbox (badness 10000) has occurred while \output is active
%
% \flushbottom vs \raggedbottom
% https://tex.stackexchange.com/questions/65355/flushbottom-vs-raggedbottom
\newpage



\section[Some encoding tests]{\englishword{\showfont}}

1. How to display the font size in use in the final output,
2. How to display the font size in use in the final output,
3. How to display the font size in use in the final output,
4. How to display the font size in use in the final output,
5. How to display the font size in use in the final output,
6. How to display the font size in use in the final output,

7. How to display the font size in use in the final output,
8. How to display the font size in use in the final output,
9. How to display the font size in use in the final output,
10. How to display the font size in use in the final output,
11. How to display the font size in use in the final output,
12. How to display the font size in use in the final output,

\subsection{\englishword{\showfont}}

1. How to display the font size in use in the final output,
2. How to display the font size in use in the final output,
3. How to display the font size in use in the final output,
4. How to display the font size in use in the final output,
5. How to display the font size in use in the final output,
6. How to display the font size in use in the final output,

7. How to display the font size in use in the final output,
8. How to display the font size in use in the final output,
9. How to display the font size in use in the final output,
10. How to display the font size in use in the final output,
11. How to display the font size in use in the final output,
12. How to display the font size in use in the final output,

\subsubsection{\englishword{\showfont}}

1. How to display the font size in use in the final output,
2. How to display the font size in use in the final output,
3. How to display the font size in use in the final output,
4. How to display the font size in use in the final output,
5. How to display the font size in use in the final output,
6. How to display the font size in use in the final output,

7. How to display the font size in use in the final output,
8. How to display the font size in use in the final output,
9. How to display the font size in use in the final output,
10. How to display the font size in use in the final output,
11. How to display the font size in use in the final output,
12. How to display the font size in use in the final output,

\subsubsubsection{\englishword{\showfont}}

1. How to display the font size in use in the final output,
2. How to display the font size in use in the final output,
3. How to display the font size in use in the final output,
4. How to display the font size in use in the final output,
5. How to display the font size in use in the final output,
6. How to display the font size in use in the final output,
7. How to display the font size in use in the final output,

8. How to display the font size in use in the final output,
9. How to display the font size in use in the final output,
10. How to display the font size in use in the final output,
11. How to display the font size in use in the final output,
12. How to display the font size in use in the final output,


Lipsum me [31-35]

\end{otherlanguage*}



    \end{apendicesenv}

    % Inicia os anexos
    \begin{anexosenv}
        % Imprime uma página indicando o início dos anexos
        \ifforcedinclude\else\partanexos\fi
        \setlength\beforechapskip{50pt}
        \setlength\midchapskip{20pt}
        \setlength\afterchapskip{20pt}

        

%
% How to fix the Underfull \vbox badness has occurred while \output is active on my memoir chapter style?
% https://tex.stackexchange.com/questions/387881/how-to-fix-the-underfull-vbox-badness-has-occurred-while-output-is-active-on-m
%

% ----------------------------------------------------------
\chapter{Artigo publicado}
% ----------------------------------------------------------


% Multiple-language document - babel - selectlanguage vs begin/end{otherlanguage}
% https://tex.stackexchange.com/questions/36526/multiple-language-document-babel-selectlanguage-vs-begin-endotherlanguage
\begin{otherlanguage*}{english}

% An environment for setting \emergencystretch locally
% https://tex.stackexchange.com/questions/84510/an-environment-for-setting-emergencystretch-locally
{
    \setlength{\emergencystretch}{10pt}
    \section[English guidelines for publication]
    {English guidelines for publication - TITLE HERE (14 PT TYPE SIZE, UPPERCASE, BOLD, CENTERED)}
}
    \noindent\textbf{Abstract:}
    The objective of this document is to instruct the authors about the preparation of the
    manuscript for its submission to the Revista Eletrônica de Potência (Brazilian Power Electronics
    Journal).~The authors should use these guidelines for preparing both the initial and final
    versions of their paper. Additional information about procedures and guidelines for publication
    can be obtained directly with the editor, or through the web site
    \url{http://www.sobraep.org.br/revista}. This text was written according to these guidelines

\end{otherlanguage*}

% What is a “Overfull \hbox (9.89561pt too wide)”?
% https://tex.stackexchange.com/questions/111948/what-is-a-overfull-hbox-9-89561pt-too-wide
interwordspace: \the\fontdimen2\font

interwordstretch: \the\fontdimen3\font

emergencystretch: \the\emergencystretch\par\relax


\modifiedincludepdf{-}{ArtigoSOBRAEP}{pictures/SOBRAEP.pdf}{0.9}



        


%
% How to fix the Underfull \vbox badness has occurred while \output is active on my memoir chapter style?
% https://tex.stackexchange.com/questions/387881/how-to-fix-the-underfull-vbox-badness-has-occurred-while-output-is-active-on-m
%

% ----------------------------------------------------------
\chapter[Anexo exemplo]{Como exibir o tamanho da fonte em uso na saída final}
% ----------------------------------------------------------


% Multiple-language document - babel - selectlanguage vs begin/end{otherlanguage}
% https://tex.stackexchange.com/questions/36526/multiple-language-document-babel-selectlanguage-vs-begin-endotherlanguage
\begin{otherlanguage*}{english}

\englishword{\showfont}

1. How to display the font size in use in the final output,
2. How to display the font size in use in the final output,
3. How to display the font size in use in the final output,


\section[Some encoding tests]{\englishword{\showfont}}

1. How to display the font size in use in the final output,
2. How to display the font size in use in the final output,
3. How to display the font size in use in the final output,
4. How to display the font size in use in the final output,
5. How to display the font size in use in the final output,
6. How to display the font size in use in the final output,

7. How to display the font size in use in the final output,
8. How to display the font size in use in the final output,
9. How to display the font size in use in the final output,
10. How to display the font size in use in the final output,
11. How to display the font size in use in the final output,
12. How to display the font size in use in the final output,

\subsection{\englishword{\showfont}}

1. How to display the font size in use in the final output,
2. How to display the font size in use in the final output,
3. How to display the font size in use in the final output,
4. How to display the font size in use in the final output,
5. How to display the font size in use in the final output,
6. How to display the font size in use in the final output,

7. How to display the font size in use in the final output,
8. How to display the font size in use in the final output,
9. How to display the font size in use in the final output,
10. How to display the font size in use in the final output,
11. How to display the font size in use in the final output,
12. How to display the font size in use in the final output,

\subsubsection{\englishword{\showfont}}

1. How to display the font size in use in the final output,
2. How to display the font size in use in the final output,
3. How to display the font size in use in the final output,
4. How to display the font size in use in the final output,
5. How to display the font size in use in the final output,
6. How to display the font size in use in the final output,

7. How to display the font size in use in the final output,
8. How to display the font size in use in the final output,
9. How to display the font size in use in the final output,
10. How to display the font size in use in the final output,
11. How to display the font size in use in the final output,
12. How to display the font size in use in the final output,

\subsubsubsection{\englishword{\showfont}}

1. How to display the font size in use in the final output,
2. How to display the font size in use in the final output,
3. How to display the font size in use in the final output,
4. How to display the font size in use in the final output,
5. How to display the font size in use in the final output,
6. How to display the font size in use in the final output,
7. How to display the font size in use in the final output,

8. How to display the font size in use in the final output,
9. How to display the font size in use in the final output,
10. How to display the font size in use in the final output,
11. How to display the font size in use in the final output,
12. How to display the font size in use in the final output,


Lipsum me [55-65]

\end{otherlanguage*}



    \end{anexosenv}

    % INDICE REMISSIVO
    \ifforcedinclude\else
        \phantompart
        \printindex
    \fi

\end{document}

