%% abtex2-modelo-trabalho-academico.tex, v-1.9.2 laurocesar
%% Copyright 2012-2014 by abnTeX2 group at http://abntex2.googlecode.com/ 
%%
%% This work may be distributed and/or modified under the
%% conditions of the LaTeX Project Public License, either version 1.3
%% of this license or (at your option) any later version.
%% The latest version of this license is in
%%   http://www.latex-project.org/lppl.txt
%% and version 1.3 or later is part of all distributions of LaTeX
%% version 2005/12/01 or later.
%%
%% This work has the LPPL maintenance status `maintained'.
%% 
%% The Current Maintainer of this work is the abnTeX2 team, led
%% by Lauro César Araujo. Further information are available on 
%% http://abntex2.googlecode.com/
%%
%% This work consists of the files abntex2-modelo-trabalho-academico.tex,
%% abntex2-modelo-include-comandos and abntex2-modelo-references.bib
%%

% ------------------------------------------------------------------------
% ------------------------------------------------------------------------
% abnTeX2: Modelo de Trabalho Academico (tese de doutorado, dissertacao de
% mestrado e trabalhos monograficos em geral) em conformidade com 
% ABNT NBR 14724:2011: Informacao e documentacao - Trabalhos academicos -
% Apresentacao
% ------------------------------------------------------------------------
% ------------------------------------------------------------------------

\documentclass[
10.5pt, % Tamanho da fonte 
a5paper, % Tamanho do papel
twoside, % Impressão nos dois lados da folha
english,
brazil,
%sumario=tradicional,
%sumario=abnt-6027-2012, %  soh pode ser usado com memoir v3.6k ou superior
sumario=UFSC,
chapter=TITLE, % Título de capítulos em caixa alta
section=TITLE  % Título de seções em caixa alta
]{ufsc-inep-thesis}


% ---
% Pacotes básicos 
% ---

% ---
% Pacotes de citações
% ---
\usepackage[brazilian,hyperpageref]{backref}	 % Paginas com as citações na bibl
%\usepackage[alf]{abntex2cite}	% Citação alfabética por autor-data [alf]
\usepackage[num]{abntex2cite} % Citação numérica [num]
\citebrackets[]

% --- 
% CONFIGURAÇÕES DE PACOTES
% --- 

% ---
% Configurações do pacote backref
% Usado sem a opção hyperpageref de backref
\renewcommand{\backrefpagesname}{Citado na(s) página(s):~}
% Texto padrão antes do número das páginas
\renewcommand{\backref}{}
% Define os textos da citação
\renewcommand*{\backrefalt}[4]{
	\ifcase #1 %
		Nenhuma citação no texto.%
	\or
		Citado na página #2.%
	\else
		Citado #1 vezes nas páginas #2.%
	\fi}%


% ---
% Informações de dados para CAPA e FOLHA DE ROSTO
% ---

% ---
% Informações de dados para CAPA e FOLHA DE ROSTO
% ---
\titulo{Modelo Canônico de Tese, Dissertação ou TCC	 do INEP com \abnTeX}
\autor{Nome do Autor}
\local{Florianópolis, Santa Catarina -- Brasil}
%\local{Florianópolis}
\data{\today}
\orientador{Prof. Dr. Nome do Orientador}
\coorientador{Prof. Dr. Nome do Coorientador}
\instituicao{%
  Universidade Federal de Santa Catarina -- UFSC
  \par
  Departamento de Engenharia Elétrica -- EEL
  \par
 Programa de Pós--Graduação em Engenharia Elétrica -- PGEEL}
\tipotrabalho{Tese (Doutorado)}
% O preambulo deve conter o tipo do trabalho, o objetivo, 
% o nome da instituição e a área de concentração 
\preambulo{Tese submetida ao Programa de Pós--Graduação em Engenharia Elétrica da Universidade Federal de Santa Catarina para a obtenção do Grau de Doutor em Engenharia Elétrica.}
% ---


% ---
% Configurações de aparência do PDF final
% ---
% Configurações de aparência do PDF final


%\definecolor{figcolor}{rgb}{1,0.4,0}  % orange
%\definecolor{tabcolor}{rgb}{1,0.4,0}  % orange
%\definecolor{eqncolor}{rgb}{1,0.4,0}  % orange
%\definecolor{linkcolor}{rgb}{1,0.4,0}  % orange
%\definecolor{citecolor}{rgb}{1,0.4,0}  % orange
%\definecolor{seccolor}{rgb}{0,0,1}  % blue
%\definecolor{abscolor}{rgb}{0,0,1}  % blue
%\definecolor{titlecolor}{rgb}{0,0,1}  % blue
%\definecolor{biocolor}{rgb}{0,0,1}  % blue

% alterando o aspecto da cor azul
\definecolor{blue}{RGB}{41,5,195}

% informações do PDF
\makeatletter
\hypersetup{
	%pagebackref=true,
	pdftitle={\@title}, 
	pdfauthor={Autores},
	pdfsubject={Resumo do artigo},
	pdfcreator={LaTeX with abnTeX2},
	pdfkeywords={abnt}{latex}{UFSC}{abntex2}{tese}{INEP}, 
	colorlinks=true,       		% false: boxed links; true: colored links
	linkcolor=linkcolor,          	% color of internal links
	citecolor=citecolor,        		% color of links to bibliography
	filecolor=black,      		% color of file links
	urlcolor=linkcolor,
	bookmarksdepth=4
}
\makeatother

% --- 


% ---
% Estilo de capítulos
%
%\chapterstyle{default}
% \chapterstyle{pedersen} 
%\chapterstyle{lyhne} 
%\chapterstyle{madsen} 
% \chapterstyle{veelo} 
%\chapterstyle{companion}

%\chapterstyle{thatcher}
%\chapterstyle{verville}

\chapterstyle{VZ14} % Ver classe para maiores detalhes


%
% Veja outros estilos em:
% http://www.tex.ac.uk/tex-archive/info/MemoirChapStyles/MemoirChapStyles.pdf
% ---




% --- 
% Espaçamentos entre linhas e parágrafos 
% --- 

% O tamanho do parágrafo é dado por:
\setlength{\parindent}{1.3cm}

% Controle do espaçamento entre um parágrafo e outro:
\setlength{\parskip}{0.2cm}  % tente também \onelineskip

% ---
% compila o indice
% ---
\makeindex
% ---


%\includeonly{PreTexto/fichacatalografica}
%\includeonly{PreTexto/agradecimentos}
%\includeonly{PreTexto/resumos}
%\includeonly{PreTexto/siglas}
%\includeonly{PreTexto/simbolos}
%
%\includeonly{Capitulos/00/CH00}
%\includeonly{Capitulos/01/CH01}
%\includeonly{Capitulos/02/CH02}
%\includeonly{Capitulos/03/CH03}
%\includeonly{Capitulos/04/CH04}



\begin{document}

% Retira espaço extra obsoleto entre as frases.
\frenchspacing 

% ----------------------------------------------------------
% ELEMENTOS PRÉ-TEXTUAIS
% ----------------------------------------------------------
% \pretextual

% ---
% Capa
% ---
\imprimircapa
% ---

% ---
% Folha de rosto
% (o * indica que haverá a ficha bibliográfica)
% ---
\imprimirfolhaderosto*
% ---

% ---
% Inserir a ficha bibliografica
% ---

% Isto é um exemplo de Ficha Catalográfica, ou ``Dados internacionais de
% catalogação-na-publicação''. Você pode utilizar este modelo como referência. 
% Porém, provavelmente a biblioteca da sua universidade lhe fornecerá um PDF
% com a ficha catalográfica definitiva após a defesa do trabalho. Quando estiver
% com o documento, salve-o como PDF no diretório do seu projeto e substitua todo
% o conteúdo de implementação deste arquivo pelo comando abaixo:
%
% \begin{fichacatalografica}
%     \includepdf{PreTexto/ficha_catalografica.pdf}     
% \end{fichacatalografica}




\begin{fichacatalografica}
	\vspace*{\fill}					% Posição vertical
%	\hrule							% Linha horizontal
	\begin{center}					% Minipage Centralizado
		

		\hspace*{1cm}
{Catalogação na fonte pela Biblioteca Universitária da
	Universidade Federal de Santa Catarina.}
		
		Arquivo compilado às \currenttime h do dia \today.
		
	\framebox[12.5cm]{
	\begin{minipage}{\textwidth}		% Largura		
	\ttfamily
	\imprimirautor
	
	\hspace{0.5cm} \imprimirtitulo  / \imprimirautor. --
	\imprimirlocal, \imprimirdata-
	
	\hspace{0.5cm} \pageref{LastPage} p. : il. (algumas color.) ; 30 cm.\\
	
	\hspace{0.5cm} \imprimirorientadorRotulo~\imprimirorientador\\
	
	\hspace{0.5cm}
	\parbox[t]{\textwidth}{\imprimirtipotrabalho~--~\imprimirinstituicao,
	\imprimirdata.}\\
	
	\hspace{0.5cm}
		1. Palavra-chave1.
		2. Palavra-chave2.
		I. Orientador.
		II. Universidade xxx.
		III. Faculdade de xxx.
		IV. Título\\ 			
	
	\hspace{7.75cm} CDU 02:141:005.7\\
	
	\end{minipage}
}
	\end{center}
%	\hrule
\end{fichacatalografica}


%% Ficha catalográfica
%\newcounter{ufscthesis@indiceficha}
%
%\newcommand{\imprimirfichacatalografica}{%
%  \setcounter{ufscthesis@indiceficha}{1}
%  \begin{fichacatalografica}
%  \vspace*{\fill}
%  \fbox{%
%    \ttfamily%
%    \begin{minipage}[b][][t]{\textwidth}
%      \indent\ufscthesis@inverter{\imprimirautor} \newline%
%      \indent~\imprimirtitulo~:~/ \imprimirautor ; orientador, \imprimirorientador%
%      \ifnotempty{\imprimircoorientador}{~; co-orientador, \imprimircoorientador}.~-~\imprimirlocal~\the\year. \newline%
%      \indent~\pageref{LastPage} p. \newline \newline%
%      \indent- \imprimirinstituicao, \imprimircentro. \imprimirprograma. \newline \newline%
%      \indent Inclui Refer\^{e}ncias \newline \newline%
%      \indent \imprimirassuntos
%      \Roman{ufscthesis@indiceficha}. \ufscthesis@inverter{\imprimirorientador}. \stepcounter{ufscthesis@indiceficha}
%      \ifnotempty{\imprimircoorientador}{\Roman{ufscthesis@indiceficha}. \ufscthesis@inverter{\imprimircoorientador}.\stepcounter{ufscthesis@indiceficha}}
%      \Roman{ufscthesis@indiceficha}. \imprimirinstituicao. \imprimirprograma.
%      \Roman{ufscthesis@indiceficha}. \imprimirtitulo.
%    \end{minipage}
%  }
%  \end{fichacatalografica}


% ---
% Inserir errata
% ---


% ---
% Inserir folha de aprovação
% ---

% Isto é um exemplo de Folha de aprovação, elemento obrigatório da NBR
% 14724/2011 (seção 4.2.1.3). Você pode utilizar este modelo até a aprovação
% do trabalho. Após isso, substitua todo o conteúdo deste arquivo por uma
% imagem da página assinada pela banca com o comando abaixo:
%
% \includepdf{folhadeaprovacao_final.pdf}
%
\begin{folhadeaprovacao}
	
	\begin{center}
		{\ABNTEXchapterfont\large\imprimirautor}
		
		%    \vspace*{\fill}\vspace*{\fill}
		\begin{center}
			\ABNTEXchapterfont\bfseries\Large\imprimirtitulo
		\end{center}
		%    \vspace*{\fill}
		    
		%    \hspace{.45\textwidth}
		\begin{minipage}{\textwidth}
			Esta Tese foi julgada adequada para obtenção do Título de Doutor em Engenharia Elétrica, na área de concentração em Eletrônica de Potência e Acionamento Elétrico, e aprovada em sua forma final pelo Programa de Pós--Graduação em Engenharia Elétrica da Universidade Federal de Santa Catarina.
		\end{minipage}%
		
	\end{center}
	\begin{center}
		Florianópolis, \imprimirdata.
	\end{center}
	   
	\assinatura{\textbf{Prof. Carlos Galup Montoro, Dr.} \\ Coordenador do Programa de Pós--Graduação em Engenharia Elétrica} 
	\begin{flushleft}  
		\textbf{Banca Examinadora:}   
	\end{flushleft}
	\assinatura{\textbf{\imprimirorientador} \\ Orientador \\ Universidade Federal de Santa Catarina -- UFSC} 
	\assinatura{\textbf{\imprimircoorientador} \\ Coorientador \\ Universidade Federal de Santa Catarina -- UFSC}    
	\assinatura{\textbf{Prof. Enio Valmor Kassik, Dr.} \\ Membro externo \\ Instituto Federal de Santa Catarina -- IFSC}   
	\assinatura{\textbf{Prof. Arnaldo José Perin, Dr.} \\ Universidade Federal de Santa Catarina -- UFSC}
	\assinatura{\textbf{Prof. Denizar Cruz Martins, Dr.} \\ Universidade Federal de Santa Catarina -- UFSC}
	\assinatura{\textbf{Prof. Marcelo Lobo Heldwein, Dr.} \\ Universidade Federal de Santa Catarina -- UFSC}
	\assinatura{\textbf{Prof. Roberto Francisco Coelho, Dr.} \\ Universidade Federal de Santa Catarina -- UFSC}
	\assinatura{\textbf{Prof. Samir Ahmad Mussa, Dr.} \\ Universidade Federal de Santa Catarina -- UFSC}
	\assinatura{\textbf{Prof. Telles Brunelli Lazzarin, Dr.} \\ Universidade Federal de Santa Catarina -- UFSC}
	  
\end{folhadeaprovacao}
% ---

% ---
% Dedicatória
% ---
\begin{dedicatoria}
	\vspace*{\fill}
	\centering
	\noindent
	\textit{ Este trabalho é dedicado às crianças adultas que,\\
		quando pequenas, sonharam em se tornar cientistas.} \vspace*{\fill}
\end{dedicatoria}
% ---

% ---
% Agradecimentos
% ---
\begin{agradecimentos}
	
Os agradecimentos principais são direcionados à Gerald Weber, Miguel Frasson,
Leslie H. Watter, Bruno Parente Lima, Flávio de Vasconcellos Corrêa, Otavio Real
Salvador, Renato Machnievscz\footnote{Os nomes dos integrantes do primeiro
projeto abn\TeX\ foram extraídos de
\url{http://codigolivre.org.br/projects/abntex/}} e todos aqueles que
contribuíram para que a produção de trabalhos acadêmicos conforme
as normas ABNT com \LaTeX\ fosse possível.

Agradecimentos especiais são direcionados ao Centro de Pesquisa em Arquitetura
da Informação\footnote{\url{http://www.cpai.unb.br/}} da Universidade de
Brasília (CPAI), ao grupo de usuários
\emph{latex-br}\footnote{\url{http://groups.google.com/group/latex-br}} e aos
novos voluntários do grupo
\emph{\abnTeX}\footnote{\url{http://groups.google.com/group/abntex2} e
\url{http://abntex2.googlecode.com/}}~que contribuíram e que ainda
contribuirão para a evolução do \abnTeX.

\end{agradecimentos}
% ---

% ---
% Epígrafe
% ---

\begin{epigrafe}
	\vspace*{\fill}
	\begin{flushright}
		\textit{``Only two things are infinite: the universe and human stupidity and I'm not sure about the former''}\\Albert Einstein
	\end{flushright}
	\begin{flushright}
		\textit{``É fácil ser humilde para quem nunca\\ fez nada do que possa se orgulhar''}\\Jackfilho Lake
	\end{flushright}
	\begin{flushright}
		\textit{``Learn from yesterday, live for today, hope for tomorrow. The important thing is not to stop questioning.''}\\	Albert Einstein		
	\end{flushright}
	\begin{flushright}
		\textit{``The true sign of intelligence is not knowledge but imagination.''}\\	Albert Einstein		
	\end{flushright}
	\begin{flushright}
		\textit{``Peace cannot be kept by force; it can only be achieved by understanding.''}\\	Albert Einstein		
	\end{flushright}
	\begin{flushright}
		\textit{``Whoever is careless with the truth in small matters cannot be trusted with important matters.''}\\	Albert Einstein		
	\end{flushright}		
	\begin{flushright}
		\textit{``Extraordinary claims require extraordinary evidence''}\\
		Carl Sagan
	\end{flushright}
	\begin{flushright}
		\textit{``Bravura é vencer o medo encarando a verdade enquanto que estupidez é coibir a percepção ignorando os fatos. Ambas são igualmente importantes para se conquistar a auto--confiança.''}\\
		Adriano Ruseler
	\end{flushright}
%	\begin{flushright}
%		\textit{``Pesquisa cientifica é a arte isolar o ego e as crenças da interpretação dos fatos.''}\\
%		Adriano Ruseler
%	\end{flushright}
%	\begin{flushright}
%		\textit{``A verdadeira motivação de um pesquisador está em desenvolver suas próprias idéias seguindo a sua intuição.''}\\
%		Adriano Ruseler
%	\end{flushright}	
	
%	\begin{flushright}
%		\textit{``O verdadeiro valor de uma tese não está na sua utilidade, mas sim na sua capacidade de trazer esperança a um jovem pesquisador.''}\\
%		Adriano Ruseler
%	\end{flushright}	
				
	\begin{flushright}
		\textit{``Catholic, which I was until I reached the age of reason.''}\\
		George Carlin
	\end{flushright}	
	\begin{flushright}
		\textit{``We made too many wrong mistakes.''}\\
		Yogi Berra
	\end{flushright}	
					
\end{epigrafe}
% ---

% ---
% RESUMOS
% ---
\setlength{\absparsep}{18pt} % ajusta o espaçamento dos parágrafos do resumo


% resumo em português
\begin{resumo}
	Segundo a \citeonline[3.1-3.2]{NBR6028:2003}, o resumo deve ressaltar o
	objetivo, o método, os resultados e as conclusões do documento. A ordem e a extensão
	destes itens dependem do tipo de resumo (informativo ou indicativo) e do
	tratamento que cada item recebe no documento original. O resumo deve ser
	precedido da referência do documento, com exceção do resumo inserido no
	próprio documento. (\ldots) As palavras-chave devem figurar logo abaixo do
	resumo, antecedidas da expressão Palavras-chave:, separadas entre si por
	ponto e finalizadas também por ponto.
	
	\textbf{Palavras-chaves}: latex. abntex. editoração de texto.
\end{resumo}




% resumo em inglês
\begin{resumo}[Abstract]
	\begin{otherlanguage*}{english}
		This is the english abstract.
		
		\vspace{\onelineskip}
		
		\noindent 
		\textbf{Key-words}: latex. abntex. text editoration.
	\end{otherlanguage*}
\end{resumo}

%
%
%% resumo em francês 
%\begin{resumo}[Résumé]
%	\begin{otherlanguage*}{french}
%		Il s'agit d'un résumé en français.
%		
%		\textbf{Mots-clés}: latex. abntex. publication de textes.
%	\end{otherlanguage*}
%\end{resumo}
%
%
%% resumo em espanhol
%\begin{resumo}[Resumen]
%	\begin{otherlanguage*}{spanish}
%		Este es el resumen en español.
%		
%		\textbf{Palabras clave}: latex. abntex. publicación de textos.
%	\end{otherlanguage*}
%\end{resumo}
% ---





% ---
% inserir lista de ilustrações
% ---
\pdfbookmark[0]{\listfigurename}{lof}
\listoffigures*
\cleardoublepage
% ---

% ---
% inserir lista de tabelas
% ---
\pdfbookmark[0]{\listtablename}{lot}
\listoftables*
\cleardoublepage
% ---

% ---
% inserir códigos fonte
% ---
\pdfbookmark[0]{\lstlistingname}{lol}
\lstlistoflistings* 
\cleardoublepage
% ---


% ---
% inserir lista de abreviaturas e siglas
% ---

\begin{siglas}
	\item[ABNT] Associação Brasileira de Normas Técnicas
	\item[abnTeX] ABsurdas Normas para TeX
\end{siglas}


% ---

% ---
% inserir lista de símbolos
% ---

% Devam aparecer na mesma ordem de ocorrência no texto.

\begin{simbolos}
	\item[$ \Gamma $] Letra grega Gama
	\item[$ \Lambda $] Lambda
	\item[$ \zeta $] Letra grega minúscula zeta
	\item[$ \in $] Pertence
\end{simbolos}
% ---


% ---
% inserir o sumario
% ---
\pdfbookmark[0]{\contentsname}{toc}
\tableofcontents*
\cleardoublepage
% ---


% ----------------------------------------------------------
% ELEMENTOS TEXTUAIS
% ----------------------------------------------------------

\textual % Configura estilo das páginas.

%\textualINEPUFSC  % Configura estilo das páginas com logos

% ----------------------------------------------------------
% Introdução (exemplo de capítulo sem numeração, mas presente no Sumário)
% ----------------------------------------------------------


% ----------------------------------------------------------
% Introdução (exemplo de capítulo sem numeração, mas presente no Sumário)
% ----------------------------------------------------------

\chapter[Introdução]{Introdução \showfont}
%\addcontentsline{toc}{chapter}{Introdução}
% ----------------------------------------------------------
\showfont
\section[Some encoding tests]{\showfont}
\subsection{\showfont}
\subsubsection{\showfont}
\subsubsubsection{\showfont}


\textsf{textsf: \showfont} 

\textrm{textrm: \showfont}

\textnormal{textnormal: \showfont}

\textbf{textbf: \showfont} 

\textit{textit: \showfont}

footnote\footnote{\showfont}

% \emph{emphasize: \showfont} 

``Modelo Canônico \showfont''

\begin{itemize}
\item \textrm{Roman family - \showfont }
\item \textsf{Sans serif family - \showfont}
\item \texttt{Typewriter/teletype family - \showfont}
\item \textit{italics text - \showfont}
\item \textsl{slanted text- \showfont}
\item \textsc{small caps text- \showfont}	
\end{itemize}




mathnormal -  default: $\mathnormal{abcXYZ}$

mathrm - roman: $\mathrm{abcXYZ}$

mathbf - bold roman: $\mathbf{abcXYZ}$

mathsf - sans serif: $\mathsf{abcXYZ}$

mathit - text italic: $\mathit{abcXYZ}$

mathtt -  typewriter: $\mathtt{abcXYZ}$

mathcal - calligraphic: $\mathcal{XYZ}$


%\pagevalues

\begin{figure}
	\caption{Page layout for this document -  \showfont} \label{fig:ptrs}
	%\drawpage
     \setlayoutscale{0.4}
%	\currentpage
	\drawparameterstrue
	\drawpage
%	\pagedesign
%	\pagevalues
\end{figure}


\begin{figure}
	\caption{Page layout values for this document} \label{fig:ptrsval}

	\pagevalues
\end{figure}



\begin{figure}
	\currentfootnote
	\drawparameterstrue
%	\setlayoutscale{0.4}
	\drawfootnote
	\footnotevalues
	\caption{The current footnote layout}\label{fig:ftry}
\end{figure}	
	

\begin{figure}
	\drawparagraph
	\paragraphvalues
	\caption{Paragraph parameters}\label{fig:fpara}
\end{figure}


\begin{figure}
	\setlayoutscale{0.6}
    \drawparameterstrue
	\drawtoc
	\tocvalues
	\caption{Table of Contents entry parameters}\label{fig:tocp}
\end{figure}





A Tabela~\ref{tab:a} mostra mais informações do template BU.

\begin{table}[!htb]
	\begin{center}
		\caption{Formatação do texto. \showfont}
		\label{tab:a}
		\begin{tabular}{ p{3cm} | p{6cm} }
			\hline
			Cor & Branco - \showfont\\ \hline
			Formato do papel & A5\\ \hline
			Gramatura & 75\\ \hline
			Impressão & Frente e verso\\ \hline
			Margens & Espelhadas: superior 2, Inferior: 1,5, Externa 1,5 e Externa: 2.\\ \hline
			Cabeçalho & 0,7\\ \hline
			Rodapé & 0,7\\ \hline
			Paginação & Externa\\ \hline
			Alinhamento vertical & Superior\\ \hline
			Alinhamento do texto & Justificado\\ \hline
			Fonte sugerida & Times New Roman \\ \hline
			Tamanho da fonte & 10,5 para o texto incluindo os títulos das seções e subseções. As citações com mais de três linhas as legendas das ilustrações e tabelas, fonte 9,5.\\ \hline
			Espaçamento entre linhas & Um (1) simples\\ \hline
			Espaçamento entre parágrafos & Anterior 0,0; Posterior 0,0\\ \hline
			Numeração da seção & As seções  primárias devem  começar  sempre em páginas ímpares. Deixar um espaço (simples) entre o título da seção e o texto e  entre o texto e o título da subseção. \\  \hline
		\end{tabular}
	\end{center}
	Fonte: Universidade Federal de Santa Catarina (2011) \showfont
\end{table}


\begin{figure}
\centering
\includegraphics[width=\linewidth]{Capitulos/00/Figs/ex01}
\caption{Exemplo de figura}
\label{fig:ex01}
\end{figure}

\begin{figure}
\centering
\includegraphics[width=0.9\linewidth]{Capitulos/00/Figs/tek0009}
\caption{Exemplo de aquisição}
\label{fig:tek0009}
\end{figure}

Este documento e seu código-fonte são exemplos de referência de uso da classe
\textsf{abntex2} e do pacote \textsf{abntex2cite}. O documento 
exemplifica a elaboração de trabalho acadêmico (tese, dissertação e outros do
gênero) produzido conforme a ABNT NBR 14724:2011 \emph{Informação e documentação
	- Trabalhos acadêmicos - Apresentação}.

A expressão ``Modelo Canônico'' é utilizada para indicar que \abnTeX\ não é
modelo específico de nenhuma universidade ou instituição, mas que implementa tão
somente os requisitos das normas da ABNT. Uma lista completa das normas
observadas pelo \abnTeX\ é apresentada em \citeonline{abntex2classe}.

Sinta-se convidado a participar do projeto \abnTeX! Acesse o site do projeto em
\url{http://abntex2.googlecode.com/}. Também fique livre para conhecer,
estudar, alterar e redistribuir o trabalho do \abnTeX, desde que os arquivos
modificados tenham seus nomes alterados e que os créditos sejam dados aos
autores originais, nos termos da ``The \LaTeX\ Project Public
License''\footnote{\url{http://www.latex-project.org/lppl.txt}}.

Encorajamos que sejam realizadas customizações específicas deste exemplo para
universidades e outras instituições --- como capas, folha de aprovação, etc.
Porém, recomendamos que ao invés de se alterar diretamente os arquivos do
\abnTeX, distribua-se arquivos com as respectivas customizações.
Isso permite que futuras versões do \abnTeX~não se tornem automaticamente
incompatíveis com as customizações promovidas. Consulte
\citeonline{abntex2-wiki-como-customizar} par mais informações.

Este documento deve ser utilizado como complemento dos manuais do \abnTeX\ 
\cite{abntex2classe,abntex2cite,abntex2cite-alf} e da classe \textsf{memoir}
\cite{memoir}. 

Esperamos, sinceramente, que o \abnTeX\ aprimore a qualidade do trabalho que
você produzirá, de modo que o principal esforço seja concentrado no principal:
na contribuição científica.

Equipe \abnTeX 

Lauro César Araujo




% ----------------------------------------------------------
% PARTE
% ----------------------------------------------------------
%\part{Preparação da pesquisa}
% ----------------------------------------------------------

% ---
% Capitulo com exemplos de comandos inseridos de arquivo externo 
% ---
%% abtex2-modelo-include-comandos.tex, v-1.9.2 laurocesar
%% Copyright 2012-2014 by abnTeX2 group at http://abntex2.googlecode.com/ 
%%
%% This work may be distributed and/or modified under the
%% conditions of the LaTeX Project Public License, either version 1.3
%% of this license or (at your option) any later version.
%% The latest version of this license is in
%%   http://www.latex-project.org/lppl.txt
%% and version 1.3 or later is part of all distributions of LaTeX
%% version 2005/12/01 or later.
%%
%% This work has the LPPL maintenance status `maintained'.
%% 
%% The Current Maintainer of this work is the abnTeX2 team, led
%% by Lauro César Araujo. Further information are available on 
%% http://abntex2.googlecode.com/
%%
%% This work consists of the files abntex2-modelo-include-comandos.tex
%% and abntex2-modelo-img-marca.pdf
%%

% ---
% Este capítulo, utilizado por diferentes exemplos do abnTeX2, ilustra o uso de
% comandos do abnTeX2 e de LaTeX.
% ---
 
\chapter{Resultados de comandos}\label{cap_exemplos}

\chapterprecis{Isto é uma sinopse de capítulo. A ABNT não traz nenhuma
normatização a respeito desse tipo de resumo, que é mais comum em romances 
e livros técnicos.}\index{sinopse de capítulo}

% ---
\section{Codificação dos arquivos: UTF8}
% ---

A codificação de todos os arquivos do \abnTeX\ é \texttt{UTF8}. É necessário que
você utilize a mesma codificação nos documentos que escrever, inclusive nos
arquivos de base bibliográficas |.bib|.

% ---
\section{Citações diretas}
\label{sec-citacao}
% ---

\index{citações!diretas}Utilize o ambiente \texttt{citacao} para incluir
citações diretas com mais de três linhas:

\begin{citacao}
As citações diretas, no texto, com mais de três linhas, devem ser
destacadas com recuo de 4 cm da margem esquerda, com letra menor que a do texto
utilizado e sem as aspas. No caso de documentos datilografados, deve-se
observar apenas o recuo \cite[5.3]{NBR10520:2002}.
\end{citacao}

Use o ambiente assim:

\small\begin{verbatim}
\begin{citacao}
As citações diretas, no texto, com mais de três linhas [...] deve-se observar
apenas o recuo \cite[5.3]{NBR10520:2002}.
\end{citacao}
\end{verbatim}

O ambiente \texttt{citacao} pode receber como parâmetro opcional um nome de
idioma previamente carregado nas opções da classe (\autoref{sec-hifenizacao}). Nesse
caso, o texto da citação é automaticamente escrito em itálico e a hifenização é
ajustada para o idioma selecionado na opção do ambiente. Por exemplo:

\begin{verbatim}
\begin{citacao}[english]
Text in English language in italic with correct hyphenation.
\end{citacao}
\end{verbatim}

Tem como resultado:

\begin{citacao}[english]
Text in English language in italic with correct hyphenation.
\end{citacao}

\index{citações!simples}Citações simples, com até três linhas, devem ser
incluídas com aspas. Observe que em \LaTeX as aspas iniciais são diferentes das
finais: ``Amor é fogo que arde sem se ver''.

% ---
\section{Notas de rodapé}
% ---

As notas de rodapé são detalhadas pela NBR 14724:2011 na seção 5.2.1\footnote{As
notas devem ser digitadas ou datilografadas dentro das margens, ficando
separadas do texto por um espaço simples de entre as linhas e por filete de 5
cm, a partir da margem esquerda. Devem ser alinhadas, a partir da segunda linha
da mesma nota, abaixo da primeira letra da primeira palavra, de forma a destacar
o expoente, sem espaço entre elas e com fonte menor
\citeonline[5.2.1]{NBR14724:2011}.}\footnote{Caso uma série de notas sejam
criadas sequencialmente, o \abnTeX\ instrui o \LaTeX\ para que uma vírgula seja
colocada após cada número do expoente que indica a nota de rodapé no corpo do
texto.}\footnote{Verifique se os números do expoente possuem uma vírgula para
dividi-los no corpo do texto.}. 


% ---
\section{Tabelas}
% ---

\index{tabelas}A \autoref{tab-nivinv} é um exemplo de tabela construída em
\LaTeX.

\begin{table}[htb]
\ABNTEXfontereduzida
\caption[Níveis de investigação]{Níveis de investigação.}
\label{tab-nivinv}
\resizebox{\textwidth}{!}{%
\begin{tabular}{p{2.6cm}|p{6.0cm}|p{2.25cm}|p{3.40cm}}
  %\hline
   \textbf{Nível de Investigação} & \textbf{Insumos}  & \textbf{Sistemas de Investigação}  & \textbf{Produtos}  \\
    \hline
    Meta-nível & Filosofia\index{filosofia} da Ciência  & Epistemologia &
    Paradigma  \\
    \hline
    Nível do objeto & Paradigmas do metanível e evidências do nível inferior &
    Ciência  & Teorias e modelos \\
    \hline
    Nível inferior & Modelos e métodos do nível do objeto e problemas do nível inferior & Prática & Solução de problemas  \\
   % \hline
\end{tabular}
}
\legend{Fonte: \citeonline{van86}}
\end{table}

Já a \autoref{tabela-ibge} apresenta uma tabela criada conforme o padrão do
\citeonline{ibge1993} requerido pelas normas da ABNT para documentos técnicos e
acadêmicos.

\begin{table}[htb]
\IBGEtab{%
  \caption{Um Exemplo de tabela alinhada que pode ser longa
  ou curta, conforme padrão IBGE.}%
  \label{tabela-ibge}
}{%
  \begin{tabular}{ccc}
  \toprule
   Nome & Nascimento & Documento \\
  \midrule \midrule
   Maria da Silva & 11/11/1111 & 111.111.111-11 \\
  \midrule 
   João Souza & 11/11/2111 & 211.111.111-11 \\
  \midrule 
   Laura Vicuña & 05/04/1891 & 3111.111.111-11 \\
  \bottomrule
\end{tabular}%
}{%
  \fonte{Produzido pelos autores.}%
  \nota{Esta é uma nota, que diz que os dados são baseados na
  regressão linear.}%
  \nota[Anotações]{Uma anotação adicional, que pode ser seguida de várias
  outras.}%
  }
\end{table}


% ---
\section{Figuras}
% ---

\index{figuras}Figuras podem ser criadas diretamente em \LaTeX,
como o exemplo da \autoref{fig_circulo}.

\begin{figure}[htb]
	\caption{\label{fig_circulo}A delimitação do espaço}
	\begin{center}
	    \setlength{\unitlength}{5cm}
		\begin{picture}(1,1)
		\put(0,0){\line(0,1){1}}
		\put(0,0){\line(1,0){1}}
		\put(0,0){\line(1,1){1}}
		\put(0,0){\line(1,2){.5}}
		\put(0,0){\line(1,3){.3333}}
		\put(0,0){\line(1,4){.25}}
		\put(0,0){\line(1,5){.2}}
		\put(0,0){\line(1,6){.1667}}
		\put(0,0){\line(2,1){1}}
		\put(0,0){\line(2,3){.6667}}
		\put(0,0){\line(2,5){.4}}
		\put(0,0){\line(3,1){1}}
		\put(0,0){\line(3,2){1}}
		\put(0,0){\line(3,4){.75}}
		\put(0,0){\line(3,5){.6}}
		\put(0,0){\line(4,1){1}}
		\put(0,0){\line(4,3){1}}
		\put(0,0){\line(4,5){.8}}
		\put(0,0){\line(5,1){1}}
		\put(0,0){\line(5,2){1}}
		\put(0,0){\line(5,3){1}}
		\put(0,0){\line(5,4){1}}
		\put(0,0){\line(5,6){.8333}}
		\put(0,0){\line(6,1){1}}
		\put(0,0){\line(6,5){1}}
		\end{picture}
	\end{center}
	\legend{Fonte: os autores}
\end{figure}

Ou então figuras podem ser incorporadas de arquivos externos, como é o caso da
\autoref{fig_grafico}. Se a figura que ser incluída se tratar de um diagrama, um
gráfico ou uma ilustração que você mesmo produza, priorize o uso de imagens
vetoriais no formato PDF. Com isso, o tamanho do arquivo final do trabalho será
menor, e as imagens terão uma apresentação melhor, principalmente quando
impressas, uma vez que imagens vetorias são perfeitamente escaláveis para
qualquer dimensão. Nesse caso, se for utilizar o Microsoft Excel para produzir
gráficos, ou o Microsoft Word para produzir ilustrações, exporte-os como PDF e
os incorpore ao documento conforme o exemplo abaixo. No entanto, para manter a
coerência no uso de software livre (já que você está usando \LaTeX e \abnTeX),
teste a ferramenta \textsf{InkScape}\index{InkScape}
(\url{http://inkscape.org/}). Ela é uma excelente opção de código-livre para
produzir ilustrações vetoriais, similar ao CorelDraw\index{CorelDraw} ou ao Adobe
Illustrator\index{Adobe Illustrator}. De todo modo, caso não seja possível
utilizar arquivos de imagens como PDF, utilize qualquer outro formato, como
JPEG, GIF, BMP, etc. Nesse caso, você pode tentar aprimorar as imagens
incorporadas com o software livre \textsf{Gimp}\index{Gimp}
(\url{http://www.gimp.org/}). Ele é uma alternativa livre ao Adobe
Photoshop\index{Adobe Photoshop}.

%\begin{figure}[htb]
%	\caption{\label{fig_grafico}Gráfico produzido em Excel e salvo como PDF}
%	\begin{center}
%	    \includegraphics[scale=0.35]{Capitulos/01/Figs/abntex2-modelo-img-grafico.pdf}
%	\end{center}
%	\legend{Fonte: \citeonline[p. 24]{araujo2012}}
%\end{figure}

% ---
\subsection{Figuras em \emph{minipages}}
% ---

\emph{Minipages} são usadas para inserir textos ou outros elementos em quadros
com tamanhos e posições controladas. Veja o exemplo da
\autoref{fig_minipage_imagem1} e da \autoref{fig_minipage_grafico2}.

%\begin{figure}[htb]
% \label{teste}
% \centering
%  \begin{minipage}{0.4\textwidth}
%    \centering
%    \caption{Imagem 1 da minipage} \label{fig_minipage_imagem1}
%    \includegraphics[scale=0.8]{Capitulos/01/Figs/abntex2-modelo-img-marca.pdf}
%    \legend{Fonte: Produzido pelos autores}
%  \end{minipage}
%  \hfill
%  \begin{minipage}{0.4\textwidth}
%    \centering
%    \caption{Grafico 2 da minipage} \label{fig_minipage_grafico2}
%    \includegraphics[scale=0.1]{Capitulos/01/Figs/abntex2-modelo-img-grafico.pdf}
%    \legend{Fonte: \citeonline[p. 24]{araujo2012}}
%  \end{minipage}
%\end{figure}



Observe que, segundo a \citeonline[seções 4.2.1.10 e 5.8]{NBR14724:2011}, as
ilustrações devem sempre ter numeração contínua e única em todo o documento:

\begin{citacao}
Qualquer que seja o tipo de ilustração, sua identificação aparece na parte
superior, precedida da palavra designativa (desenho, esquema, fluxograma,
fotografia, gráfico, mapa, organograma, planta, quadro, retrato, figura,
imagem, entre outros), seguida de seu número de ordem de ocorrência no texto,
em algarismos arábicos, travessão e do respectivo título. Após a ilustração, na
parte inferior, indicar a fonte consultada (elemento obrigatório, mesmo que
seja produção do próprio autor), legenda, notas e outras informações
necessárias à sua compreensão (se houver). A ilustração deve ser citada no
texto e inserida o mais próximo possível do trecho a que se
refere. \cite[seções 5.8]{NBR14724:2011}
\end{citacao}

% ---
\section{Expressões matemáticas}
% ---

\index{expressões matemáticas}Use o ambiente \texttt{equation} para escrever
expressões matemáticas numeradas:

\begin{equation}
  \forall x \in X, \quad \exists \: y \leq \epsilon
\end{equation}

Escreva expressões matemáticas entre \$ e \$, como em $\lim_{x \to \infty}
\exp(-x) = 0 $, para que fiquem na mesma linha.

Também é possível usar colchetes para indicar o início de uma expressão
matemática que não é numerada.

\[
\left|\sum_{i=1}^n a_ib_i\right|
\le
\left(\sum_{i=1}^n a_i^2\right)^{1/2}
\left(\sum_{i=1}^n b_i^2\right)^{1/2}
\]

Consulte mais informações sobre expressões matemáticas em
\url{https://code.google.com/p/abntex2/wiki/Referencias}.


\subsection{Expressões matemáticas em títulos e subtítulos: \texorpdfstring{$\boldsymbol{\gamma_x}$}{Gammax}}






% ---
\section{Enumerações: alíneas e subalíneas}
% ---

\index{alíneas}\index{subalíneas}\index{incisos}Quando for necessário enumerar
os diversos assuntos de uma seção que não possua título, esta deve ser
subdividida em alíneas \cite[4.2]{NBR6024:2012}:

\begin{alineas}

  \item os diversos assuntos que não possuam título próprio, dentro de uma mesma
  seção, devem ser subdivididos em alíneas; 
  
  \item o texto que antecede as alíneas termina em dois pontos;
  \item as alíneas devem ser indicadas alfabeticamente, em letra minúscula,
  seguida de parêntese. Utilizam-se letras dobradas, quando esgotadas as
  letras do alfabeto;

  \item as letras indicativas das alíneas devem apresentar recuo em relação à
  margem esquerda;

  \item o texto da alínea deve começar por letra minúscula e terminar em
  ponto-e-vírgula, exceto a última alínea que termina em ponto final;

  \item o texto da alínea deve terminar em dois pontos, se houver subalínea;

  \item a segunda e as seguintes linhas do texto da alínea começa sob a
  primeira letra do texto da própria alínea;
  
  \item subalíneas \cite[4.3]{NBR6024:2012} devem ser conforme as alíneas a
  seguir:

  \begin{alineas}
     \item as subalíneas devem começar por travessão seguido de espaço;

     \item as subalíneas devem apresentar recuo em relação à alínea;

     \item o texto da subalínea deve começar por letra minúscula e terminar em
     ponto-e-vírgula. A última subalínea deve terminar em ponto final, se não
     houver alínea subsequente;

     \item a segunda e as seguintes linhas do texto da subalínea começam sob a
     primeira letra do texto da própria subalínea.
  \end{alineas}
  
  \item no \abnTeX\ estão disponíveis os ambientes \texttt{incisos} e
  \texttt{subalineas}, que em suma são o mesmo que se criar outro nível de
  \texttt{alineas}, como nos exemplos à seguir:
  
  \begin{incisos}
    \item \textit{Um novo inciso em itálico};
  \end{incisos}
  
  \item Alínea em \textbf{negrito}:
  
  \begin{subalineas}
    \item \textit{Uma subalínea em itálico};
    \item \underline{\textit{Uma subalínea em itálico e sublinhado}}; 
  \end{subalineas}
  
  \item Última alínea com \emph{ênfase}.
  
\end{alineas}

% ---
\section{Espaçamento entre parágrafos e linhas}
% ---

\index{espaçamento!dos parágrafos}O tamanho do parágrafo, espaço entre a margem
e o início da frase do parágrafo, é definido por:

\begin{verbatim}
   \setlength{\parindent}{1.3cm}
\end{verbatim}

\index{espaçamento!do primeiro parágrafo}Por padrão, não há espaçamento no
primeiro parágrafo de cada início de divisão do documento
(\autoref{sec-divisoes}). Porém, você pode definir que o primeiro parágrafo
também seja indentado, como é o caso deste documento. Para isso, apenas inclua o
pacote \textsf{indentfirst} no preâmbulo do documento:

\small\begin{verbatim}
   \usepackage{indentfirst}      % Indenta o primeiro parágrafo de cada seção.
\end{verbatim}

\index{espaçamento!entre os parágrafos}O espaçamento entre um parágrafo e outro
pode ser controlado por meio do comando:

\begin{verbatim}
  \setlength{\parskip}{0.2cm}  % tente também \onelineskip
\end{verbatim}

\index{espaçamento!entre as linhas}O controle do espaçamento entre linhas é
definido por:

\begin{verbatim}
  \OnehalfSpacing       % espaçamento um e meio (padrão); 
  \DoubleSpacing        % espaçamento duplo
  \SingleSpacing        % espaçamento simples	
\end{verbatim}

Para isso, também estão disponíveis os ambientes:

\begin{verbatim}
  \begin{SingleSpace} ...\end{SingleSpace}
  \begin{Spacing}{hfactori} ... \end{Spacing}
  \begin{OnehalfSpace} ... \end{OnehalfSpace}
  \begin{OnehalfSpace*} ... \end{OnehalfSpace*}
  \begin{DoubleSpace} ... \end{DoubleSpace}
  \begin{DoubleSpace*} ... \end{DoubleSpace*} 
\end{verbatim}

Para mais informações, consulte \citeonline[p. 47-52 e 135]{memoir}.

% ---
\section{Inclusão de outros arquivos}\label{sec-include}
% ---

É uma boa prática dividir o seu documento em diversos arquivos, e não
apenas escrever tudo em um único. Esse recurso foi utilizado neste
documento. Para incluir diferentes arquivos em um arquivo principal,
de modo que cada arquivo incluído fique em uma página diferente, utilize o
comando:

\begin{verbatim}
   \include{documento-a-ser-incluido}      % sem a extensão .tex
\end{verbatim}

Para incluir documentos sem quebra de páginas, utilize:

\begin{verbatim}
   \input{documento-a-ser-incluido}      % sem a extensão .tex
\end{verbatim}

% ---
\section{Compilar o documento \LaTeX}
% ---

Geralmente os editores \LaTeX, como o
TeXlipse\footnote{\url{http://texlipse.sourceforge.net/}}, o
Texmaker\footnote{\url{http://www.xm1math.net/texmaker/}}, entre outros,
compilam os documentos automaticamente, de modo que você não precisa se
preocupar com isso.

No entanto, você pode compilar os documentos \LaTeX usando os seguintes
comandos, que devem ser digitados no \emph{Prompt de Comandos} do Windows ou no
\emph{Terminal} do Mac ou do Linux:

\begin{verbatim}
   pdflatex ARQUIVO_PRINCIPAL.tex
   bibtex ARQUIVO_PRINCIPAL.aux
   makeindex ARQUIVO_PRINCIPAL.idx 
   makeindex ARQUIVO_PRINCIPAL.nlo -s nomencl.ist -o ARQUIVO_PRINCIPAL.nls
   pdflatex ARQUIVO_PRINCIPAL.tex
   pdflatex ARQUIVO_PRINCIPAL.tex
\end{verbatim}

% ---
\section{Remissões internas}
% ---

Ao nomear a \autoref{tab-nivinv} e a \autoref{fig_circulo}, apresentamos um
exemplo de remissão interna, que também pode ser feita quando indicamos o
\autoref{cap_exemplos}, que tem o nome \emph{\nameref{cap_exemplos}}. O número
do capítulo indicado é \ref{cap_exemplos}, que se inicia à
\autopageref{cap_exemplos}\footnote{O número da página de uma remissão pode ser
obtida também assim:
\pageref{cap_exemplos}.}.
Veja a \autoref{sec-divisoes} para outros exemplos de remissões internas entre
seções, subseções e subsubseções.

O código usado para produzir o texto desta seção é:

\small\begin{verbatim}
Ao nomear a \autoref{tab-nivinv} e a \autoref{fig_circulo}, apresentamos um
exemplo de remissão interna, que também pode ser feita quando indicamos o
\autoref{cap_exemplos}, que tem o nome \emph{\nameref{cap_exemplos}}. O número
do capítulo indicado é \ref{cap_exemplos}, que se inicia à
\autopageref{cap_exemplos}\footnote{O número da página de uma remissão pode ser
obtida também assim:
\pageref{cap_exemplos}.}.
Veja a \autoref{sec-divisoes} para outros exemplos de remissões internas entre
seções, subseções e subsubseções.
\end{verbatim}

% ---
\section{Divisões do documento: seção}\label{sec-divisoes}
% ---

Esta seção testa o uso de divisões de documentos. Esta é a
\autoref{sec-divisoes}. Veja a \autoref{sec-divisoes-subsection}.

\subsection{Divisões do documento: subseção}\label{sec-divisoes-subsection}

Isto é uma subseção. Veja a \autoref{sec-divisoes-subsubsection}, que é uma
\texttt{subsubsection} do \LaTeX, mas é impressa chamada de ``subseção'' porque
no Português não temos a palavra ``subsubseção''.

\subsubsection{Divisões do documento: subsubseção}
\label{sec-divisoes-subsubsection}

Isto é uma subsubseção.

\subsubsection{Divisões do documento: subsubseção}

Isto é outra subsubseção.

\subsection{Divisões do documento: subseção}\label{sec-exemplo-subsec}

Isto é uma subseção.

\subsubsection{Divisões do documento: subsubseção}

Isto é mais uma subsubseção da \autoref{sec-exemplo-subsec}.


\subsubsubsection{Esta é uma subseção de quinto
nível}\label{sec-exemplo-subsubsubsection}

Esta é uma seção de quinto nível. Ela é produzida com o seguinte comando:

\begin{verbatim}
\subsubsubsection{Esta é uma subseção de quinto
nível}\label{sec-exemplo-subsubsubsection}
\end{verbatim}

\subsubsubsection{Esta é outra subseção de quinto nível}\label{sec-exemplo-subsubsubsection-outro}

Esta é outra seção de quinto nível.


\paragraph{Este é um parágrafo numerado}\label{sec-exemplo-paragrafo}

Este é um exemplo de parágrafo nomeado. Ele é produzida com o comando de
parágrafo:

\small\begin{verbatim}
\paragraph{Este é um parágrafo nomeado}\label{sec-exemplo-paragrafo}
\end{verbatim}


A numeração entre parágrafos numeradaos e subsubsubseções são contínuas.

\paragraph{Esta é outro parágrafo numerado}\label{sec-exemplo-paragrafo-outro}

Este é outro parágrafo nomeado.

% ---
\section{Este é um exemplo de nome de seção longo. Ele deve estar
alinhado à esquerda e a segunda e demais linhas devem iniciar logo abaixo da
primeira palavra da primeira linha}
% ---

Isso atende à norma \citeonline[seções de 5.2.2 a 5.2.4]{NBR14724:2011} 
 e \citeonline[seções de 3.1 a 3.8]{NBR6024:2012}.

% ---
\section{Diferentes idiomas e hifenizações}
\label{sec-hifenizacao}
% ---

Para usar hifenizações de diferentes idiomas, inclua nas opções do documento o
nome dos idiomas que o seu texto contém. Por exemplo (para melhor
visualização, as opções foram quebras em diferentes linhas):

\begin{verbatim}
\documentclass[
	12pt,
	openright,
	twoside,
	a4paper,
	english,
	french,
	spanish,
	brazil
	]{abntex2}
\end{verbatim}

O idioma português-brasileiro (\texttt{brazil}) é incluído automaticamente pela
classe \textsf{abntex2}. Porém, mesmo assim a opção \texttt{brazil} deve ser
informada como a última opção da classe para que todos os pacotes reconheçam o
idioma. Vale ressaltar que a última opção de idioma é a utilizada por padrão no
documento. Desse modo, caso deseje escrever um texto em inglês que tenha
citações em português e em francês, você deveria usar o preâmbulo como abaixo:

\begin{verbatim}
\documentclass[
	12pt,
	openright,
	twoside,
	a4paper,
	french,
	brazil,
	english
	]{abntex2}
\end{verbatim}

A lista completa de idiomas suportados, bem como outras opções de hifenização,
estão disponíveis em \citeonline[p.~5-6]{babel}.

Exemplo de hifenização em inglês\footnote{Extraído de:
\url{http://en.wikibooks.org/wiki/LaTeX/Internationalization}}:

\begin{otherlanguage*}{english}
\textit{Text in English language. This environment switches all language-related
definitions, like the language specific names for figures, tables etc. to the other
language. The starred version of this environment typesets the main text
according to the rules of the other language, but keeps the language specific
string for ancillary things like figures, in the main language of the document.
The environment hyphenrules switches only the hyphenation patterns used; it can
also be used to disallow hyphenation by using the language name
`nohyphenation'.}
\end{otherlanguage*}

Exemplo de hifenização em francês\footnote{Extraído de:
\url{http://bigbrowser.blog.lemonde.fr/2013/02/17/tu-ne-tweeteras-point-le-vatican-interdit-aux-cardinaux-de-tweeter-pendant-le-conclave/}}:

\begin{otherlanguage*}{french}
\textit{Texte en français. Pas question que Twitter ne vienne faire une
concurrence déloyale à la traditionnelle fumée blanche qui marque l'élection
d'un nouveau pape. Pour éviter toute fuite précoce, le Vatican a donc pris un
peu d'avance, et a déjà interdit aux cardinaux qui prendront part au vote
d'utiliser le réseau social, selon Catholic News Service. Une mesure valable
surtout pour les neuf cardinaux – sur les 117 du conclave – pratiquants très
actifs de Twitter, qui auront interdiction pendant toute la période de se
connecter à leur compte.}
\end{otherlanguage*}

Pequeno texto em espanhol\footnote{Extraído de:
\url{http://internacional.elpais.com/internacional/2013/02/17/actualidad/1361102009_913423.html}}:

\foreignlanguage{spanish}{\textit{Decenas de miles de personas ovacionan al pontífice en su
penúltimo ángelus dominical, el primero desde que anunciase su renuncia. El Papa se
centra en la crítica al materialismo}}.

O idioma geral do texto por ser alterado como no exemplo seguinte:

\begin{verbatim}
  \selectlanguage{english}
\end{verbatim}

Isso altera automaticamente a hifenização e todos os nomes constantes de
referências do documento para o idioma inglês. Consulte o manual da classe
\cite{abntex2classe} para obter orientações adicionais sobre internacionalização de
documentos produzidos com \abnTeX.

A \autoref{sec-citacao} descreve o ambiente \texttt{citacao} que pode receber
como parâmetro um idioma a ser usado na citação.

% ---
\section{Consulte o manual da classe \textsf{abntex2}}
% ---

Consulte o manual da classe \textsf{abntex2} \cite{abntex2classe} para uma
referência completa das macros e ambientes disponíveis. 

Além disso, o manual possui informações adicionais sobre as normas ABNT
observadas pelo \abnTeX\ e considerações sobre eventuais requisitos específicos
não atendidos, como o caso da \citeonline[seção 5.2.2]{NBR14724:2011}, que
especifica o espaçamento entre os capítulos e o início do texto, regra
propositalmente não atendida pelo presente modelo.

% ---
\section{Referências bibliográficas}
% ---

A formatação das referências bibliográficas conforme as regras da ABNT são um
dos principais objetivos do \abnTeX. Consulte os manuais
\citeonline{abntex2cite} e \citeonline{abntex2cite-alf} para obter informações
sobre como utilizar as referências bibliográficas.

%-
\subsection{Acentuação de referências bibliográficas}
%-

Normalmente não há problemas em usar caracteres acentuados em arquivos
bibliográficos (\texttt{*.bib}). Porém, como as regras da ABNT fazem uso quase
abusivo da conversão para letras maiúsculas, é preciso observar o modo como se
escreve os nomes dos autores. Na ~\autoref{tabela-acentos} você encontra alguns
exemplos das conversões mais importantes. Preste atenção especial para `ç' e `í'
que devem estar envoltos em chaves. A regra geral é sempre usar a acentuação
neste modo quando houver conversão para letras maiúsculas.

\begin{table}[htbp]
\caption{Tabela de conversão de acentuação.}
\label{tabela-acentos}

\begin{center}
\begin{tabular}{ll}\hline\hline
acento & \textsf{bibtex}\\
à á ã & \verb+\`a+ \verb+\'a+ \verb+\~a+\\
í & \verb+{\'\i}+\\
ç & \verb+{\c c}+\\
\hline\hline
\end{tabular}
\end{center}
\end{table}


% ---
\section{Precisa de ajuda?}
% ---

Consulte a FAQ com perguntas frequentes e comuns no portal do \abnTeX:
\url{https://code.google.com/p/abntex2/wiki/FAQ}.

Inscreva-se no grupo de usuários \LaTeX:
\url{http://groups.google.com/group/latex-br}, tire suas dúvidas e ajude
outros usuários.

Participe também do grupo de desenvolvedores do \abnTeX:
\url{http://groups.google.com/group/abntex2} e faça sua contribuição à
ferramenta.

% ---
\section{Você pode ajudar?}
% ---

Sua contribuição é muito importante! Você pode ajudar na divulgação, no
desenvolvimento e de várias outras formas. Veja como contribuir com o \abnTeX\
em \url{https://code.google.com/p/abntex2/wiki/ComoContribuir}.

% ---
\section{Quer customizar os modelos do \abnTeX\ para sua instituição ou
universidade?}
% ---

Veja como customizar o \abnTeX\ em:
\url{https://code.google.com/p/abntex2/wiki/ComoCustomizar}.


% ---

% ----------------------------------------------------------
% PARTE
% ----------------------------------------------------------
%\part{Referenciais teóricos}
% ----------------------------------------------------------

% ---
% Capitulo de revisão de literatura
% ---

\chapter{Lorem ipsum dolor sit amet}

\begin{flushright}
	\showfont
\end{flushright}

\newpage


% ---
\section{Aliquam vestibulum fringilla lorem}
% ---

\lipsum[1]

\lipsum[2-3]

% ---

% ----------------------------------------------------------
% PARTE
% ----------------------------------------------------------
%\part{Resultados}
% ----------------------------------------------------------

% ---
% primeiro capitulo de Resultados
% ---


\cleardoublepage
\phantomsection %The \phantomsection command is needed to create a link to a place in the document that is not a figure, equation, table, section, subsection, chapter, etc.
\addcontentsline{toc}{chapter}{\texorpdfstring{\MakeTextUppercase{Capítulo 4}}{Capítulo 4}}

\chapter{Lectus lobortis condimentum}
% ---

\begin{flushright}
	\showfont
\end{flushright}

\newpage

% ---
\section{Vestibulum ante ipsum primis in faucibus orci luctus et ultrices
	posuere cubilia Curae}
% ---

\lipsum[21-22]

\showfont



% ---
% segundo capitulo de Resultados
% ---

\chapter[Nam sed tellus sit amet lectus]{Nam sed tellus sit amet lectus urna ullamcorper tristique interdum
elementum}
% ---

\showfont
\section[Some encoding tests]{\showfont}
\subsection{\showfont}
\subsubsection{\showfont}
\subsubsubsection{\showfont}


% ---
\section{Pellentesque sit amet pede ac sem eleifend consectetuer}
% ---

\lipsum[24-30]



% ----------------------------------------------------------
% Finaliza a parte no bookmark do PDF
% para que se inicie o bookmark na raiz
% e adiciona espaço de parte no Sumário
% ----------------------------------------------------------
\phantompart

% ---
% Conclusão (outro exemplo de capítulo sem numeração e presente no sumário)
% ---

% ---
% Conclusão (outro exemplo de capítulo sem numeração e presente no sumário)
% ---



\chapter*[Conclusão]{Conclusão}
\addcontentsline{toc}{chapter}{Conclusão}
% ---

\lipsum[31-33]

% ----------------------------------------------------------
% ELEMENTOS PÓS-TEXTUAIS
% ----------------------------------------------------------
\postextual
% ----------------------------------------------------------

% ----------------------------------------------------------
% Referências bibliográficas
% ----------------------------------------------------------

\bibliography{abntex2-modelo-references}


% ----------------------------------------------------------
% Glossário
% ----------------------------------------------------------
%
% Consulte o manual da classe abntex2 para orientações sobre o glossário.
%
%\glossary

% ----------------------------------------------------------
% Apêndices
% ----------------------------------------------------------

% ---
% Inicia os apêndices
% ---
\begin{apendicesenv}

	% Imprime uma página indicando o início dos apêndices
	\partapendices
	
	
% ---


\chapter[Morbi ultrices rutrum lorem]{Morbi ultrices rutrum lorem. \showfont}
% ---

\showfont
\section[Some encoding tests]{\showfont}
\subsection{\showfont}
\subsubsection{\showfont}
\subsubsubsection{\showfont}



\lipsum[30]


	


% ---
\chapter[Cras non urna sed feugiat]{Cras non urna sed feugiat cum sociis natoque penatibus et magnis dis
	parturient montes nascetur ridiculus mus}
% ---



\showfont
\section[Some encoding tests]{\showfont}
\subsection{\showfont}
\subsubsection{\showfont}
\subsubsubsection{\showfont}



\lipsum[31-35]


	

% ---
\chapter{Fusce facilisis lacinia dui}
% ---

\showfont
\section[Some encoding tests]{\showfont}
\subsection{\showfont}
\subsubsection{\showfont}
\subsubsubsection{\showfont}



\lipsum[32-33]



	
\end{apendicesenv}
% ---


% ----------------------------------------------------------
% Anexos
% ----------------------------------------------------------

% ---
% Inicia os anexos
% ---
\begin{anexosenv}
	
	% Imprime uma página indicando o início dos anexos
	\partanexos
	
	

% ----------------------------------------------------------
\chapter{Quisque libero justo}
% ----------------------------------------------------------


\section[Some encoding tests]{\showfont}
\subsection{\showfont}
\subsubsection{\showfont}
\subsubsubsection{\showfont}

\showfont

\lipsum[50-54]
	


% ----------------------------------------------------------
\chapter{Nullam elementum urna vel imperdiet sodales elit ipsum pharetra ligula
	ac pretium ante justo a nulla curabitur tristique arcu eu metus}
% ----------------------------------------------------------
\lipsum[55-57]



	


% ----------------------------------------------------------
\chapter{Nullam elementum urna vel imperdiet sodales elit ipsum pharetra ligula
	ac pretium ante justo a nulla curabitur tristique arcu eu metus}
% ----------------------------------------------------------
\lipsum[55-57]



	
\end{anexosenv}

%---------------------------------------------------------------------
% INDICE REMISSIVO
%---------------------------------------------------------------------
\phantompart
\printindex
%---------------------------------------------------------------------

\end{document}
