\documentclass[14pt]{article}
% load the necessary packages

%X= largura da lombada em mm
%N = é o número de páginas
%150 = é o valor da espessura do papel off set de 75g/m2
%capaLombada = 0,0075*Npages
%capaLargura= 39,2+capaLombada
%\usepackage[T1]{fontenc}
%\usepackage{lmodern}

\newcommand{\capaAltura}{21cm}
\newcommand{\capaLombada}{1.8cm}
\newcommand{\capaLargura}{41cm} % 39,2+capaLombada


\usepackage[paperheight=\capaAltura,paperwidth=\capaLargura,margin=0cm]{geometry}

\usepackage[dvipsnames,prologue,table]{pstricks}
\usepackage{pst-text}
\usepackage{color}
\usepackage{pst-grad}
\usepackage{graphicx}
\usepackage{lipsum}
\usepackage{datetime}

% alterando o aspecto da cor azul
\definecolor{red}{RGB}{255,0,0} % TCC
\definecolor{yellow}{RGB}{254,227,0} % Monografia
\definecolor{green}{RGB}{0,146,6} % Dissertação
\definecolor{blue}{RGB}{0,56,147} % Tese
\definecolor{gray}{RGB}{114,112,111} % PosDoutorado
%\definecolor{black}{RGB}{0,0,0} % PosDoutorado

% begin the document and suppress page numbers
\begin{document}
	\pagestyle{empty}
	
	% create the box with the front cover picture
	\newsavebox\IBox
	\sbox\IBox{\includegraphics[height=1.5cm]{brasaoUFSC.eps}}
	% set up the picture environment
%	\psset{unit=1cm}	
	
	\begin{pspicture}(\capaLargura,\capaAltura)
	
	% Informação sobre o nível do trabalho: Tese, Dissertação, 	TCC, Monografia, Relatório, 	etc.
	\DeclareFixedFont{\ArialDezesete}{T1}{ptm}{m}{n}{17pt} % TEX extended text - ptm - Adobe Times
	\DeclareFixedFont{\ArialCatorze}{T1}{ptm}{m}{n}{14pt} % TEX extended text - ptm - Adobe Times	
	\DeclareFixedFont{\ArialTreze}{T1}{ptm}{m}{n}{13pt} % TEX extended text - ptm - Adobe Times	
	\DeclareFixedFont{\ArialNove}{T1}{ptm}{m}{n}{9pt} % TEX extended text - ptm - Adobe Times	
	
		
	\DeclareFixedFont{\PTsmall}{T1}{ppl}{b}{it}{0.4in}
	\DeclareFixedFont{\PTsmallest}{T1}{ppl}{b}{it}{0.3in}
	\DeclareFixedFont{\PTtext}{T1}{ppl}{b}{it}{11pt}
	\DeclareFixedFont{\Logo}{T1}{pbk}{m}{n}{0.3in}
	
	% create the background
	\psframe[fillstyle=solid,fillcolor=black](0,19.5)(\capaLargura,21)
	\psframe[fillstyle=solid,fillcolor=blue](0,2.5)(\capaLargura,19.5) % Altere aqui a cor de fundo
	\psframe[fillstyle=solid,fillcolor=black](0,0)(\capaLargura,2.5)
%	
	
	% Linhas que delimitam as dobras na capa
	\psline[linecolor=white](4.5,0)(4.5,\capaAltura)
	\psline[linecolor=white](19.3,0)(19.3,\capaAltura)	
	\psline[linecolor=white](21.1,0)(21.1,\capaAltura)	
	\psline[linecolor=white](36.1,0)(36.1,\capaAltura)	
	
	% place the front cover picture
	\rput[lb](22,0.5){\usebox\IBox} % 41 - 19 = 22
%	\rput[bl]{90}(-1,0){Here is a marginal note.}
	
	
	% put the text on the front cover
%	\rput[lb](7.74,7){\PT \color{white}{Secrets of the Stamen}}
	
	
	
	
	\rput[lb]{0}(31,20){\ArialDezesete \color{white}{Tese de Doutorado}}
	\rput[lb]{0}(28,1.7){\ArialTreze \color{white}{Universidade Federal de Santa Catarina}}	
	\rput[lb]{0}(29.3,1){\ArialTreze \color{white}{Programa de Pós--Graduação em}}
    \rput[lb]{0}(31.7,0.4){\ArialTreze\color{white}{Engenharia Elétrica}}
	
	
%	\rput[l]{0}(16,13){\ArialCatorze \color{white}{Nome do Autor}}
%	\rput[l]{0}(25,18){\ArialTreze \color{white}{Modelo Canônico de Tese, Dissertação ou TCC do INEP com ABNTEX2}}		

		
	
	\rput[lb](19.6,20){\ArialDezesete \color{white}{2014}}	

	% put the text on the spine (note the rotation over 270 degrees)
	\rput[l]{270}(20.5,18){\ArialTreze \color{white}{Modelo Canônico de Tese, Dissertação ou TCC do INEP com ABNTEX2}}
	\rput[l]{270}(19.9,13){\ArialCatorze \color{white}{Nome do Autor}}
	
	
	
	% Create a Box containing the text for the back cover
%	\begin{minipage}[pos][height][contentpos]{width} text \end{minipage} 

%		
%		\newsavebox\BoxCapa
%		\sbox\BoxCapa{\begin{minipage}{15cm}
%			\ArialCatorze\textcolor{white}{{Modelo Canônico de Tese, Dissertação ou TCC do INEP com ABNTEX2 				
%					Nome do Autor}
%				\end{minipage}}	
%			% And position the box
%			\rput[tl](16,13){\usebox\BoxCapa}
%			
%
%	\newsavebox\ABAcapa2
%	\sbox\ABAcapa2{\begin{minipage}{3.5cm}
%		\ArialNove\textcolor{white}{\textbf{Aba da Capa:} Deve ter informações básicas sobre a proposta do trabalho, seguido
%			da identificação do orientador e co-orientador (se houver).	\\
%			Caixa de texto com 	largura de 3,5 cm, posicionada na parte superior da aba, abaixo da barra superior padrão
%		}
%		\end{minipage}}	
%	% And position the box
%	\rput[tl](15,19){\usebox\ABAcapa2}
%	
	
			
	
	\newsavebox\ABAcapa
	\sbox\ABAcapa{\begin{minipage}{3.5cm}
		\ArialNove\textcolor{white}{\textbf{Aba da Capa:} Deve ter informações básicas sobre a proposta do trabalho, seguido
			da identificação do orientador e co-orientador (se houver).	\\
			Caixa de texto com 	largura de 3,5 cm, posicionada na parte superior da aba, abaixo da barra superior padrão
		}
		\end{minipage}}	
	% And position the box
	\rput[tl](36.5,19){\usebox\ABAcapa}
	
	\newsavebox\ABAcontracapa
	\sbox\ABAcontracapa{\begin{minipage}{3.5cm}
		\ArialNove\textcolor{white}{\textbf{Aba da Contra--Capa:} Aba da contra-capa: deve ter a identificação da instituição (UFSC), do programa de pósgraduação, do departamento, curso, etc, cidade e estado onde o trabalho foi
			desenvolvido.	\\
			Caixa de texto com 	largura de 3,5 cm, posicionada na parte superior da aba, abaixo da barra superior padrão
		}
		\end{minipage}}	
	% And position the box
	\rput[tl](0.5,19){\usebox\ABAcontracapa}
		
	
	% Then we close all open environments
	\end{pspicture}
\end{document}