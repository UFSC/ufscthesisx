
% Changing babel package inside a single chapter
% https://tex.stackexchange.com/questions/20987/changing-babel-package-inside-a-single-chapter
%
% Multiple-language document - babel - selectlanguage vs begin/end{otherlanguage}
% https://tex.stackexchange.com/questions/36526/multiple-language-document-babel-selectlanguage-vs-begin-endotherlanguage
\begin{otherlanguage*}{brazil}
\begin{resumo}

    Segundo a \citeonline[3.1-3.2]{NBR6028:2003}, o resumo deve ressaltar o
    objetivo, o método, os resultados e as conclusões do documento. A ordem e a extensão
    destes itens dependem do tipo de resumo (informativo ou indicativo) e do
    tratamento que cada item recebe no documento original. O resumo deve ser
    precedido da referência do documento, com exceção do resumo inserido no
    próprio documento. (\ldots) As palavras-chave devem figurar logo abaixo do
    resumo, antecedidas da expressão Palavras-chave:, separadas entre si por
    ponto e finalizadas também por ponto. \showfont

\vspace{\onelineskip}
\noindent \textbf{Palavras-chaves}: latex. abntex. editoração de texto.

\end{resumo}
\end{otherlanguage*}


% resumo em inglês
\begin{otherlanguage*}{english}
\begin{resumo}[Abstract]

    This is the english abstract.

    \vspace{\onelineskip}
    \noindent\textbf{Key-words}: latex. abntex. text editoration.

\end{resumo}
\end{otherlanguage*}


%
%
%% resumo em francês
%\begin{resumo}[Résumé]
%   \begin{otherlanguage*}{french}
%       Il s'agit d'un résumé en français.
%
%       \textbf{Mots-clés}: latex. abntex. publication de textes.
%   \end{otherlanguage*}
%\end{resumo}
%
%
%% resumo em espanhol
%\begin{resumo}[Resumen]
%   \begin{otherlanguage*}{spanish}
%       Este es el resumen en español.
%
%       \textbf{Palabras clave}: latex. abntex. publicación de textos.
%   \end{otherlanguage*}
%\end{resumo}
% ---


