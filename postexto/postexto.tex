


% ELEMENTOS PÓS-TEXTUAIS
\postextual
\setlength\beforechapskip{0pt}
\setlength\midchapskip{15pt}
\setlength\afterchapskip{15pt}

% Referências bibliográficas
\bibliography{modelo-ufsc-references}

% Glossário, consulte o manual da classe abntex2 para orientações sobre o glossário.
% \glossary

% Apêndices, inicia os apêndices
\begin{apendicesenv}

    % Imprime uma página indicando o início dos apêndices
    \partapendices

    \setlength\beforechapskip{50pt}
    \setlength\midchapskip{20pt}
    \setlength\afterchapskip{20pt}

    
% ---


\chapter[Morbi ultrices rutrum lorem]{Morbi ultrices rutrum lorem. \showfont}
% ---

\showfont
\section[Some encoding tests]{\showfont}
\subsection{\showfont}
\subsubsection{\showfont}
\subsubsubsection{\showfont}



Lipsum me [30]


    


% ---
\chapter[Cras non urna sed feugiat]{Cras non urna sed feugiat cum sociis natoque penatibus et magnis dis
	parturient montes nascetur ridiculus mus}
% ---



\showfont
\section[Some encoding tests]{\showfont}
\subsection{\showfont}
\subsubsection{\showfont}
\subsubsubsection{\showfont}



Lipsum me [31-35]


    

% ---
\chapter{Fusce facilisis lacinia dui}
% ---

\showfont
\section[Some encoding tests]{\showfont}
\subsection{\showfont}
\subsubsection{\showfont}
\subsubsubsection{\showfont}



Lipsum me [32-33]




\end{apendicesenv}

% Anexos, inicia os anexos
\begin{anexosenv}

    % Imprime uma página indicando o início dos anexos
    \partanexos

    \setlength\beforechapskip{50pt}
    \setlength\midchapskip{20pt}
    \setlength\afterchapskip{20pt}

    

%
% How to fix the Underfull \vbox badness has occurred while \output is active on my memoir chapter style?
% https://tex.stackexchange.com/questions/387881/how-to-fix-the-underfull-vbox-badness-has-occurred-while-output-is-active-on-m
%

% ----------------------------------------------------------
\chapter{Artigo publicado na revista SOBRAEP}
% ----------------------------------------------------------


\section[English guidelines for publication]{ english guidelines for publication - TITLE HERE (14 PT TYPE SIZE, UPPERCASE, BOLD, CENTERED)}

\begin{otherlanguage*}{english}
    \noindent\textbf{Abstract:}
    The objective of this document is to instruct the authors about the preparation of the manuscript for its submission to the Revista Eletrônica de Potência (Brazilian Power Electronics Journal).~The authors should use these guidelines for preparing both the initial and final versions of their paper. Additional information about procedures and guidelines for publication can be obtained directly with the editor, or through the web site \url{http://www.sobraep.org.br/revista}. This text was written according to these guidelines
\end{otherlanguage*}


\modifiedincludepdf{-}{ArtigoSOBRAEP}{pictures/SOBRAEP.pdf}{0.9}

    


%
% How to fix the Underfull \vbox badness has occurred while \output is active on my memoir chapter style?
% https://tex.stackexchange.com/questions/387881/how-to-fix-the-underfull-vbox-badness-has-occurred-while-output-is-active-on-m
%

% ----------------------------------------------------------
\chapter[Nullam elementum urna vel imperdiet sodales elit ipsum]{Nullam
    elementum urna vel imperdiet sodales elit ipsum pharetra ligula
    ac pretium ante justo a nulla curabitur tristique arcu eu metus}
% ----------------------------------------------------------

\showfont

1. How to display the font size in use in the final output,
2. How to display the font size in use in the final output,
3. How to display the font size in use in the final output,


\section[Some encoding tests]{\showfont}

1. How to display the font size in use in the final output,
2. How to display the font size in use in the final output,
3. How to display the font size in use in the final output,
4. How to display the font size in use in the final output,
5. How to display the font size in use in the final output,
6. How to display the font size in use in the final output,

7. How to display the font size in use in the final output,
8. How to display the font size in use in the final output,
9. How to display the font size in use in the final output,
10. How to display the font size in use in the final output,
11. How to display the font size in use in the final output,
12. How to display the font size in use in the final output,

\subsection{\showfont}

1. How to display the font size in use in the final output,
2. How to display the font size in use in the final output,
3. How to display the font size in use in the final output,
4. How to display the font size in use in the final output,
5. How to display the font size in use in the final output,
6. How to display the font size in use in the final output,

7. How to display the font size in use in the final output,
8. How to display the font size in use in the final output,
9. How to display the font size in use in the final output,
10. How to display the font size in use in the final output,
11. How to display the font size in use in the final output,
12. How to display the font size in use in the final output,

\subsubsection{\showfont}

1. How to display the font size in use in the final output,
2. How to display the font size in use in the final output,
3. How to display the font size in use in the final output,
4. How to display the font size in use in the final output,
5. How to display the font size in use in the final output,
6. How to display the font size in use in the final output,

7. How to display the font size in use in the final output,
8. How to display the font size in use in the final output,
9. How to display the font size in use in the final output,
10. How to display the font size in use in the final output,
11. How to display the font size in use in the final output,
12. How to display the font size in use in the final output,

\subsubsubsection{\showfont}

1. How to display the font size in use in the final output,
2. How to display the font size in use in the final output,
3. How to display the font size in use in the final output,
4. How to display the font size in use in the final output,
5. How to display the font size in use in the final output,
6. How to display the font size in use in the final output,
7. How to display the font size in use in the final output,

8. How to display the font size in use in the final output,
9. How to display the font size in use in the final output,
10. How to display the font size in use in the final output,
11. How to display the font size in use in the final output,
12. How to display the font size in use in the final output,


Lipsum me [55-65]



    

% ----------------------------------------------------------
\chapter[Nullam elementum urna vel imperdiet sodales elit ipsum]{Nullam elementum urna vel imperdiet sodales elit ipsum pharetra ligula
	ac pretium ante justo a nulla curabitur tristique arcu eu metus}
% ----------------------------------------------------------

\showfont
\section[Some encoding tests]{\showfont}
\subsection{\showfont}
\subsubsection{\showfont}
\subsubsubsection{\showfont}



Lipsum me [55-57]




\end{anexosenv}

% INDICE REMISSIVO
\phantompart
\printindex




