%----------------------------------------------------------------------------------------
%   Thesis Tweaks and Utilities
%----------------------------------------------------------------------------------------

% Uncomment this if you are debugging pages' badness Underfull & Overflow
% https://tex.stackexchange.com/questions/115908/geometry-showframe-landscape
% https://tex.stackexchange.com/questions/387077/what-is-the-difference-between-usepackageshowframe-and-usepackageshowframe
% https://tex.stackexchange.com/questions/387257/how-to-do-the-memoir-headings-fix-but-not-have-my-text-going-over-the-page-botto
% https://tex.stackexchange.com/questions/14508/print-page-margins-of-a-document
% \usepackage[showframe,pass]{geometry}

% To use the font Times New Roman, instead of the default LaTeX font
% more up-to-date than '\usepackage{mathptmx}'
% \usepackage{newtxtext}
% \usepackage{newtxmath}

% https://tex.stackexchange.com/questions/182569/how-to-manually-set-where-a-word-is-split
\hyphenation{Ge-la-im}
\hyphenation{Cis-la-ghi}

% Add missing translations for Portuguese
% https://tex.stackexchange.com/questions/8564/what-is-the-right-way-to-redefine-macros-defined-by-babel
\makeatletter
\@ifpackageloaded{babel}{\@ifpackagewith{babel}{brazil}{\addto\captionsbrazil{%
    \renewcommand{\mytextpreliminarylistname}{Breve Sumário}
}}{}}{}
\@ifundefined{advisor}{\newcommand{\advisor}[2]{#1}}{}
\makeatother

% Selects a sans serif font family
% \renewcommand{\sfdefault}{cmss}

% Selects a monospaced (“typewriter”) font family
% \renewcommand{\ttdefault}{cmtt}

% Spacing between lines and paragraphs
% https://tex.stackexchange.com/questions/70212/ifpackageloaded-question
\makeatletter
\@ifclassloaded{memoir}
{
    % New custom chapter style VZ14, see other chapters styles in:
    % http://repositorios.cpai.unb.br/ctan/info/latex-samples/MemoirChapStyles/MemoirChapStyles.pdf
    \newcommand\thickhrulefill{\leavevmode \leaders \hrule height 1ex \hfill \kern \z@}
    \makechapterstyle{VZ14} { %
        % \thispagestyle{empty}
        \setlength\beforechapskip{50pt}
        \setlength\midchapskip{20pt}
        \setlength\afterchapskip{20pt}
        \renewcommand\chapternamenum{}
        \renewcommand\printchaptername{}
        \renewcommand\chapnamefont{\Huge\scshape}
        \renewcommand\printchapternum {%
            \chapnamefont\null\thickhrulefill\quad
            \@chapapp\space\thechapter\quad\thickhrulefill
        }
        \renewcommand\printchapternonum {%
            \par\thickhrulefill\par\vskip\midchapskip
            \hrule\vskip\midchapskip
        }
        \renewcommand\chaptitlefont{\huge\scshape\centering}
        \renewcommand\afterchapternum {%
            \par\nobreak\vskip\midchapskip\hrule\vskip\midchapskip
        }
        \renewcommand\afterchaptertitle {%
            \par\vskip\midchapskip\hrule\nobreak\vskip\afterchapskip
        }
    }

    % Apply the style `VZ14` just created
    % \chapterstyle{VZ14}

    % http://mirrors.ibiblio.org/CTAN/macros/latex/contrib/memoir/memman.pdf
    \setlength\beforechapskip{0pt}
    \setlength\midchapskip{15pt}
    \setlength\afterchapskip{15pt}

    % Memoir: Warnings “The material used in the headers is too large” w/ accented titles
    % https://tex.stackexchange.com/questions/387293/how-to-change-the-page-layout-with-memoir
    \setheadfoot{30.0pt}{\footskip}
    \checkandfixthelayout
}{}
\makeatother

% Controlling the spacing between one paragraph and another, try also \onelineskip
% Default value for UFSC 0.0cm
\setlength{\parskip}{\advisor{0.0cm}{0.2cm}}

% Paragraph size is given by
% Default value for UFSC 1.0cm
\setlength{\parindent}{1.3cm}

% https://tex.stackexchange.com/questions/148647/how-to-remove-space-before-enumerate
% https://tex.stackexchange.com/questions/433543/behaviour-of-enumitem-setlist
\advisor{}{
\setlist*[enumerate]{nosep}
\setlist*[enumerateoptional]{nosep}
}


% Color settings across the document
\makeatletter
\@ifpackageloaded{xcolor}
{
    % RGB colors in absolute values from 0 to 255 by using `RGB` tag
    \definecolor{darkblue}{RGB}{26,13,178}

    % Colors names definitions as RGB colors in percentage notation by using `rgb` tag
    \definecolor{mygreen}{rgb}{0,0.6,0}
    \definecolor{mygray}{rgb}{0.5,0.5,0.5}
    \definecolor{mymauve}{rgb}{0.58,0,0.82}
    \definecolor{figcolor}{rgb}{1,0.4,0}
    \definecolor{tabcolor}{rgb}{1,0.4,0}
    \definecolor{eqncolor}{rgb}{1,0.4,0}
    \definecolor{linkcolor}{rgb}{1,0.4,0}
    \definecolor{citecolor}{rgb}{1,0.4,0}
    \definecolor{seccolor}{rgb}{0,0,1}
    \definecolor{abscolor}{rgb}{0,0,1}
    \definecolor{titlecolor}{rgb}{0,0,1}
    \definecolor{biocolor}{rgb}{0,0,1}
    \definecolor{blue}{RGB}{41,5,195}

    % PDF Hyperlinks settings
    \@ifpackageloaded{hyperref}
    {
        \hypersetup
        {
            colorlinks=true,     % false: boxed links; true: colored links
            linkcolor=darkblue,  % color of internal links
            citecolor=darkblue, % color of links to bibliography
            filecolor=black,     % color of file links
            urlcolor=\advisor{black}{darkgreen},
            bookmarksdepth=4,
            pdfencoding=auto,%
            psdextra,
        }
    }
}{}
\makeatother

% Backref package settings, pages with citations in bibliography
\newcommand{\biblatexcitedntimes}{\autocap{c}ited \arabic{citecounter} times}
\newcommand{\biblatexcitedonetime}{\autocap{c}ited one time}
\newcommand{\biblatexcitednotimes}{\autocap{n}o citation in the text}

\makeatletter
\@ifpackageloaded{babel}{\@ifpackagewith{babel}{brazil}{\addto\captionsbrazil{%
    \renewcommand{\biblatexcitedntimes}{\autocap{c}itado \arabic{citecounter} vezes}
    \renewcommand{\biblatexcitedonetime}{\autocap{c}itado uma vez}
    \renewcommand{\biblatexcitednotimes}{\autocap{n}enhuma citação no texto}
}}{}}{}
\@ifpackageloaded{biblatex}
{%
    % https://tex.stackexchange.com/questions/483707/how-to-detect-whether-the-option-citecounter-was-enabled-on-biblatex
    \ifx\blx@citecounter\relax
        \message{Is citecounter defined? NO!^^J}
    \else
        \message{Is citecounter defined? YES!^^J}
        \ifbacktracker
            \message{Is backtracker defined? YES!^^J}
            \renewbibmacro*{pageref}
            {
                \iflistundef{pageref}
                {\printtext{\biblatexcitednotimes}}
                {%
                    \printtext
                    {%
                        \ifnumgreater{\value{citecounter}}{1}
                            {\biblatexcitedntimes}
                            {\biblatexcitedonetime}
                    }%
                    \setunit{\addspace}%
                    \ifnumgreater{\value{pageref}}{1}
                        {\bibstring{backrefpages}\ppspace}
                        {\bibstring{backrefpage}\ppspace}%
                    \printlist[pageref][-\value{listtotal}]{pageref}%
                }%
            }

            \DefineBibliographyStrings{brazil}
            {
                backrefpage  = {na página},
                backrefpages = {nas páginas},
            }

            \DefineBibliographyStrings{english}
            {
                backrefpage  = {on page},
                backrefpages = {on pages},
            }
        \else
            \message{Is backtracker defined? NO!^^J}
        \fi
    \fi
}{}
\makeatother

% https://tex.stackexchange.com/questions/391695/is-possible-to-remove-the-link-color-of-the-comma-on-the-citation-link
% \DeclareFieldFormat{citehyperref}{#1}

% https://tex.stackexchange.com/questions/19105/how-can-i-put-more-space-between-bibliography-entries-biblatex
\advisor{}{\setlength\bibitemsep{2.1\itemsep}}

% % https://tex.stackexchange.com/questions/203764/reduce-font-size-of-bibliography-overfull-bibliography
% \newcommand{\bibliographyfontsize}{\fontsize{10.0pt}{10.5pt}\selectfont}
% \renewcommand*{\bibfont}{\bibliographyfontsize}

% Uncomment this to insert the abstract into your bibliography entries when the abstract is available
% https://tex.stackexchange.com/questions/398666/how-to-correctly-insert-and-justify-abstract
\ifadvisor
\else
    \DeclareFieldFormat{abstract}%
    {%
        \vspace*{-0.5mm}\par\justifying
        \begin{adjustwidth}{1cm}{}
            \textbf{\bibsentence\bibstring{abstract}:} #1
        \end{adjustwidth}
    }
    \renewbibmacro*{finentry}%
    {%
        \iffieldundef{abstract}
        {\finentry}
        {\finentrypunct
            \printfield{abstract}%
            \renewcommand*{\finentrypunct}{}%
            \finentry
        }
    }
\fi


% https://tex.stackexchange.com/questions/14314/changing-the-font-of-the-numbers-in-the-toc-in-the-memoir-class
\renewcommand{\cftpartfont}{\ABNTEXpartfont\color{black}}
\renewcommand{\cftpartpagefont}{\ABNTEXpartfont\color{black}}

\renewcommand{\cftchapterfont}{\ABNTEXchapterfont\color{black}}
\renewcommand{\cftchapterpagefont}{\ABNTEXchapterfont\color{black}}

\renewcommand{\cftsectionfont}{\ABNTEXsectionfont\color{black}}
\renewcommand{\cftsectionpagefont}{\ABNTEXsectionfont\color{black}}

\renewcommand{\cftsubsectionfont}{\ABNTEXsubsectionfont\color{black}}
\renewcommand{\cftsubsectionpagefont}{\ABNTEXsubsectionfont\color{black}}

\renewcommand{\cftsubsubsectionfont}{\ABNTEXsubsubsectionfont\color{black}}
\renewcommand{\cftsubsubsectionpagefont}{\ABNTEXsubsubsectionfont\color{black}}

\renewcommand{\cftparagraphfont}{\ABNTEXsubsubsubsectionfont\color{black}}
\renewcommand{\cftparagraphpagefont}{\ABNTEXsubsubsubsectionfont\color{black}}

% Memoir has another mechanism for the job: \cftsetindents{‹kind›}{indent}{numwidth}. Here kind is
% chapter, section, or whatever; the indent specifies the ‘margin’ before the entry starts; and the
% width is of the box into which the number is typeset (so needs to be wide enough for the largest
% number, with the necessary spacing to separate it from what comes after it in the line.
% http://www.tex.ac.uk/FAQ-tocloftwrong.html
% https://tex.stackexchange.com/questions/264668/memoir-indentation-of-unnumbered-sections-in-table-of-contents
% https://tex.stackexchange.com/questions/394227/memoir-toc-indent-the-second-line-by-numberspace
%
% `\cftlastnumwidth` and these `\cftsetindents` are defined by the abntex2 class,
% obeying the `ABNTEXsumario-abnt-6027-2012`. \newlength{\cftlastnumwidth}
% \setlength{\cftlastnumwidth}{\cftsubsubsectionnumwidth}
% \addtolength{\cftlastnumwidth}{-1em}

% http://www.tex.ac.uk/FAQ-tocloftwrong.html
% Use \setlength\cftsectionnumwidth{4em} to override all these values at once
\makeatletter
\ifadvisor
\else
    \makechapterstyle{fixedabntex2indentation}
    {%
        \renewcommand{\chapterheadstart}{}

        \ifx \chapternamenumlength \undefined
            \newlength{\chapternamenumlength}
        \fi

        % tamanhos de fontes de chapter e part
        \ifthenelse{\equal{\ABNTEXisarticle}{true}}{%
            \setlength\beforechapskip{\baselineskip}%
            \renewcommand{\chaptitlefont}{\ABNTEXsectionfont\ABNTEXsectionfontsize}%
        }{%else
           \setlength{\beforechapskip}{0pt}%
           \renewcommand{\chaptitlefont}{\ABNTEXchapterfont\ABNTEXchapterfontsize}%
        }

        \renewcommand{\chapnumfont}{\chaptitlefont}
        \renewcommand{\parttitlefont}{\ABNTEXpartfont\ABNTEXpartfontsize}
        \renewcommand{\partnumfont}{\ABNTEXpartfont\ABNTEXpartfontsize}
        \renewcommand{\partnamefont}{\ABNTEXpartfont\ABNTEXpartfontsize}

        % tamanhos de fontes de section, subsection, subsubsection e subsubsubsection
        \setsecheadstyle{\ABNTEXsectionfont\ABNTEXsectionfontsize\ABNTEXsectionupperifneeded}
        \setsubsecheadstyle{\ABNTEXsubsectionfont\ABNTEXsubsectionfontsize\ABNTEXsubsectionupperifneeded}
        \setsubsubsecheadstyle{\ABNTEXsubsubsectionfont\ABNTEXsubsubsectionfontsize\ABNTEXsubsubsectionupperifneeded}
        \setsubsubsubsecheadstyle{\ABNTEXsubsubsubsectionfont\ABNTEXsubsubsubsectionfontsize\ABNTEXsubsubsubsectionupperifneeded}

        % Impressão do número do capítulo
        \renewcommand{\chapternamenum}{}

        % Impressão do nome do capítulo
        \renewcommand{\printchaptername}{%
           \chaptitlefont%
           \ifthenelse{\boolean{abntex@apendiceousecao}}{\appendixname}{}%
        }

        % Impressão do título do capítulo
        \def\printchaptertitle##1{%
            \chaptitlefont%
            \ifthenelse{\boolean{abntex@innonumchapter}}{\centering\ABNTEXchapterupperifneeded{##1}}{%
            \ifthenelse{\boolean{abntex@apendiceousecao}}{%
                \centering%
                \settowidth{\chapternamenumlength}{\printchaptername\printchapternum\afterchapternum}%
                \ABNTEXchapterupperifneeded{##1}%
            }{%
                \settowidth{\chapternamenumlength}{\printchaptername\printchapternum\afterchapternum}%
                \parbox[t]{\columnwidth-\chapternamenumlength}{\ABNTEXchapterupperifneeded{##1}}}%
            }%
        }

        % https://tex.stackexchange.com/questions/264668/memoir-indentation-of-unnumbered-sections-in-table-of-contents
        \renewcommand{\tocinnonumchapter}{%
          \addtocontents{toc}{\cftsetindents{chapter}{2.5em}{2em}}%
          \cftinserthook{toc}{A}}

        % Impressão do número do capítulo (no capítulo e não toc)
        \renewcommand{\printchapternum}{%
            \setboolean{abntex@innonumchapter}{false}%
            \chapnumfont%
            ~~\thechapter~%
            \ifthenelse{\boolean{abntex@apendiceousecao}}{%
                \tocinnonumchapter%
                ~\ABNTEXcaptiondelim~~%
            }{}%
        }

        \renewcommand{\ABNTEXcaptiondelim}{\textendash}
        \renewcommand{\afterchapternum}{}

        % Impressão do capítulo não numerado
        \renewcommand\printchapternonum{%
            \setboolean{abntex@innonumchapter}{true}%
        }
    }
    \chapterstyle{fixedabntex2indentation}

    \cftsetindents{part}          {0em} {3em}
    \cftsetindents{chapter}       {0em} {3em}
    \cftsetindents{section}       {0em} {4.3em}
    \cftsetindents{subsection}    {0em} {5.2em}
    \cftsetindents{subsubsection} {0em} {5.1em}
    \cftsetindents{paragraph}     {0em} {6.0em}
    \cftsetindents{subparagraph}  {0em} {7.0em}
\fi
\makeatother
