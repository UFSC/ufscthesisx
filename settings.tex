

% Thesis settings
\newcommand{\brazil}[1]{\foreignlanguage{brazil}{#1}}
\newcommand{\english}[1]{\foreignlanguage{english}{#1}}

% What is the difference between \def and \newcommand?
% https://tex.stackexchange.com/questions/655/what-is-the-difference-between-def-and-newcommand
\def\mytextpreliminarylistname{\chooselang{Brief Table of Contents}{Breve Sumário}}

% How to manually set where a word is split?
% https://tex.stackexchange.com/questions/182569/how-to-manually-set-where-a-word-is-split
\hyphenation{Ge-la-im}

% Informações de dados para CAPA e FOLHA DE ROSTO
\titulo
{%
    \chooselang
    {Canonical Model of Monograph, Dissertation, Thesis or Report}
    {Modelo Canônico de Monografia,  Dissertação, Tese ou Relatório}
}
\subtitulo
{%
    \chooselang
    {Post\hyp{}Doctorate from UFSC with \abnTeX{}}
    {Pós--Doutorado da UFSC com \abnTeX{}}
}

\data{\today}
\autor{\chooselang{Author's name}{Nome do Autor}}
\local{\chooselang{\brazil{Florianópolis, Santa Catarina} -- Brazil}{Florianópolis, Santa Catarina -- Brasil}}

\biblioteca{\chooselang{University Library}{Biblioteca Universitária}}
\orientador{\chooselang{Prof. PhD. Supervisor Name}{Prof. Dr. Nome do Coorientador}}
\coorientador{\chooselang{Prof. PhD. Co\hyp{}supervisor Name}{Prof. Dr. Nome do Coorientador}}

\instituicaosigla{UFSC}
\instituicao{\chooselang{Federal University of \brazil{Santa Catarina}}{Universidade Federal de Santa Catarina}}
\tipotrabalho{\chooselang{Doctoral Thesis}{Tese de Doutorado}}

\area{\chooselang{concentration in Power Electronics and Electrical Drive}{concentração em Eletrônica de Potência e Acionamento Elétrico}}
\formacao{\chooselang{Doctor's Degree in Electrical Engineering}{Grau de Doutor em Engenharia Elétrica}}
\programa{\chooselang{Postgraduate Program in Electrical Engineering}{Programa de Pós\hyp{}Graduação em Engenharia Elétrica}}
\centro{\chooselang{Department of Electrical and Electronic Engineering}{Departamento de Engenharia Elétrica e Eletrônica}}

% O preambulo deve conter tipo do trabalho, objetivo, nome da instituição e a área de concentração.
\preambulo
{%
    \chooselang
    {Thesis submitted to the~\imprimirprograma~of the~\imprimirinstituicao~to obtain the \imprimirformacao. \showfont}
    {Tese submetida ao \imprimirprograma da \imprimirinstituicao para a obtenção do Título de \imprimirformacao. \showfont}
}

% Keywords
\newcommand{\palavraschaveingles}
{%
    \item latex. \item abntex. \item text publishing.
}
\newcommand{\palavraschaveportugues}
{%
    \item latex. \item abntex. \item editoração de texto.
}

% Remove the colon appended to theses variables, allowing us to use other separators
\addto\captionsbrazil
{
    \renewcommand{\orientadorname}{Orientador}
    \renewcommand{\coorientadorname}{Coorientador}
}

% Create caption English translations as the sections headers
% https://tex.stackexchange.com/questions/8564/what-is-the-right-way-to-redefine-macros-defined-by-babel
\addto\captionsenglish
{
    %% adjusts names from abnTeX2
    \renewcommand{\folhaderostoname}{Title page}
    \renewcommand{\epigraphname}{Epigraph}
    \renewcommand{\dedicatorianame}{Dedication}
    \renewcommand{\errataname}{Errata sheet}
    \renewcommand{\agradecimentosname}{Acknowledgements}
    \renewcommand{\anexoname}{ANNEX}
    \renewcommand{\anexosname}{Annex}
    \renewcommand{\apendicename}{APPENDIX}
    \renewcommand{\apendicesname}{Appendix}
    \renewcommand{\orientadorname}{Supervisor}
    \renewcommand{\coorientadorname}{Co\hyp{}supervisor}
    \renewcommand{\folhadeaprovacaoname}{Approval}
    \renewcommand{\resumoname}{Abstract}
    \renewcommand{\listadesiglasname}{List of abbreviations and acronyms}
    \renewcommand{\listadesimbolosname}{List of symbols}
    \renewcommand{\fontename}{Source}
    \renewcommand{\notaname}{Note}
    %% adjusts names used by \autoref
    \renewcommand{\pageautorefname}{page}
    \renewcommand{\sectionautorefname}{section}
    \renewcommand{\subsectionautorefname}{subsection}
    \renewcommand{\subsubsectionautorefname}{subsubsection}
    \renewcommand{\paragraphautorefname}{subsubsubsection}
}

% References translation
\ifenglish
    % These default values are already in English
\else
    % Configurações de códigos fonte no documento
    \makeatletter
    \@ifpackageloaded{listings}
    {
        % Listing -> Codigo fonte
        \renewcommand{\lstlistingname}{Código--fonte}

        % List of Listings -> Lista de códigos-fonte
        \renewcommand{\lstlistlistingname}{Lista de códigos--fonte}

        % Calculate the size of the header
        \calculatelisteningsheader
    }
    \makeatother

    % Configurações do pacote backref, paginas com as citações na bibliografia
    \makeatletter
    \@ifpackageloaded{backref}
    {
        % Usado sem a opção hyperpageref de backref
        \renewcommand{\backrefpagesname}{Citado na(s) página(s):~}

        % Texto padrão antes do número das páginas
        \renewcommand{\backref}{}

        % Define os textos da citação
        \renewcommand*{\backrefalt}[4]
        {
            \ifcase #1
                Nenhuma citação no texto.
            \or
                Citado na página #2.
            \else
                Citado #1 vezes nas páginas #2.
            \fi
        }
    }{}
    \makeatother
\fi

% Espaçamentos entre linhas e parágrafos
%
% ifpackageloaded question
% https://tex.stackexchange.com/questions/70212/ifpackageloaded-question
\makeatletter
\@ifclassloaded{memoir}
{
    % Estilo de capítulos, ver classe para maiores detalhes.Veja outros estilos em:
    % http://mirrors.ibiblio.org/CTAN/macros/latex/contrib/memoir/memman.pdf
    \chapterstyle{VZ14}
    \setlength\beforechapskip{0pt}
    \setlength\midchapskip{15pt}
    \setlength\afterchapskip{15pt}

    % O tamanho do parágrafo é dado por:
    \setlength{\parindent}{1.3cm}

    % Controle do espaçamento entre um parágrafo e outro. Tente também
    % \onelineskip
    \setlength{\parskip}{0.2cm}

    % Memoir: Warnings “The material used in the headers is too large” w/ accented titles
    % https://tex.stackexchange.com/questions/387293/how-to-change-the-page-layout-with-memoir
    \setheadfoot{30.0pt}{\footskip}
    \checkandfixthelayout
}{}
\makeatother

% Color settings across the document
\makeatletter
\@ifpackageloaded{xcolor}
{
    % RGB colors in absolute values from 0 to 255 by using `RGB` tag
    \definecolor{darkblue}{RGB}{26,13,178}

    % Definição de cores, RGB colors in percentage notation by using `rgb` tag
    \definecolor{mygreen}{rgb}{0,0.6,0}
    \definecolor{mygray}{rgb}{0.5,0.5,0.5}
    \definecolor{mymauve}{rgb}{0.58,0,0.82}

    % Configurações de aparência do PDF final
    \definecolor{figcolor}{rgb}{1,0.4,0}  % orange
    \definecolor{tabcolor}{rgb}{1,0.4,0}  % orange
    \definecolor{eqncolor}{rgb}{1,0.4,0}  % orange
    \definecolor{linkcolor}{rgb}{1,0.4,0} % orange
    \definecolor{citecolor}{rgb}{1,0.4,0} % orange
    \definecolor{seccolor}{rgb}{0,0,1}    % blue
    \definecolor{abscolor}{rgb}{0,0,1}    % blue
    \definecolor{titlecolor}{rgb}{0,0,1}  % blue
    \definecolor{biocolor}{rgb}{0,0,1}    % blue

    % Alterando o aspecto da cor azul
    \definecolor{blue}{RGB}{41,5,195}

    % Informações do PDF
    \@ifpackageloaded{hyperref}
    {
        \hypersetup
        {
            pdftitle={\@title},
            colorlinks=true, % false: boxed links; true: colored links
            linkcolor=darkblue, % color of internal links
            citecolor=darkgreen, % color of links to bibliography
            filecolor=black, % color of file links
            urlcolor=linkcolor,
            bookmarksdepth=4
        }
        \ifenglish
            \hypersetup
            {
                pdfauthor={Author},
                pdfsubject={Thesis' Abstract},
                pdfcreator={LaTeX with abnTeX2 for UFSC},
                pdfkeywords={abnt}{latex}{UFSC}{abntex2}{thesis},
            }
        \else
            \hypersetup
            {
                pdfauthor={Autores},
                pdfsubject={Resumo da tese},
                pdfcreator={LaTeX com abnTeX2 para UFSC},
                pdfkeywords={abnt}{latex}{UFSC}{abntex2}{tese},
            }
        \fi
    }
}{}
\makeatother


% Fontes das entradas do sumario
\makeatletter
\renewcommand*{\l@chapter}[2]
{%
    \l@chapapp{\uppercase{#1}}{#2}{\cftchaptername}
}
\renewcommand*{\l@section}[2]
{%
    \l@chapapp{\ABNTEXsectionfont\uppercase{#1}}{#2}{\cftsectionname}
}
\makeatother

% Changing the font of the numbers in the ToC in the memoir class
% https://tex.stackexchange.com/questions/14314/changing-the-font-of-the-numbers-in-the-toc-in-the-memoir-class
\renewcommand{\cftpartfont}{\ABNTEXpartfont\color{darkblue}}
\renewcommand{\cftpartpagefont}{\ABNTEXpartfont\color{black}}

\renewcommand{\cftchapterfont}{\ABNTEXchapterfont\color{darkblue}}
\renewcommand{\cftchapterpagefont}{\ABNTEXchapterfont\color{black}}

\renewcommand{\cftsectionfont}{\ABNTEXsectionfont\color{darkblue}}
\renewcommand{\cftsectionpagefont}{\ABNTEXsectionfont\color{black}}

\renewcommand{\cftsubsectionfont}{\ABNTEXsubsectionfont\color{darkblue}}
\renewcommand{\cftsubsectionpagefont}{\ABNTEXsubsectionfont\color{black}}

\renewcommand{\cftsubsubsectionfont}{\ABNTEXsubsubsectionfont\color{darkblue}}
\renewcommand{\cftsubsubsectionpagefont}{\ABNTEXsubsubsectionfont\color{black}}

\renewcommand{\cftparagraphfont}{\ABNTEXsubsubsubsectionfont\color{darkblue}}
\renewcommand{\cftparagraphpagefont}{\ABNTEXsubsubsubsectionfont\color{black}}


