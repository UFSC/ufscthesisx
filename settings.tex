

% Thesis settings
%
\renewcommand{\mythesispreliminarylistname}{\chooselang{Brief Table of Contents}{Breve Sumário}}

% Informações de dados para CAPA e FOLHA DE ROSTO
\titulo{Modelo Canônico de TCC, Monografia,  Dissertação, Tese ou Relatório}
\subtitulo{Pós--Doutorado da UFSC com \abnTeX{}}
\autor{Nome do Autor}
\local{Florianópolis, Santa Catarina -- Brasil}
\data{\today}
\orientador{Prof. Dr. Nome do Orientador}
\coorientador{Prof. Dr. Nome do Coorientador}
\instituicao{Universidade Federal de Santa Catarina \newline Biblioteca Universitária}
\tipotrabalho{Tese de Doutorado}
\programa{Programa de Pós\hyp{}Graduação em Engenharia Elétrica -- PGEEL}
\centro{Departamento de Engenharia Elétrica e Eletrônica -- EEL}

% O preambulo deve conter o tipo do trabalho, o objetivo. O nome da instituição e a área de
% concentração.
\preambulo{Tese submetida ao Programa de Pós--Graduação em Engenharia Elétrica da Universidade
Federal de Santa Catarina para a obtenção do Grau de Doutor em Engenharia Elétrica. \showfont}

% Back references translation
\ifenglish
\else
    % Configurações de códigos fonte no documento
    \makeatletter
    \@ifpackageloaded{listings}
    {
        % Listing -> Codigo fonte
        \renewcommand{\lstlistingname}{Código--fonte}

        % List of Listings -> Lista de códigos-fonte
        \renewcommand{\lstlistlistingname}{Lista de códigos--fonte}

        % Calculate the size of the header
        \calculatelisteningsheader
    }
    \makeatother

    % Configurações do pacote backref, paginas com as citações na bibliografia
    \makeatletter
    \@ifpackageloaded{backref}
    {
        % Usado sem a opção hyperpageref de backref
        \renewcommand{\backrefpagesname}{Citado na(s) página(s):~}

        % Texto padrão antes do número das páginas
        \renewcommand{\backref}{}

        % Define os textos da citação
        \renewcommand*{\backrefalt}[4]
        {
            \ifcase #1
                Nenhuma citação no texto.
            \or
                Citado na página #2.
            \else
                Citado #1 vezes nas páginas #2.
            \fi
        }
    }{}
    \makeatother
\fi

% Espaçamentos entre linhas e parágrafos
%
% ifpackageloaded question
% https://tex.stackexchange.com/questions/70212/ifpackageloaded-question
\makeatletter
\@ifclassloaded{memoir}
{
    % Estilo de capítulos, ver classe para maiores detalhes.Veja outros estilos em:
    % http://mirrors.ibiblio.org/CTAN/macros/latex/contrib/memoir/memman.pdf
    \chapterstyle{VZ14}
    \setlength\beforechapskip{0pt}
    \setlength\midchapskip{15pt}
    \setlength\afterchapskip{15pt}

    % O tamanho do parágrafo é dado por:
    \setlength{\parindent}{1.3cm}

    % Controle do espaçamento entre um parágrafo e outro. Tente também
    % \onelineskip
    \setlength{\parskip}{0.2cm}
}{}
\makeatother

% Color settings across the document
\makeatletter
\@ifpackageloaded{xcolor}
{
    % RGB colors in absolute values from 0 to 255 by using `RGB` tag
    \definecolor{darkblue}{RGB}{26,13,178}

    % Definição de cores, RGB colors in percentage notation by using `rgb` tag
    \definecolor{mygreen}{rgb}{0,0.6,0}
    \definecolor{mygray}{rgb}{0.5,0.5,0.5}
    \definecolor{mymauve}{rgb}{0.58,0,0.82}

    % Configurações de aparência do PDF final
    \definecolor{figcolor}{rgb}{1,0.4,0}  % orange
    \definecolor{tabcolor}{rgb}{1,0.4,0}  % orange
    \definecolor{eqncolor}{rgb}{1,0.4,0}  % orange
    \definecolor{linkcolor}{rgb}{1,0.4,0} % orange
    \definecolor{citecolor}{rgb}{1,0.4,0} % orange
    \definecolor{seccolor}{rgb}{0,0,1}    % blue
    \definecolor{abscolor}{rgb}{0,0,1}    % blue
    \definecolor{titlecolor}{rgb}{0,0,1}  % blue
    \definecolor{biocolor}{rgb}{0,0,1}    % blue

    % Alterando o aspecto da cor azul
    \definecolor{blue}{RGB}{41,5,195}

    % Informações do PDF
    \@ifpackageloaded{hyperref}
    {
        \hypersetup
        {
            pdftitle={\@title},
            colorlinks=true, % false: boxed links; true: colored links
            linkcolor=darkblue, % color of internal links
            citecolor=darkgreen, % color of links to bibliography
            filecolor=black, % color of file links
            urlcolor=linkcolor,
            bookmarksdepth=4
        }
        \ifenglish
            \hypersetup
            {
                pdfauthor={Author},
                pdfsubject={Thesis' Abstract},
                pdfcreator={LaTeX with abnTeX2 for UFSC},
                pdfkeywords={abnt}{latex}{UFSC}{abntex2}{thesis},
            }
        \else
            \hypersetup
            {
                pdfauthor={Autores},
                pdfsubject={Resumo da tese},
                pdfcreator={LaTeX com abnTeX2 para UFSC},
                pdfkeywords={abnt}{latex}{UFSC}{abntex2}{tese},
            }
        \fi
    }
}{}
\makeatother

